\textbf{Recall:}
\[A/B=\set{x\in K(D)}{xB\subset A}\]
Is this the same as $AB^{-1}$? \\
\textbf{Answer:} No, because $B$ might not be invertible.

\thm Let $D$ be a domain, $K(D)$ its fraction field, $A$, $B$ two fractional ideals of $D$, with $B$ invertible.  Then
\[ A/B = AB^{-1} \]
\pf Clearly $B(A/B)\subset A$%everything in (A/B) multiplied by B is in A
, so $A/B\subset AB^{-1}$.

Conversely, say $x\in AB^{-1}$.  We want to show $x\in A/B$.  Well, $x\in AB^{-1}\implies xB\subset A$, so $x\in A/B$. \qed \\
\cor Let $I$ be an invertible ideal of a domain $D$.  Then $I^{-1}=D/I$.

\textbf{Warning:} If $B$ is not invertible, then $(A/B)B\neq A$, necessarily.

\eg Compute $(2,\sqrt{-5}+1)^{-1}$ in $\Z[\sqrt{-5}]=D$. \\
\soln Let $J=(2,1+\sqrt{-5})$.  If $a+b\sqrt{-5}\in J^{-1}$, then
\begin{gather}
2(a+b\sqrt{-5}) \in \Z[\sqrt{-5}] \tag{1} \\
\text{\emph{and }} (1+\sqrt{-5})(a+b\sqrt{-5}) \in \Z[\sqrt{-5}] \tag{2}
\end{gather}
\begin{gather*}
(1) \implies a,b\in\tfrac12\Z \\
(2) \implies \left\{\begin{gathered}
a-5b\in\Z \\
a+b\in\Z
\end{gathered}\right.
\end{gather*}
Write $a=\frac{c}{2}$, $b=\frac{d}{2}$.  Then $c-5d$ and $c+d$ are even.  This is equivalent to $c\equiv d\bmod2$:
\begin{align*}
a+b\sqrt{-5} &= \frac{c+(c+2k)\sqrt{-5}}{2} \qquad k\in\Z \\
&= c\paren*{\frac{1+\sqrt{-5}}{2}} + k\sqrt{-5}
\end{align*}
So guess: $J^{-1}=(\frac{1+\sqrt{-5}}{2})D+(\sqrt{-5})D=I$ \\
Check: $((\frac{1+\sqrt{-5}}{2})D+\sqrt{-5}D)(2D+(1+\sqrt{-5})D)=(1+\sqrt{-5})D+(-2+\sqrt{-5})D+(2\sqrt{-5})D+(-5+\sqrt{-5})D$
\begin{gather*}
\begin{aligned}
3 &= (1+\sqrt{-5})-(-2+\sqrt{-5}) \in IJ \\
-4 &= (1+\sqrt{-5})-(2\sqrt{-5})+(-5+\sqrt{-5}) \in IJ %\\
\end{aligned} \\
-(3+(-4)) \in IJ \\
\implies D \subset IJ
\end{gather*}
Since $IJ\subset D$, we get $IJ=D\implies I=J^{-1}$.

\eg Factor $(6)$ in $\Z[\sqrt7]$. \\
\soln $(6)=(2)(3)$. \\
Is $(2)$ prime?  Compute $\Z[\sqrt7]/(2)$: $\brace{0,1,\sqrt7,1+\sqrt7}$
\begin{gather*}
(\sqrt7)^2 = 7 \neq 0 \\
\sqrt7(1+\sqrt7) = 7 + \sqrt7 = 1 + \sqrt7 \neq 0 \\
(1+\sqrt7)^2 = 1 + 2\sqrt7 + 7 = 0 !
\end{gather*}
Consider $(2,1+\sqrt7)$.  Since $(1+\sqrt7)^2\equiv0\bmod(2)$, we're guessing that $(2)=(2,1+\sqrt7)^2$:
\begin{align*}
(2,1+\sqrt7)^2 &= (4,2+2\sqrt7,8+2\sqrt7) \\
&= (4,6,2+2\sqrt7,8+2\sqrt7) \\
&= (2)
\end{align*}
Is $(2,1+\sqrt7)$ prime?  Yes, because $\Z[\sqrt7]/(2,1+\sqrt7)\cong\Z/2\Z$ via $a+b\sqrt7\mapsto a+b\pmod2$.
So $(6)=(2,1+\sqrt7)^2(3)$ \\
Is $(3)$ prime?
%\[ \Z[\sqrt7]/(3) \cong \Z[x]/(x^2-7,3) \]
\begin{align*}
\Z[\sqrt7]/(3) &\cong \Z[x]/(x^2-7,3) \\
&\cong \Z_3[x]/(x^2-7) \\
&\cong \Z_3[x]/(x^2-1)
\end{align*}
This is not a domain, since $x^2-1$ is reducible.
$1\pm\sqrt7$ are zero divisors mod $3$:
\[ (1+\sqrt7)(1-\sqrt7) = -6 \equiv 0 \bmod 3 . \]
Compute $(3,1+\sqrt7)(3,1-\sqrt7)=(9,3+3\sqrt7,3-3\sqrt7,-6)=(3)$ \\
$(3,1\pm\sqrt7)$ is prime, because: \\
$\Z[\sqrt7]/(3,1\pm\sqrt7)\cong\Z_3$ via
\[ a + b \sqrt7 \mapsto a \mp b \bmod 3 \]
So $(6)=(2,1+\sqrt7)^2(3,1+\sqrt7)(3,1-\sqrt7)$.
