Note, the proof of the Arzela--Ascoli Theorem has these lines
\[ \begin{array}{cccccr}
f_1, & \dotsc, & f_n, & \dotsc \\
\ovalbox{$f_{11}$}, & f_{12}, & f_{13}, & f_{14}, & \dotsc, & \text{converging at $x_1$} \\
f_{21}, & \ovalbox{$f_{22}$}, & f_{23}, & f_{24}, & \dotsc, & \text{converging at $x_2$} \\
\vdots \\
f_{m1}, & f_{m2}, & \dotsc, & \ovalbox{$f_{mm}$}, & & \text{converging at $x_m$} \\
\vdots
\end{array} \]
Let $g_n=f_{nn}$. \\
Claim: $g_n$ is a subsequence of all $f_{m1}$, $f_{m2}$, $\dotsc$ \\
(From text page 300) \\
The claim should be modified as $g_n$, \emph{starting with the $m$th term}, is a subsequence of $f_{m1}$, $f_{m2}$, $f_{m3}$, $\dotsc$

\eg Consider the sequence of functions $f_n\colon[0,1]\to\R$ belonging to $C([0,1],\R)$ given by
\[ f_n(t) = \begin{cases}
0 & 1 \geq t \geq \tfrac1n \\
1-nt & 0 \leq t \leq \tfrac1n
\end{cases} \]\marginpar{figure of $f_n(t)$}%
\[ \norm{f_n}_\infty=1 \text{ for each $n$} \]
For each fixed $t$, the sequence \[\begin{cases}
f_n(t) \to 0 & \text{if $0<t$} \\
f_n(0) \to 1 & \text{if $0=t$}
\end{cases}\]
That is, $f_n$ tends to the function $\phi\colon[0,1]\to\R$
\[ \begin{cases}
\phi(t) = 0 & \text{if $t>0$ \emph{pointwise}} \\
\phi(t) = 1 & \text{otherwise}
\end{cases} \]
Is $\phi\in C([0,1],\R)$?  No.

Does $f_n$ converge to some function in the $C([0,1],\R)$ under $\norm{\cdot}_\infty$? \\
i.e., Does $f_n$ tends to some $f_n$ in $C([0,1],\R)$ uniformly? \\
No (uniform convergence implies pointwise convergent.)

Does $f_n$ has a convergent subsequence in $C([0,1],\R)$ under $\norm{\cdot}_\infty$? \\
No. \\
Let $\B=\set{f_n}{n\in\N}\subset C([0,1],\R)$.

$\B$ is not sequentially compact.  It is not compact (we are dealing with metric spaces).

Some conditions of the A--A theorem must fail. \\
$\B$ is clearly bounded, as $\norm{f_n}_\infty=1$.  Exercise: Is $\B$ weakly compact?  Is $\B$ equicontinuous?

Approximating continuous functions.

The $e^x$ can be approximated by finite polynomials on $[a,b]$ in the sense that for all $\epsilon>0$, there exists polynomial $p$ so that $\abs{f(x)-p(x)}\leq\epsilon$ for all $x\in[a,b]$
\begin{center}i.e., $\norm{f-p}_\infty<\epsilon$ in $C([a,b],\R)$.\end{center}
(Taylor series)

Question: Can a continuous function $f\colon[a,b]\to\R$ be approximated by a polynomial? \\
Theorem: (Weierstrass Approximation Theorem): Every $f\in C([a,b],\R)$ can be approximated by a polynomial $p\in C([a,b],\R)$.

Rephrased: The set of polynomials is dense in $C([a,b],\R)$. \\
See Theorem 5.8.1 (page 305). \\
Indeed the Bernstein polynomials
\[ p_n(x) = \sum_{r=0}^n \binom{n}{r} f\paren*{\frac{r}{n}} x^r (1-x)^{n-r} \]
is a sequence of polynomials approximating a continuous $f\colon[0,1]\to\R$ \\
i.e., $\norm{p_n-f}_\infty\to0$ as $n\to\infty$.
%Read Bernstein's proof
