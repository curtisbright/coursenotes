To complete the proof of the equivalence of the four statements, we now show that
\begin{enumerate}
\item[(4)] $f(\cl(A))\subset\cl(f(A))$
\end{enumerate}
implies (2): $f$ is continuous on $X$. \\
\pf Let $a\in X$ be given. \\
Let $u\in\U(f(a))$ be given.\marginpar{$f\colon X\to Y$\\topological spaces $X$ and $Y$\\figure: $a\mapsto f(a)$} \\
Without loss of generality, we may assume that $u$ is open. \\
Then $F\coloneqq u^c$ is closed and $f(a)\notin F$. \\
Consider $f^{-1}(u)$ which clearly contains $a$.  We need only to show that $f^{-1}(u)$ is a neighbourhood of $a$. \\
Observe that $f^{-1}(u)^c=f^{-1}(F)$. \\
In particular $f[\underbrace{f^{-1}(u)^c}_{=A\text{, say}}]\subset F$.\marginpar{Note: $f(f^{-1}(F))\subset F$.} \\
By assumption (4),
\[ f(\cl[f^{-1}(u)^c]) \subset \cl(f(A)) \]
Now, as $f(A)\subset F$ and $F$ is closed,
we have $\cl(f(A))\subset F$. \\
Hence $f(\cl[A])\subset F$. \\
So $\cl([A])\subset f^{-1}(F)=A$ by definition of pre-image
\[ \cl(A)\subset A \]
As $\cl(A)\supset A$ always, we get $\cl(A)=A$.  So $A$ is closed. \\
So $f^{-1}(u)=A^c$ is open. \\
So $f^{-1}(u)$ is a neighbourhood of $a$.

\thm Let $X$ be a set, $Y$ be a topological space and let $f\colon X\to Y$ be a mapping. \\
Then the set
\[ \T = \set{f^{-1}(G)}{\text{$G$ open in $Y$}} \]
is a topology on $X$.  Clearly, it is the smallest topology in $X$ with which $f$ is continuous. \\
\pf [Checking that $\T$ is indeed a topology on $X$.]
\begin{enumerate}[label=(\arabic*)]
\item $\bigcap_{i\in I}f^{-1}(G_i)$ (where $I$ is finite) $=f^{-1}(\bigcap_{i\in I}G_i)$, %by assignment
where $\bigcap_{i\in I}G_i$ is open.  Then $\T$ is closed under finite intersection.
\item Similarly $\T$ is closed under arbitrary union.
\end{enumerate}
\marginpar{figure: step function}
