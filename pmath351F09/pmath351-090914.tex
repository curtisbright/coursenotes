PM351 \\
Real Analysis \\
Prof.~Kathryn Hare \\
MC 5072

Office Hours \\
Wed 2:30--3:30 \\
Thursday 3--4

Wed Sept 16 \\
12--1:30 \\
DC 1302 \\
NSERC Scholarships \\
(due Sept 25)

\defin Two sets $A$ and $B$ have the same \emph{cardinality} (and write $\abs{A}=\abs{B}$) if there is a bijection between $A$ and $B$.

Say cardinality of $A$ is $\leq$ cardinality of $B$ (write $\abs A\leq\abs B$) if there is an injection: $A\to B$.

Cardinality is an equivalence relation:
\begin{enumerate}
\item $\abs A=\abs A$ (reflexive) (identity map)
\item $\abs A=\abs B\iff\abs B=\abs A$ (symmetric)
\item $\abs A=\abs B$ and $\abs B=\abs C$ $\implies$ $\abs A=\abs C$
\[ A \underset{\substack{\text{1--1}\\\text{onto}}}{\overset{f}{\to}} B \underset{\substack{\text{1--1}\\\text{onto}}}{\overset{g}{\to}} C \]
\[ g \circ f \colon A \to C \footnote{bijection} \]
\end{enumerate}
\textbf{Example:} Say $A$ has $n$ elements and $\abs A=\abs B$.  Here $f\colon A\to B$ is 1--1, onto. %\\
\begin{itemize}
\item[$\implies$] $B$ has at least $n$ elements, because $f$ is 1--1.
\item[$\implies$] $B$ has at most $n$ elements because $f$ is onto.
\item[] Thus $B$ has $n$ elements.
\end{itemize}
On the other hand, if $A$ and $B$ both have $n$ elements then there exists a bijection: $A\to B$. \\
Say $A=\brace{a_1,a_2,\dotsc,a_n}$, $B=\brace{b_1,b_2,\dotsc,b_n}$. \\
Define $f(a_j)=b_j$, bijection. \\
Therefore $\abs A=\abs B$.

\textbf{Example:} $\N\subseteq\Z\subseteq\Q\subseteq\R$ \\
$\abs\N\leq\abs\Z\leq\abs\Q\leq\abs\R$ \\
since the embedding maps are injections
\[f\qquad\begin{matrix}
\Z & 0 & 1 & -1 & 2 & -2 & 3 & -3 & \cdots \\
\N & 1 & 2 & 3 & 4 & 5 & 6 & 7 & \cdots
\end{matrix}\]
$f\colon\Z\to\N$ is a bijection, therefore $\abs\N=\abs\Z$.

\defin Say a set $A$ is \emph{countable} if it is either finite or $\abs A=\abs\N$.  Say $A$ is \emph{countably infinite} if countable and infinite. \\
$A$ is \emph{uncountable} if it is not countable. \\
e.g., $\Z$ is countable.

Countable sets can be written as $a_1,a_2,a_3,\dotsc$

Have $f\colon\N\to A$.  Put $a_j=f(j)$.

Conversely, if there is such a list then just define bijection $g\colon A\to\N$ by $g(a_j)=j$.

$\Q=\set{p/q}{p\in\Z\c q\in\N\c\text{$(p,q)$ coprime}}$,
\marginpar{figure: diagonal winding through $\N^2$}%
$\abs\Q=\abs\N$

e.g., $\abs{\N\times\N}=\abs\N$

\textbf{Problem:} $\abs{\R^2}=\abs\R$

e.g., Any countable union of countable sets is countable.  i.e.,
\[ A = \bigcup_{i=1}^\infty A_i \qquad \abs{A_i}=\abs\N \]
then $\abs A=\abs\N$

\pf
\begin{align*}
A_i &= \brace{a^{(i)}_1,a^{(i)}_2,a^{(i)}_3,\dotsc} \\
    &= \brace{a(i,1),a(i,2),\dotsc}
\end{align*}
\marginpar{figure: diagonal winding through $a(i,j)$}%
%
\prop If $\abs A\leq\abs\N$ then either $A$ is finite or $\abs A=\abs\N$. \\
\cor Hence any subset of a countable set is countable.
