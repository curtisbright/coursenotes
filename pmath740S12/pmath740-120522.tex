For $2\leq n\in\Z$, $n$ is a \emph{Carmichael number} when $n$ is composite and $b^n=b\bmod n$ for all $b\in\Z$.

\eg Show that $2\leq n\in\Z$ is a Carmichael number if and only if
\[ n = \prod_{i=1}^l p_i \]
for some distinct primes $p_1$, $\dotsc$, $p_l$ with $l\geq2$ such that $(p_i-1)\div(n-1)$ for all $i$. \\
\soln Suppose $n=p_1p_2\dotsm p_l$ where the $p_i$ are distinct primes with $l\geq2$ and $(p_i-1)\div(n-1)$ for all $i$. \\
Then $n$ is composite since $l\geq2$ and for $b\in\Z$, \\
for each $i$, if $p_i\div b$ then $b^n=0=b\bmod p_i$ \\
and if $p_i\ndiv b$ then $b^{p_i-1}=1\bmod p_i$ \\
so $b^m=1\bmod p_i$ for any multiple $m$ of $p_i-1$ \\
$\therefore b^{n-1}=1\bmod p_i$ \\
$\therefore b^n=b\mod p_i$ \\
Since $b^n=b\bmod p_i$ for all $i$ we have $b^n=b\bmod n$ by the CRT.

Conversely, suppose that $n$ is a Carmichael number.  Say $n=\prod_{i=1}^l p_i^{k_i}$ where $p_1$, $p_2$, $\dotsc$, $p_l$ are distinct primes with $l\geq2$ (since $n$ is composite) and each $k_i\geq1$. \\
Since $n$ is a Carmichael number \\
$b^n=b\bmod n$ for all $b\in\Z$ \\
$\therefore b^n=b$ for all $b\in U_n$ \\
$b^{n-1}=1\bmod n$ for all $b\in U_n$ \\
$\therefore n-1$ is a multiple of $\ord_n(b)$ for all $b\in U_n$ in particular, $n-1$ is a multiple of $\kappa(n)=\lcm(\kappa(p_i^{k_i}))$. \\
If we had $k_i=2$ for some $i$ then we would have $p_i\div\kappa(p_i^{k_i})$

Aside: for $p$ odd
\[ \kappa(p^k) = \phi(p^k) = p^{k-1}(p-1) \]
and
\[ \kappa(2) = 1\co\kappa(4)=2\co\kappa(2^n)=\tfrac12\phi(2^k)=2^{k-2} \]

but then since $\kappa(p_i^{k_i})\div n-1$ we would have $p_i\div(n-1)$ \\
but $p_i\div n$ so this is not possible. \\
So we have $k_i=1$ for all $i$, and $n=p_1p_2\dotsm p_l$ and $(n-1)$ is a multiple of $\kappa(n)=\lcm(\kappa(p_i))=\lcm(p_i-1)$ so $(p_i)\div(n-1)$ for all $i$

\ex If $n=p_1p_2\dotsm p_l$ is a Carmichael number then no $p_i=2$ and $l\geq3$.

\eg Note that $3\cdot11\cdot17=561$ is a Carmichael number since $2\div560$, $10\div560$, $16\div560$

\remark There are infinitely many Carmichael number.

\textbf{Miller's Test:} Let $n\geq3$ be odd.  Write $n-1=2^kq$ with $q$ odd.  If $n$ is prime
\begin{align*}
b^{n-1} &= 1 \bmod n \text{ for $1\leq b<n$} \\
b^{q2^k}-1 &= 0 \bmod n \\
0 = (b^q)^{2^k} &= ((b^q)^{2^{k-1}})(b^{q2^{k-1}}-1) \\
&= (b^{q2^{k-1}}+1)(b^{q2^{k-2}}+1)(b^{q2^{k-2}}-1) \\
0 &= (b^{q2^{k-1}}+1)(b^{q2^{k-2}}+1)\dotsm(b^{q2}+1)(b^q+1)(b^q-1) \bmod n
\end{align*}
(and $\Z_n$ is a field) and so either $b^q=1\bmod n$ or $b^{q2^i}=-1$ for some $i=0$, $1$, $\dotsc$, $k-1$

PRIMES is in P: $(x+b)^n=x^n+b\bmod\Z_n[x]/(x^r-1)$

\textbf{Chapter 9 Quadratic Residues} \\
Problem: given $a\in\Z_n$ or $U_n$ determine whether $a=x^2$ for some $x\in U_n$ \\
\defn $Q_n=\set{a\in U_n}{\text{$a=x^2$ for some $x\in U_n$}}$ is called the group of \emph{quadratic residues} modulo $n$

Note $Q_n$ is a group (since $1\in Q_n$, if $a=x^2$ and $b=y^2$ then $ab=(xy)^2$, and if $a=x^2$ then $a^{-1}=(x^{-1})^2$)

\note The bijection
\[ f\colon \Z_{ab} \to \Z_a \oplus \Z_b \]
given by $f(x)=(x,x)$ in the case that $\gcd(a,b)=1$ gives an isomorphism
\[ f\colon Q_{ab} \to Q_a \oplus Q_b \]
since if $u=x^2\in U_{ab}$ then $u=x^2\in U_a$ and $u=x^2\in U_b$ \\
and if $u=z^2\in U_a$ and $v\in y^2\in U_b$ \\
and if we solve $w=x\bmod a$ and $w=y\bmod b$ \\
then $w^2\in Q_{ab}$
\begin{align*}
\text{with } w^2 &= x^2 = u \in Q_a \\
w^2 &= y^2 = v \in Q_b
\end{align*}
\note $U_2=\brace1$, $Q_2=\brace1$ \\
$U_4=\brace{1,3}$, $Q_4=\brace{1}$ \\
$U_{2^k}=\brace{\pm5^i}$, $\abs{U_{2^k}}=2^{k-1}$ \\
$Q_{2^k}=\brace{+5^{2i}}$, $\abs{Q_{2^k}}=2^{k-3}$

for $p$ an odd prime $U_{p^k}=\chev{u}$ for some $u\in U_{p^k}$ and $Q_{p^k}=\chev{u^2}$ and $\phi(p^k)=p^{k-1}(p-1)$ is even so $\chev{Q_{p^k}}=\frac12\chev{U_{p^k}}=\frac12 p^{k-1}(p-1)$

Given $a\in U_{p^k}=\chev{u}$ \\
$a=u^k$ for some $k$
