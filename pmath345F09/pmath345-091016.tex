Proof that $\psi$ is well-defined.  Let $\frac as = \frac{a'}{s'}$.
\begin{align*}
\implies as' &= a's \\
\implies \phi(as') &= \phi(a's) \\
\implies \phi(a)\phi(s') &= \phi(a')\phi(s) \\
\implies \phi(a)\phi(s)^{-1} &= \phi(a')\phi(s')^{-1} \\
\implies \psi(\tfrac as) &= \psi(\tfrac{a'}{s'}) \text{, so $\psi$ is well-defined.}
\end{align*}
\check $\psi$ is a homomorphism \\
Now, show $\psi$ in injective.  Let $\frac as\in\ker\psi$
\begin{align*}
\implies \psi(\tfrac as) &= 0 \\
\implies \phi(a)\phi(s)^{-1} &= 0 \\
\implies \phi(a) &= 0 \text{, since $\phi(s)$ is a unit} \\
\implies a &= 0\footnote{since $\phi$ is an embedding} \implies \tfrac as = 0\text{, so $\psi$ is an embeddding}
\end{align*}
Now, we will show $\psi(\phi(a))=\phi(a)$
\begin{align*}
\psi(\phi(a)) &= \psi(\tfrac a1) \\
&= \phi(a) \phi(1)^{-1} \\
&= \phi(a) 1^{-1} \\
&= \phi(a) \text{, as required}
\end{align*}
Lastly, we will show $\psi$ in unique. \\
Suppose $\psi'\colon R_S\to T$ is an embedding such that $\psi'\circ\rho=\phi$.  Let $\frac as\in R_S$. \\
Then, $\psi'(\frac a1)=\psi'(\rho(a)) = \phi(a)$ \\
And, $1=\psi'(1)=\psi'(\frac s1\cdot\frac1s)=\psi'(\frac s1)\psi'(\frac1s)=\phi(s)\psi'(\frac1s)$, so $\psi'(\frac1s)=\phi(s)^{-1}$ \\
So, $\psi'(\frac a1)\psi'(\frac1s)=\phi(a)\phi(s)^{-1}$ \\
$\implies \psi'(\frac a1\cdot\frac1s)=\phi(a)\phi(s)^{-1}$ \\
$\implies \psi'(\frac as)=\phi(a)\phi(s)^{-1}$ \\
$\implies \psi'(\frac as)=\psi(\frac as)$.  So $\psi$ is unique.

\textbf{Convention:}~We usually identify $R$ with its image under $\rho$ in $R_S$, i.e., we view $R$ as a subring of $R_S$, with $a=\frac a1$ \\
\defin Suppose $R$ is an integer domain, and let $S=R\setminus\brace0$.  Then $R_S$ is called the \emph{field of fractions of $R$}, and we will denote it by $Q(R)$.

The obvious example is $Q(\Z)=\Q$. \\
\note $Q(R)$ is a field. \\
\pf Let $\frac ab\in Q(R)\implies a\in R\c b\neq0\in R$ \\
If $\frac ab\neq0$, then $a\neq0$, then $\frac ba\in Q(R)$ \\
And, $\frac ab\cdot\frac ba=\frac{ab}{ba}=\frac11=1$ \\
So $\frac ab$ is a unit.  Therefore $Q(R)$ is a field. \\
\ex Let $R$ be an integral domain.\\
$R[x]$ is an integral domain.
\begin{align*}
Q(R[x]) & = \set{f/g}{f,g\in R[x]\c g\neq0} \\
&\coloneqq R(x) \text{ called \emph{rational functions} on $R$}
\end{align*}
Perhaps later we will talk about $Q(R[[x]])$, called the set of \emph{Laurent series}. \\
\prop Let $R$ be a principal ideal domain, (respectively integral domain) and let $S\subseteq R$ satisfy the properties. \\
Then $R_S$ is a principal ideal domain.  (respectively integral domain) \\
\pf $R$ is not trivial $\implies$ $R_S$ is not trivial. \\
And, $R_S$ is commutative. \\
Suppose $\frac as,\frac bt\in R_S$
\begin{align*}
\frac{ab}{st} = 0 = \frac01 &\implies ab = 0 \\
&\implies a=0 \text{ or } b=0 \text{, since $R$ is an integral domain} \\
&\implies \tfrac as = 0 \text{ or } \tfrac bt = 0
\end{align*}
And, recall that principal ideal domains are all integral domains. \\
Let $I\subseteq R_S$ be an ideal in $R_S$. \\
Identify $R\subseteq R_S$, and let $I^*=I\cap R$. \\
\check $I^*$ is an ideal in $R$. \\
Thus, $I^*=cR$ for some $c\in R$ \\
Suppose $\frac as\in I$. \\
Then, $a=s(\frac as)\in I\cap R=I^*$ \\
$\implies a=cr$ for some $r\in R$ \\
$\implies \frac as=\frac{cr}s=c\frac rs\in cR$ \\
$\implies I\subseteq cR_S$ \\
And, since $c\in I$, $cR_S\subseteq I$. \\
Therefore $I=cR_S$, so $R_S$ is a principal ideal domain.
