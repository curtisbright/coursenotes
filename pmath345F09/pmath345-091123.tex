\defin $F$ field, $f\in F[x]$ irreducible is \emph{separable} if it has no repeated roots. \\
\cor $f\in F[x]$ irreducible, $f$ separable $\iff$ $f'\neq0$ \\
\pf $f$ separable $\implies$ $f'\neq0$ by the previous theorem \\
(in fact we showed $f'(a)\neq0$ for any root $a$ of $f$ in a splitting field of $f$.) \\
Suppose $f'\neq0$ and $f$ has repeated roots. \\
$\overset{\text{thm}}{\Longrightarrow}\gcd(f,f')\neq1$.  Since $f$ is irreducible, the prime factorization of $f$ is $f=cg$ where $c\in F\setminus\brace0$, $g\in F[x]$ monic irreducible \\
$\gcd(f,f')\neq1 \implies g\mid f' \implies f\mid f'$.  But $\deg f'\leq\deg f-1<\deg f$. \\
\cor $\Char(F)=0$, $f\in F[x]$ irreducible, then $f$ is separable. \\
\pf $f=a_0+a_1x+\dotsb+a_nx^n$, $n=\deg f$, $a_n\neq0$, $n>0$ \\
$f'=a_1+2a_2+\dotsb+na_n^{n-1}$ \\
$n\neq0$ in $F$ since $\Z$ embeds in $F$ \\
(i.e., $\underbrace{1+1+\dotsb+1}_\text{$n$ times}\neq0$ in $F$, $na_n=(1+\dotsb+1)a_n$) \\
$\implies na_n\neq0 \implies f'\neq0$. \\
\ex $\Z_2$, $t$ indeterminant
\begin{gather*}
\mathllap{L={}} \Z_2(t) \\
| \\
\mathllap{F={}} \Z_2(t^2)
\end{gather*}
$f\in F[x]$ \\
$f=-t^2+x^2$ \\
Since $t\notin F$ it's not hard to check that $t^2$ is prime in $F$.  Apply Eisenstein $\implies$ $f$ is irreducible $F[x]$ \\
In $L$,
\begin{align*}
f = x^2 - t^2 &= (x-t)(x+t) \\
&= (x-t)^2 \qquad \text{since $1=-1$ in $L$}
\end{align*}
$\Longrightarrow\;$ $f$ \emph{not} separable. \\
\note
\begin{itemize}
\item $f'=2x=0$ in $F$
\item $f=\text{minimal polynomial of $f$ over $F$}$
\end{itemize}

\textbf{10. Finite fields} \\
$F$ finite field \\
$\implies \Q \subsetneq F$ \\
$\implies \Char(F)\neq0$ \\
$\implies \Char(F)=p$, $p$ prime \\
$\Z_p \subseteq F$ \\
Since $F$ is finite $\implies$ $[F:\Z_p]\in\N$ \\
$\implies$ $F$ is an algebraic extension of $\Z_p$ \\
$F$ finite dimensional over $\Z_p$, say $\dim = n$ \\
$\implies$ As vector spaces $F\approx(\Z_p)^n$ \\
$\implies$ $\abs F=p^n$ \\
\prop $F$ finite field then $\Char(F)=p$, $p$ a prime \\
$F$ is a finite extension of $\Z_p$, and cardinality of $F$ is a power of $p$.

Suppose $\abs F=p^n=q$. \\
If $a\in F\setminus\brace0$,
\[ \brace{1,a,a^2,\dotsc,a^{q-1}} \subseteq F\setminus\brace0 \]
$\Longrightarrow\;$ $a^i=a^j$ for some $0\leq i<j\leq q-1$. \\
$\implies$ $a^{j-i}=1$, $0<j-i<q$

\defin $F$ finite field, $a\in F\setminus\brace0$ \\
The \emph{order of $a$}, $o(a)$, is the least positive integer $m$ such that $a^m=1$.
\begin{itemize}
\item[$\to$] Always exists by previous remarks, and $o(a)\leq q-1$
\[ q = p^n = \abs F \]
\end{itemize}
\lem $\abs F=p^n=q$. $a,b\in F\setminus\brace0$, $k>0$
\begin{enumerate}
\item[(a)] $a^k=1 \implies o(a)\mid k$
\item[(b)] $o(a^k) = \frac{o(a)}{\gcd(k,o(a))}$
\item[(c)] \emph{If} $\gcd(o(a),o(b))=1$ then $o(ab)=o(a)\cdot o(b)$.
\end{enumerate}
\pf
\begin{enumerate}
\item[(a)] $k=qo(a)+r$, $0\leq r<o(a)$
\[ 1 = a^k = (a^{o(a)})^q\cdot a^r = a^r \]
$\Longrightarrow\; r=0$ $\checkmark$
\item[(b)] $d=\gcd(k,o(a))$
\[ (a^k)^{o(a)/d} = a^{ko(a)/d} = (a^{o(a)})^{k/d} = 1 \]
$\overset{\text{(a)}}{\Longrightarrow} o(a^k)\mid \frac{o(a)}{d}$ \\
On the other hand,
\[ a^{k\cdot o(a^k)} = (a^k)^{o(a^k)} = 1 \]
$\underset{\text{(a)}}\Longrightarrow o(a)\mid k\cdot o(a^k)$ \\
$\implies \frac{o(a)}{d} \mid \frac{k}{d}\cdot o(a^k)$ \\
since $\gcd(\frac{o(a)}{d},\frac{k}{d})=1$ \\
$\implies \frac{o(a)}{d} \mid o(a^k)$ \\
Therefore $o(a^k)=\frac{o(a)}{d}$
\item[(c)] \begin{align*}
(ab)^{o(a)\cdot o(b)} &= a^{o(a)\cdot o(b)} \cdot b^{o(a)\cdot o(b)} \\
&= 1
\end{align*}
$\overset{\text{(a)}}\Longrightarrow o(ab)\mid o(a)o(b)$
\begin{align*}
a^{o(ab)\cdot o(b)} &= a^{o(ab)\cdot o(b)}\cdot b^{o(ab)\cdot o(b)} \\
&= (ab)^{o(ab)o(b)} = 1
\end{align*}
$\Longrightarrow\; o(a)\mid o(ab)\cdot o(b) \implies o(a)\mid o(ab)$ \\
Similarly $o(b)\mid o(ab)$. \\
Since $\gcd(o(a),o(b))=1$ \\
Therefore $o(a)o(b)\mid o(ab)$ \\
Therefore $o(ab)=o(a)o(b)$
\end{enumerate}
\thm $\abs F=p^n=q$
\begin{enumerate}
\item[(a)] There exists $a\in F\setminus\brace0$ such that $o(a)=q-1=\abs F-1$.
\item[(b)] Every element of $F$ is a root of $x^q-x\in F[x]$
\end{enumerate}
\cor $a\in F\setminus\brace0\implies o(a)\mid q-1$. \\
\pf $\text{Theorem (b)}\implies a^q=a \implies a^{q-1}=1$ \\
$\overset{\text{Lemma (a)}}\implies o(a)\mid q-1$. \\
\defin $a\in F$ is an \emph{primitive element} if $o(a)=\abs F-1$ \\
\remark If $a$ is primitive in $F$, then
\[ \brace{1,a,a^2,\dotsc,a^{q-2}}=F\setminus\brace0 \qquad q=\abs F \]
