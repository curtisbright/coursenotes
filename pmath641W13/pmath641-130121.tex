\textbf{Corollary 12:} Let $d$ be a squarefree integer.  The set of algebraic integers in $\Q(\sqrt{d})$ is given by
\begin{align*}
\set{a+b\sqrt{d}}{a,b\in\Z} & \text{ if $d\equiv 2 \text{ or } 3\pmod 4$} \\
\set[\Big]{\frac{a+b\sqrt{d}}{2}}{a,b\in\Z} & \text{ if $d\equiv1\pmod4$}
\end{align*}
\pf Suppose that $\alpha\in\Q(\sqrt{d})$ then $\alpha=r+s\sqrt{d}$ with $r$, $s\in\Q$.  Suppose that $\alpha$ is an algebraic integer.

First note that if $s=0$ then $r\in\Z$.  Suppose $s\neq0$.  Then observe that
\[ f(x) = (x-(r+s\sqrt{d}))(x-(r-s\sqrt{d}))=x^2-2rx+(r^2-ds^2) \]
is a monic polynomial over $\Q$ with $\alpha$ as a root.  Since $\alpha\notin\Q$, $f$ is the minimal polynomial of $\alpha$.  We need only check when $f\in\Z[x]$.  Note that $2r\in\Z$ so either $r\in\Z$ or $r=a/2$ with $a\in\Z$ and $a\equiv1\pmod2$.  In the first case then $r^2-ds^2\in\Z\implies ds^2\in\Z$.  But $d$ is squarefree and so $s\in\Z$.

In the second case $r=a/2$ and then
\[ r^2-ds^2 = a^2/4 - d s^2 \in \Z \implies s = b/2 \text{ with } b\equiv1\pmod 2 \]
and then
\[ \frac{a^2-db^2}{4} \in \Z \implies d\equiv 1 \pmod 4 \]
Objective: Prove \begin{enumerate}
\item[i)] the set of all algebraic integers forms a ring.
\item[ii)] For any finite extension $K$ of $\Q$ the set of algebraic integers in $K$, so $A\cap K$, forms a ring.
\end{enumerate}
For any $\alpha$, $\beta\in A$ we plan to show that $\alpha-\beta$ and $\alpha\beta$ are in $A$ since this shows $A$ is a subring of $\C$.

Let $\alpha=\alpha_1$, $\dotsc$, $\alpha_n$ be the conjugates of $\alpha$.  Let $\beta=\beta_1$, $\dotsc$, $\beta_m$ be the conjugates of $\beta$.

Consider $\Q(\alpha,\beta)$.  Let $\sigma_1$, $\dotsc$, $\dots_k$ be the embeddings of $\Q(\alpha,\beta)$ in $\C$ which fix $\Q$.  Then put $g(x)=\prod_{i=1}^k(x-\sigma_i(\alpha-\beta))$.  Note that $g$ is monic.  To prove $\alpha-\beta$ is an algebraic integer it suffices to prove $g\in\Z[x]$.  This can be done using the elementary symmetric polynomials but there is an easier approach.

\textbf{Theorem 13:} Let $\alpha\in\C$.  The following are equivalent:
\begin{enumerate}
\item[i)] $\alpha$ is an algebraic integer
\item[ii)] The additive subgroup of $\Z[\alpha]$ in $\C$ is finitely generated
\item[iii)] $\alpha$ is a member of some subring of $\C$ having a finitely generated additive group.
\item[iv)] $\alpha A\subseteq A$ for some finitely generated additive subgroup of $\C$.
\end{enumerate}
\pf i) $\implies$ ii) by Theorem 3 since
\[ \Z[\alpha] = \set{a_0+a_1\alpha+\dotsb+a_{n-1}\alpha^{n-1}}{a_j\in\Z} \]
where $n$ is the degree of $\alpha$ over $\Q$.

ii) $\implies$ iii) $\implies$ iv) immediate

Finally suppose iv) is true.  Since $A$ is a finitely generated additive subgroup of $\C$ there exist $a_1$, $\dotsc$, $a_n$ which generate $A$.  Since $\alpha A\subseteq A$ we see that for $i=1$, $\dotsc$, $n$
\[ \alpha a_i = m_{i,1}a_1 + \dotsb + m_{i,n}a_n \]
with $m_{i,1}$, $\dotsc$, $m_{i,n}\in\Z$.  Put $M=(m_{i,j})$.  Then
\[ (\alpha I_n-M)\begin{pmatrix}
a_1 \\
\vdots \\
a_n
\end{pmatrix} = \begin{pmatrix}
0 \\
\vdots \\
0
\end{pmatrix} \]
Since $(a_1,\dotsc,a_n)\neq(0,\dotsc,0)\implies\det(\alpha I_n-M)=0\implies$ $\alpha$ is a root of a monic polynomial with coefficients in $\Z$, hence is an algebraic integer.  Thus iv) $\implies$ i).

\textbf{Corollary 14:} If $\alpha$ and $\beta$ are algebraic integers then so are $\alpha-\beta$ and $\alpha\cdot\beta$. \\
\pf Suppose $\alpha$ has degree $n$ over $\Q$ and $\beta$ has degree $m$ over $\Q$ then $\Z[\alpha,\beta]$ is generated over $\Q$ by $\set{\alpha^i\beta^j}{i=0,\dotsc,n-1\co j=0,\dotsc,m-1}$.  Note $\alpha-\beta$ and $\alpha\beta$ are in the subring generated by this.  The result follows by Theorem 13 ((i), (iii)).

\textbf{Theorem 15:} If $\alpha$ is an algebraic number then there exists a positive integer $r$ such that $r\alpha$ is an algebraic integer. \\
\pf Since $\alpha$ is an algebraic number it is the root of a polynomial $f(x)=x^n+a_1x^{n-1}+\dotsb+a_n$ with $a_i\in\Q$.  Clear denominators to get that $\alpha$ is a root of a polynomial
\[ b_n x^n + \dotsb + b_0 \text{ with } b_i\in\Z . \]
Then note $b_n\alpha$ is a root of
\[ x^n + b_{n-1} x^{n-1} + \dotsb + b_0 b_n^{n-1} \]
and so $b_n\alpha$ is an algebraic integer.