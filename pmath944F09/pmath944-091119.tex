\textbf{Recall: }$C=\paren*{\begin{array}{@{}c|c@{}}
I & B
\end{array}}$, $B=\paren*{\begin{smallmatrix}0 & 1 & 1 & \cdots & 1 \\
1 & & & & \\
1 & & & & \\[-1ex]
\vdots & & \raisebox{1ex}{\rule{1ex}{0ex}\rlap{$A$}} & & \\
1 & & & & \end{smallmatrix}}$
%$B=\paren*{\begin{smallmatrix}0 & 1 & 1 & \cdots & 1 & 1 \\
%1 & & & & & \\
%1 & & & & & \\
%\vdots & & & A & & \\
%1 & & & & & \\
%1 & & & & & \end{smallmatrix}}$
%$B=\paren*{\begin{smallmatrix}0 & 1 & \cdots & 1 \\
%1 & & & \\
%\vdots & & A & \\
%1 & & & \end{smallmatrix}}$
\[ A = \begin{pmatrix}
1 & 1 & 1 & 0 & 1 & 1 & 0 & 1 & 0 & 0 & 0 \\
1 & 1 & 0 & 1 & 1 & 0 & 1 & 0 & 0 & 0 & 1 \\
%\vdots & \vdots & \vdots & \vdots & \vdots & \vdots & \vdots & \vdots & \vdots & \vdots & \vdots \\
%\vdots & & & & & & & & & & \vdots \\
%\multicolumn{11}{c}{\vdots} \\
 & & & & &\vdots& & & & & \\
0 & 1 & 1 & 1 & 0 & 1 & 1 & 0 & 1 & 0 & 0
\end{pmatrix} \]

We claim that the subspace $S$ of $\F_2^{24}$ in Proposition 28 is generated by the rows of $C$.  Let us consider the subspace $S_1$ generated by the rows of $C$.

First note that $(1,1,1,\dotsc,1)$ is in $S_1$ since we can obtain it by adding the rows of $C$ over $\F_2$.  Next we remark that the number of $1$s in any row of $C$ is either $8$ or $12$ and any two rows of $C$ are orthogonal since any two rows of $A$ have precisely 3 $1$s in common columns.

For any $24$-tuple $\s=(s_1,\dotsc,s_{24})$ in $\F_2^{24}$ we put $\norm\s$ equal to the number of coordinates of $\s$ which are $1$.  We prove first that if $\s$ is a linear combination of rows of $C$ then $\norm\s\equiv0\pmod4$.

To see this we remark that any two rows of $C$ are orthogonal so that if we add one row of $C$ to another to get a matrix $C'$ then the rows of $C'$ will be orthogonal.  If a row $\s_1$ is obtained by adding a row $\s$ of $C$ to a row $\r$ of $C$ then
\begin{equation} \norm{\s_1} = \norm{\r} + \norm{\s} - 2n \label{star091119} \end{equation}
where $n$ denotes the number of columns for which both entries are $1$.  Since $\r$ and $\s$ are orthogonal $n$ is even and since $\r$ and $\s$ are in $C$, $\norm\r$ and $\norm\s$ are in $\brace{8,12}$.  Thus $\norm\r\equiv0\pmod4$, $\norm\s\equiv0\pmod4$ and so by~\eqref{star091119}, $\norm{\s_1}\equiv0\pmod4$.  The result now follows by induction.

We are now in a position to prove that if $\s$ is a non-zero linear combination of the rows of $C$ then $\norm\s\geq8$.  Since $\norm\s\equiv0\pmod4$ it suffices to prove that $\norm\s\geq5$.

Suppose that $\s$ is a linear combination of $k$ elements of $C$.  If $k=1$ then the result follows since $\norm\s$ is $8$ or $12$.

If $k=2$ then since the rows of $A$ have exactly three columns with two $1$s and each row of $A$ has 6 $1$s we find that $\norm\s$ is again $8$ or $12$.

If $k=3$ then and $\s$ is the sum of the first row and two other rows then since the rows of $A$ have exactly three columns with two $1$s we see that $\norm\s=8$.  On the other hand if the three rows do not include the first row then the first 13 coordinates of $\s$ contain 4 $1$s.  If there are no other $1$s in $\s$ the sum of three rows of $A$ give the zero vector $(0,\dotsc,0)$ in $\F_2^{11}$ which does not happen.  Thus $\norm\s\geq5$ hence $\norm\s\geq8$.

If $k=4$ then we see that the first 12 coordinates of $\s$ have 4 $1$s.  If the remaining coordinates are $0$ then the sum of 4 rows of $B$ are $(0,0,\dotsc,0)$ which contradicts the fact that $B$ is non-singular; recall $B^2=BB^\text{tr}=I$.  Thus $\norm\s\geq5$ hence $\norm\s\geq8$.

Finally if $k\geq5$ then we get at least 5 $1$s in the first 12 coordinates and the result follows.  Thus completes the proof of Proposition 28.  Take $\s=\s_1$.

\textbf{The construction of the Leech lattice} \\ %\par
Let $\e_1=(1,0,\dotsc,0)$, $\dotsc$, $\e_{24}=(0,\dotsc,0,1)$ in $\R^{24}$ and we put $\b_i=\frac{1}{\sqrt8}\e_i$ for $i=1\c \dotsc\c 24$.

Let $L_0$ be the lattice generated by $\b_1\c \dotsc\c \b_{24}$ in $\R^{24}$.  Let $L$ be the sublattice of $L_0$ whose elements are of the form
\[ t_1\b_1 + \dotsb + t_{24}\b_{24} \]
where $t_1\c \dotsc\c t_{24}$ are integers satisfying either
\begin{enumerate}
\item[(i)] $t_1\c \dotsc\c t_{24}$ are even, $t_1+\dotsb+t_{24}\equiv0\pmod8$ and $\frac12(t_1,\dotsc,t_{24})$ reduced mod $2$ lies in the subspace $S$ of Proposition 28.
\end{enumerate}
or
\begin{enumerate}
\item[(ii)] $t_1\c \dotsc\c t_{24}$ are odd, $t_1+\dotsb+t_{24}\equiv4\pmod8$ and $\frac12(1+t_1,\dotsc,1+t_{24})$ reduced mod $2$ lies in the subspace $S$ of Proposition 28.
\end{enumerate}

Notice that $L$ is a lattice and hence a sublattice of $L_0$.  To see this note that $L$ contains 24 linearly independent vectors since it contains $8\b_1\c \dotsc\c 8\b_{24}$.  Further it is discrete since it is contained in $L_0$.  Next observe that if $\x\in L$ then $-\x\in L$ and if $\x$, $\y$ are in $L$ then $\x+\y\in L$ since $S$ is a subspace of $\F_2^{24}$.  $L$ is the Leech lattice.

We now show that if $\x$ is a %non-zero
vector in $L$ then $\x\cdot\x\equiv0\pmod2$.  If $\x=t_1\b_1+\dotsb+t_{24}\b_{24}$ then $\x\cdot\x=\frac18(t_1^2+\dotsb+t_{24}^2)$.  Thus we want to prove that $t_1^2+\dotsb+t_{24}^2\equiv0\pmod{16}$.

Consider first the case when $t_1\c \dotsc\c t_{24}$ are all even.  Then if $t_i\equiv0\pmod4$ then $t_i^2\equiv0\pmod{16}$ and if $t_i\equiv2\pmod4$ then $t_i^2\equiv4\pmod{16}$.  Recall that if $\s\in S$ then $\norm\s\equiv0\pmod4$ and so the number of indices $i$ for which $t_i\equiv2\pmod4$ is a multiple of $4$ and thus
\[ t_1^2 + \dotsb + t_{24}^2 \equiv 0 \pmod{16} . \]

On the other hand if $t_1\c \dotsc\c t_{24}$ are all odd then if $t_i\equiv\pm1\pmod8$ we have $t_i^2\equiv1\pmod{16}$ while if $t_i\equiv\pm3\pmod8$ we have $t_i^2\equiv9\pmod{16}$.  Let $\alpha_j$ be the number of $t_i$s with $t_i\equiv j\pmod8$.  Then
\begin{equation} t_1^2 + \dotsb + t_{24}^2 \equiv \alpha_1 + 9\alpha_3 + 9\alpha_5 + \alpha_7 \pmod{16} \label{zero091119} . \end{equation}
We also have
\begin{equation} 24 = \alpha_1 + \alpha_3 + \alpha_5 + \alpha_7 \equiv 0 \pmod 8 \label{one091119} \end{equation}
and, by the definition of $L$,
\begin{equation} \alpha_1 + 3\alpha_3 + 5\alpha_5 + 7\alpha_7 \equiv 4 \pmod 8 \label{two091119} . \end{equation}
Further, by Proposition 28,
\[ \alpha_1 + \alpha_5 \equiv 0 \pmod 4 \]
so
\begin{equation} 2(\alpha_1+\alpha_5)\equiv0\pmod8 \label{three091119} . \end{equation}
Adding \eqref{one091119} and~\eqref{two091119} and subtracting~\eqref{three091119} we find that
\[ 4(\alpha_3+\alpha_5) \equiv 4 \pmod 8 . \]
Thus $\alpha_3+\alpha_5$ is odd.  Therefore, by~\eqref{zero091119},
\[ t_1^2 + \dotsb + t_{24}^2 \equiv 24 + 8(\alpha_3+\alpha_5) \equiv 0 \pmod {16} \]
as required.
%
%Geometry of numbers \\
%Sign-up sheet for talks
%
%Tues, Dec.\ 1 \\
%9--10 AM: Steven Karp \\
%10--11 AM: Carmen Poruni \\
%1:30--2:30 PM: \\
%2:30--3:30 PM:
%
%Wednesday, Dec.\ 2. \\
%9--10 AM: \\
%10--11 AM: Reine Al-Housseini\\
%1:30--2:30 PM: Somit Gupta \\
%2:30--3:30 PM: David Tweedle \\
%3:30--4:30 PM: Leo Luong
%
%Thursday, Dec.\ 3 \\
%9--10 AM: \\
%10--11 AM: Pei Pei \\
%1:30--2:30 PM: \\
%2:30--3:30 PM:
%
%Caley:7
%Verouikak:2
%Karp:17
%Reine:21
%Ma: 3
%Brun:20
%Pei:9
%Kuong:10
%Tweedle:15
%Davis:16
%Alderson:5
