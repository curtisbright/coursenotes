\thm Every Euclidean domain is a pid. \\
\pf $I\subseteq R$, $R$ Euclidean domain $I\neq(0)$. \\
Let $N\colon R\to\N$ be a Euclidean norm on $R$. \\
Let $a\in I\setminus\brace0$ be of least norm. \\
\textbf{Show:} $I=(a)$.  Clearly $(a)\subseteq I$. \\
If not, let $b\in I\setminus(a)$. \\
Divide $b$ by $a$ to get
\begin{gather*}
b = aq + r \qquad q,r\in R \\
N(r) < N(a) \\
r = b - aq \in I
\end{gather*}
By minimality of $N(a)$ \\
$\implies r=0$ \\
$\implies b=aq$ \\
$\implies b\in(a)$ Contradiction. \\
Therefore $I=(a)$. \\
Therefore $R$ is a pid. \\
\cor $F[x]$ is a pid if $F$ is a field.

\defin $R$ integral domain. \\
$a,b\in R$, $a\mid b$ mean \emph{$a$ divides $b$} which means there is $r\in R$ such that $b=ar$. \\
(\note $a\mid b \iff b\in(a)\iff(b)\subseteq(a)$.) \\
(\note units divide everything: take $r=\frac ba$.  $0$ divides only $0$.) \\
A nonzero and nonunit $a\in R$ is called \emph{prime} if whenever $a\mid bc$, either $a\mid b$ or $a\mid c$. \\
A nonzero nonunit $a\in R$ is called \emph{irreducible} if whenever $a=bc$, either $a\mid b$ or $a\mid c$. \\
\ex In $\Z$, $\text{prime}=\text{irreducible}$ ($=\text{prime \#s}$)

\note $\text{prime}\implies\text{irreducible}$ \\
\ex ($\text{prime}\neq\text{irreducible}$) \\
$F$ field. $F[x]$. \\
$R\subseteq F[x]$ be the subring of polynomials with no linear term. \\
i.e., coefficient of $x$ is $0$.

\ex $R$ \emph{is} a subring of $F[x]$. \\
Consider $x^2$. \\
\textbf{Claim: } $x^2$ is irreducible in $R$. \\
\pf $x^2=fg$, $f,g\in R$ \\
$2=\deg f+\deg g$. \\
Since $f,g\in R$, $\deg f\neq1$, $\deg g\neq1$ \\
Without loss of generality, $f=a\in F\setminus\brace0$ \\
$g=\frac1ax^2$ \\
$\implies x^2\mid g$.

\textbf{Claim: } $x^2$ is not prime in $R$. \\
\pf $x^2\mid x^4\cdot x^2=x^6=x^3\cdot x^3$ \\
but if $x^2\mid x^3$ then $x^3=x^2f$ for some $f\in R$ \\
$\implies \deg f = 1$, contradiction.  So $x^2\nmid x^3$.

\prop If $R$ is a pid then $\text{prime}=\text{irreducible}$. \\
\pf Need $\text{irreducible}\implies\text{prime}$. \\
$a\in R$ be \emph{irreducible}.  Suppose $a\mid bc$.  Assume $a\nmid b$. \\
$I=(a)+(b)=\set{ar+bs}{r,s\in R}=(a,b)$ \\
$R$ pid $\implies$ $I=(d)$, for some $d\in R$. \\
$d\mid a$ and $d\mid b$ \\
$\Downarrow$ \\
$a=du$ for some $u\in R$ \\
$a \nmid d$ (else $a\mid b$) \\
$\implies$ $a\mid u$ as $a$ is irreducible \\
$\implies$ $u=ar$ for some $v\in R$ \\
$\implies$ $a=ard$ $\implies$ $1=vd$ $\implies$ $d$ is a unit \\
therefore $I=R$ \\
there exists $r,s\in R$
\begin{gather*}
ar + bs = 1 \\
acr + cbs = c
\end{gather*}
$a\mid cbs$ as $a\mid bc$ \\
$a\mid acr$ $\checkmark$ \\
$\implies a\mid c$\marginpar{[end of midterm material]}

\cor In $F[x]$, $\text{prime}=\text{irreducible}$, $F$ a field.

\defin $R$ integral domain is a \emph{Unique Factorization Domain} (UFD) if every nonzero nonunit is a product of primes.

\defin A ring $R$ is \emph{Noetherian} if there does not exist any infinite \emph{increasing} sequence of ideals.  i.e., \emph{cannot} have $I_1\subsetneq I_2\subsetneq I_3\subsetneq\dotsb$

\thm If $R$ is a Noetherian integral domain then every nonzero nonunit is a product of irreducibles.

\cor A noetherian pid is a ufd.

\lem pids are always noetherian.

\cor $\text{pid}\implies\text{ufd}$
