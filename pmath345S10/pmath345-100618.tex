How can one tell if $\brace{g_1,\dotsc,g_r}$ is a Gr\"obner basis? \\
\defn Let $f$, $g\in F[x_1,\dotsc,x_n]$ be two nonzero polynomials.  Then:
\[ \S(f,g) = \paren[\Big]{\frac{\LCM}{\LT(f)}}f-\paren[\Big]{\frac{\LCM}{\LT(g)}}g \]
where $\LCM=\LCM(\LM(f),\LM(g))$.

\thm (Buchberger's Criterion) Say $I=(f_1,\dotsc,f_r)$ is an ideal of $F[x_1,\dotsc,x_n]$.  Then $\brace{f_1,\dotsc,f_r}$ is a Gr\"obner basis for $I$ \emph{iff} for all $i$, $j$, $\S(f_i,f_j)$ gives zero remainder upon division by $\brace{f_1,\dotsc,f_r}$. \\
\pf Forwards is trivial.  Backwards is too hard. \qed

\eg Is $\brace{xy-1,y^2-1}$ a Gr\"obner basis?  By Buchberger's Criterion:
\begin{align*}
\S(xy-1,y^2-1) &= y(xy-1)-x(y^2-1) \\
&= xy^2-y-xy^2+x \\
&= x-y
\end{align*}
Clearly, a long division of $x-y$ by $\brace{xy-1,y^2-1}$ yields a remainder of $x-y$.  Since this is nonzero, we conclude that $\brace{xy-1,y^2-1}$ is not a Gr\"obner basis.

\thm (Buchberger's Algorithm) One can compute a Gr\"obner basis for $I=(f_1,\dotsc,f_r)$ by the following method:
\begin{enumerate}[label=(\arabic*)]
\item Compute $\S(f_i,f_j)$ and divide it by $\brace{f_1,\dotsc,f_r}$ for each $i$, $j$
\item If all remainders are zero, STOP; you have a Gr\"obner basis.
\item Otherwise, enlarge the set $\brace{f_1,\dotsc,f_r}$ by the nonzero remainders, and return to step (1).
\end{enumerate}
\pf This algorithm terminates because the ideal generated by $\brace{\LT(f_i)}$ strictly increases at each iteration, so by the ACC, the set of nonzero remainders must eventually be empty.  When this happens, Buchberger's Criterion implies that $\brace{f_i}$ is a Gr\"obner basis. \qed

\eg Find a Gr\"obner basis of $(xy-1,y^2-1)$.
\[ \S(xy-1,y^2-1)=x-y \]
This gives remainder $x-y$, so:
\begin{gather*}
\brace{xy-1,y^2-1,x-y} \\
\begin{aligned}
\S(xy-1,x-y) &= 1(xy-1)-y(x-y) \\
&= xy-1-xy+y^2 \\
&= y^2-1
\end{aligned}
\end{gather*}
This clearly gives remainder $0$, so we just need to check:
\begin{align*}
\S(y^2-1,x-y) &= x(y^2-1)-y^2(x-y) \\
&= xy^2 - x - xy^2 + y^3 \\
&= -x + y^3
\end{align*}
Long divide:
\[
%\begin{gather*}
%\begin{array}{ccc}
%xy-1 & y^2-1 & x-y \\ \hline
%& y & -1
%\end{array} \\
%-x+y^3 \\
%-x+y \\
%y^3-y \\
%y^3-y \\
%0 \\
\begin{array}{@{}r@{}r@{}r@{}r@{}r@{}r@{}r@{}}
& & & & \mathllap{0\co y\co-1} \\
\cline{2-5}
xy-1\co y^2-1\co x-y\rule{0pt}{1em}& \bigl) & -x & {}+{} & y^3 \\
& & -x & {}+{} & y & \\
\cline{3-5}
\rule{0pt}{1em}& & & & y^3 & {}-{} & y \\
& & & & y^3 &{}-{}& y \\
\cline{5-7}
\rule{0pt}{1em}& & & & & & 0
\end{array}
%\end{gather*}
\]
Zero remainder of all $\S$-polynomials implies (by Buchberger) that $\brace{xy-1,y^2-1,x-y}$ is a Gr\"obner basis.

Notice that $\LT(x-y)\mid\LT(xy-1)$ so:
\[ (\LT(xy-1),\LT(y^2-1),\LT(x-y)) = (\LT(y^2-1),\LT(x-y)) = \LT(xy-1,y^2-1) \]
Therefore, \emph{since $\brace{xy-1,y^2-1,x-y}$\/ is a Gr\"obner basis}, we see that $\brace{x-y,y^2-1}$ is also a Gr\"obner basis.

Any subset of $I$ that contains a Gr\"obner basis for $I$ is itself a Gr\"obner basis for $I$.

\defn Let $I\subset F[x_1,\dotsc,x_n]$ be a nonzero ideal.  Then $\brace{f_1,\dotsc,f_r}$ is a minimal Gr\"obner basis for $I$ \emph{iff}
\begin{enumerate}[label=(\arabic*)]
\item $\brace{f_1,\dotsc,f_r}$ is a Gr\"obner basis for $I$
\item $\LC(f_i)=1$ for all $i$
\item $\LT(f_i)\nmid\LT(f_j)$ for $i\neq j$ \\
$\iff\LT(f_i)\notin(\LT(f_j))_{j\neq i}$
\end{enumerate}
\eg $\brace{xy-1,y^2-1,x-y}$ is not minimal, because $\LT(x-y)\mid\LT(xy-1)$.
By deleting $f_i$ whose leading terms are redundant (\emph{i.e.,} divisible by some other leading term), we can always construct a minimal Gr\"obner basis from an arbitrary one.  Since Gr\"obner bases always exist, therefore, so do minimal Gr\"obner bases.

\eg $\brace{y^2-1,x-y}$ is a minimal Gr\"obner basis.  So is $\brace{y^2-1,x-y+\frac{1}{17}(y^2-1)}$.
