% Extra office hours
% Friday, Dec.\ 11 2:30--3:30/4 look for Room # on web page
% Monday, Dec.\ 14 10:30--12
%
\emph{Classical algebraic geometry} is the study of simultaneous solutions to systems of polynomial equations.

$K$ algebraically closed field. \\
$S\subseteq K[x_1,\dotsc,x_n]$ a set of polynomials
\[ V(S) = \set{(a_1,\dotsc,a_n)\in K^n}{f(a_1,\dotsc,a_n)=0\text{ for \emph{all} }f\in S} \]
\emph{affine variety in $K^n$ defined by $S$}

\note $V(S)=V(S\cdot K[x_1,\dotsc,x_n])$ where
\begin{align*}
S\cdot K[x_1,\dotsc,x_n] &= \text{ideal generated by $S$} \\
&= \set{g_1f_1+\dotsb+g_lf_l}{f_1,\dotsc,f_l\in S\c g_1,\dotsc,g_l\in K[x_1,\dotsc,x_n]}
\end{align*}
Therefore all affine varieties are of the form $V(I)$.

\textbf{Hilbert's Basis Theorem:} \\
$R$ commutative Noetharian ring $\implies$ $R[x]$ is also.

Hence $K[x_1,\dotsc,x_n]$ is Noetharian. \\
$\implies$ every ideal in $K[x_1,\dotsc,x_n]$ is finitely generated. \\
\begin{align*}
\text{Therefore } V(S) &= V(S\cdot K[x_1,\dotsc,x_n]) \\
&= V(f_1,\dotsc,f_l)
\end{align*}
where $S\cdot K[x_1,\dotsc,x_n]=(f_1,\dotsc,f_l)$. \\
Every affine variety is defined by a finite set of polynomials.

\defin Given any subset $X\subseteq K^n$
\[ I(X) = \set{f\in K[x_1,\dotsc,x_n]}{f(a_1,\dotsc,a_n)=0\text{ for all }(a_1,\dotsc,x_n)\in X} \]
This is an ideal, the \emph{ideal of $X$}.

\remarks $S\c T\subseteq K[x_1,\dotsc,x_n]$\quad $X\c Y\subseteq K^n$
\begin{enumerate}
\item[(a)] $S\subseteq T\implies V(T)\subseteq V(S)$ \\
$X\subseteq Y \implies I(Y)\subseteq T(X)$
\item[(b)] $S\subseteq I(V(S))$ \\
$X\subseteq V(I(X))$
\item[(c)] $V(S)=V(I(V(S)))$ \\
$I(X)=I(V(I(X)))$
\begin{itemize}
\item[$\to$] exercise
\end{itemize}
\end{enumerate}

\textbf{Hilbert's Nullstellensatz} \\
If $S\cdot K[x_1,\dotsc,x_n]$ is a \emph{proper} ideal then $V(S)\neq\emptyset$.

\textbf{case $n=1$:} $K[x]$ is a pid. \\
$S\cdot K[x]=(f)$\quad $f$ if not a \emph{unit} in $K[x]$ since the ideal is proper. \\
$\implies$ $V(S)=V(f)$ \\
\[ \implies\; \begin{gathered}
f=0 \mathrlap{\implies V(S)=K} \\
\text{or} \\
\deg f > 0 \mathrlap{\implies\text{since \emph{$K$ algebraically closed}}}
\end{gathered} \]
$f$ has a root, $a\in K$ \\
$\implies a\in V(S)$. \\
Note $V(K[x_1,\dotsc,x_n])=\emptyset$

Is $J=I(V(J))$ for all ideals $J$? \\
No. \\
$f\in K[x_1,\dotsc,x_n]$\quad $J=(f^2)$ \\
$f^2$ vanishes on $V(J)$ \\
$\implies$ $f$ vanishes on $V(J)$ \\
$\implies$ $f\in I(V(J))\setminus J$ \\
This is the only problem: \\
\thm If $J$ is an ideal in $K[x_1,\dotsc,x_n]$, then
\begin{align*}
I(V(J)) &= \set{f\in K[x_1,\dotsc,x_n]}{f^n\in J\text{ for some $n>0$}} \\
&= \Rad J
\end{align*}
\pf $\supseteq$ is clear.
\begin{align*}
f^n\in J &\implies f^n \text{ vanishes on $V(J)$} \\
&\implies f \text{ vanishes on $V(J)$} \\
&\implies f \in I(V(J))
\end{align*}
Conversely, $f\in I(V(J))$ \\
\textbf{Want:} $f\in\Rad J$ \\
We may assume $f\neq0$ \\
$\text{HBT}\implies J=(f_1,\dotsc,f_l)$ \\
Consider $K[x_1,\dotsc,x_n,y]$
\[ J' = (f_1,\dotsc,f_l,y\cdot f-1) \]
Suppose $(a_1,\dotsc,a_{n+1})\in V(J')$ \\
$\implies (a_1,\dotsc,a_n\in V(J))$
\begin{align*}
0 &= (y\cdot f-1)(a_1,\dotsc,a_{n+1}) \\
&= a_{n+1}\cdot \underbrace{f(a_1,\dotsc,a_n)}_{\mathclap{=0\text{ since }(a_1,\dotsc,a_n)\in V(J)}}-1 \\
&= -1
\end{align*}
Contradiction; therefore $V(J')=\emptyset$ \\
$\text{HN}\implies J'=K[x_1,\dotsc,x_n,y]$
\[ 1 = g_1f_1 + \dotsb + g_lf_l + h(yf-1) \text{ where } g_1,\dotsc,g_l,h\in K[x_1,\dotsc,x_n,y] \label{star}\tag{$*$} \]
\begin{gather*}
K[x_1,\dotsc,x_n,y] \overset{\phi}{\to} K(x_1,\dotsc,x_n) \\
g \mapsto g(x_1,\dotsc,x_n,1/f)
\end{gather*}
Apply $\phi$ to both sides of \eqref{star}
\[ 1 = g_1(x_1,\dotsc,x_n,1/f)f_1 + \dotsb + g_l(x_1,\dotsc,x_n,1/f)f_l + h(x_1,\dotsc,x_n,1/f)\cdot0 \]
\[ \implies 1 = g_1(x_1,\dotsc,x_n,1/f)f_1 + \dotsb + g_l(x_1,\dotsc,x_n,1/f)f_l \]
in $K(x_1,\dotsc,x_n)$ \\
clear denominators to get $N>0$, such that
\[ f^N = \overbrace{f^Ng_1(x_1,\dotsc,x_n,1/f)}^{\footnotemark}\footnotetext{in $K[x_1,\dotsc,x_n]$}f_1 + \dotsb + f^N g_l(x_1,\dotsc,x_n,1/f)\footnote{in $K[x_1,\dotsc,x_n]$}f_l \]
\emph{in} $K[x_1,\dotsc,x_n]$ \\
each $f^Ng_i(x_1,\dotsc,x_n,1/f)\in K[x_1,\dotsc,x_n]$ \\
$\implies f^N\in(f_1,\dotsc,f_l)=J$ \\
$\implies f\in\Rad J$

An ideal $J$ is \emph{radical} if $J=\Rad J$.

We get a 1--1, onto correspondence
\begin{gather*}
\text{Radical ideals in $K[x_1,\dotsc,x_n]$} \longleftrightarrow \text{affine varieties in $K^n$} \\
J \longmapsto V(J) \\
I(W) \longmapsfrom W
\end{gather*}
\begin{itemize}
\item[$\to$] exercises
\end{itemize}
%
%disciple_of_nuitari@hotmail.com
