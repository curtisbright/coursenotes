\claim The ideals of $\Z$ are precisely the sets $n\Z=\set{nr}{r\in\Z}$. \\
\pf First, $n\Z$ is an ideal by a quick check of the definition.  It only remains to show that every ideal is of the form $n\Z$.
Thus, say $I\subset\Z$ is an ideal.  It could be that $I=\brace0=0\Z$.  Otherwise, $I$ must contain some nonzero integer, which we may assume is positive.  Let $n$ be the smallest positive element of $I$.  We will show that $I=(n)=n\Z$.
Clearly $n\Z\subset I$, since $n\in I$.  Thus, $x\in I$.  We want to show $x\in n\Z$.  After long division:
\[ x = qn + r \]
where $q$, $r\in\Z$, $0\leq r<n$.  But $r=x-qn\in I$, so by minimality of $n$, we get $r=0$, and hence $x=qn\in n\Z$.
Thus, $I=n\Z$. \qed

\defn Let $R$ be a ring, $a_1$, $\dotsc$, $a_n\in R$ any elements.  The ideal generated by $a_1$, $\dotsc$, $a_n$ is:
\[ (a_1,\dotsc,a_n) = \set{r_1a_1+\dotsb+r_na_n}{r_1,\dotsc,r_n\in R} \]
It is easy to see that this is an ideal.

\eg $(6,8)\subset\Z$
\begin{align*}
&= \set{6a+8b}{a,b\in\Z} \\
&= \set{2(3a+4b)}{a,b\in\Z}
\end{align*}
so $2\in(6,8)$.  This immediately means that $(2)\subset(6,8)$.

Conversely, $6$, $8\in(2)$, so $(6,8)\subset(2)$, and hence $(2)=(6,8)$.

\fact Given an ideal $I$ and elements $a_1$, $\dotsc$, $a_n\in R$, if $a_1$, $\dotsc$, $a_n\in I$ then $(a_1,\dotsc,a_n)\subset I$.

\eg $(x,y)\subset\Q[x,y]$
\begin{align*}
(x,y) &= \set{xp(x,y)+yq(x,y)}{p,q\in\Q[x,y]} \\
&= \set{r(x,y)}{r(0,0)=0}
\end{align*}
%
\defn Let $I$, $J$ be ideals.  Then these are ideals:
\begin{align*}
I + J &= \set{a+b}{a\in I, b\in J} \\
\text{and } IJ &= \set{a_1b_1+\dotsb+a_nb_n}{a_i\in I, b_i\in J}
\end{align*}
\begin{align*}
(a_1,\dotsc,a_n) + (b_1,\dotsc,b_m) &= (a_1,\dotsc,a_n,b_1,\dotsc,b_m) \\
(a_1,\dotsc,a_n)(b_1,\dotsc,b_m) &= (a_1b_1,a_1b_2,\dotsc,a_1b_m,a_2b_1,\dotsc,a_2b_m,\dotsc,a_nb_1,\dotsc,a_nb_m) \\
&= (a_ib_j)_{\substack{i\in\brace{1,\dotsc,n}\\j\in\brace{1,\dotsc,m}}}
\end{align*}
\eg In $\Q[x,y]$:
\[ (x,y^2)\cdot(x-y,y^3-y)=(x^2-xy,xy^2-y^3,xy^3-xy,y^5-y^3) \]
If $R$ is a ring, then $R^*=\text{group of units of $R$}$

\thm Let $I$ be an ideal of a ring $R$.  Then $I=(1)=R$ \emph{iff} $I$ contains some unit of $R$. \\
\pf Forwards is trivial.  For backwards, assume $u\in I$ is a unit.  Then $1=uu^{-1}\in I\implies I=(1)$. \qed

\thm Let $R$ be a ring, $R\neq\brace0$.  Then $R$ is a field \emph{iff} it has exactly two ideals, $(0)$ and $(1)$. \\
\pf Forwards: Assume $R$ is a field, $I\subset R$ any ideal.  If $I=(0)$, we're done.  If not, $I$ contains some $x\in R$, $x\neq0$.  Since $R$ is a field, $x$ is a unit, so $I=(1)$.

Backwards: Let $x\in R$ be any nonzero element.  We want to show $x\in R^*$.  Well, $(x)\subset R$ is an ideal with $(x)\neq(0)$, so by assumption $(x)\neq(1)$.  This means $1\in(x)=\set{xr}{r\in R}$
\[ \implies 1 = rx \text{ for some $r\in R$} \]
so $x\in R^*$ and $R$ is a field. \qed

\textbf{Quotient rings} \\
Let $R$ be a ring, $I\subset R$ an ideal. (e.g., $R=\Z$, $I=(n)$) \\
We want to build a ring $R/I$ and a homomorphism $q\colon R\to R/I$ such that $\ker q=I$.

If we had such a thing, then $q(x)=q(y)\iff x-y\in\ker q=I$.

Thus, elements of $R/I$ ought to be equivalence classes of elements of $R$ under the equivalence relation
\[ x\equiv y \bmod I \quad\text{\emph{iff}}\quad x-y\in I . \]
