\textbf{Accumulation point:} $x\in X$ is an accumulation point of $U\subseteq X$ if $\forall r>0$, $B(x,r)\cap(U\setminus\brace x)\neq\emptyset$.

\ex \begin{enumerate}
\item $U=[0,1)\cup\brace2$ in $\R$ \\
Accumulation points of $U=[0,1]$\marginpar{figure: $U$ on real line}
\item $\Q$ in $\R$: All points of $\R$ are accumulation points.
\item $U=B(x_0,1)$ in $\R^2$ with any of these metrics $d_1,d_2,d_\infty$.\marginpar{figures: $y$ on boundard of $B(x_0,1)$}%
\\ Take $y\in\R^2$ with $d(x_0,y)=1$ \\
These points are accumulation points in all 3 cases. \\
Now let $U=B(x_0,1)$ in $X$. \\
Take $y\in X$ with $d(x_0,y)=1$. \\
Is $y$ an accumulation point of $U$? \\
Not if $X$ is the discrete metric space. \\
Take $B(y,1/2)=\brace y$: Does it intersect $U$?  No.
\item Any set $U$ in discrete metric space
\begin{itemize}
\item No point is an accumulation point since balls of radius $r\leq1$ are singletons
\end{itemize}
Every point in discrete metric space is isolated.
\item $\Z$: every point is isolated.
\end{enumerate}
\thm A set $U$ is closed if and only if $U$ contains all its accumulation points.

\cor \begin{enumerate}
\item Any finite set is closed
\item In the discrete metric space every set is closed
\item Any set with no accumulation points is closed.
\end{enumerate}
\pf ($\Longrightarrow$) Assume $U$ is closed.  Take $x\notin U$ and show $x$ is not an accumulation point of $U$. \\
$x\in U^\Co$ and this set is open.  Hence $\exists r>0$ such that $B(x,r)\subseteq U^\Co$.  Thus $B(x,r)\cap U=\emptyset$. \\
Therefore $x$ is not an accumulation point of $U$. \\
($\Longleftarrow$) Assume $U$ contains all its accumulation points. \\
Show $U^\Co$ is open.  Take $x\in U^\Co$.  By assumption $x$ is not an accumulation point of $U$.  Hence $\exists r>0$ such that $B(x,r)\cap U=\emptyset$, i.e., $B(x,r)\subseteq U^\Co$. $\implies$ $U^\Co$ is open $\implies$ $U$ is closed.

\textbf{Notation:} $\overline{A}=\text{closure of $A$}=A\cup\brace{\text{accumulation points of $A$}}$

\textbf{Notes:} If $A$ is closed then $\overline A=A$ \\
If $\overline A=A$ then all accumulation points of $A$ are in $A$, therefore $A$ is closed. \\
e.g., $\overline\Q$ in $\R$ is $\R$.

\thm \begin{enumerate}
\item $\overline A$ is a closed set
\item $\overline A=\bigcap_{\substack{\text{$B$ closed}\\ B\supseteq A}} B$
\end{enumerate}
\pf \begin{enumerate}
\item Show that $\overline{A}^\Co$ is open. \\
Let $x\in\overline A^\Co$.  Then $x$ is not in $A$ and even $x$ is not an accumulation point of $A$. \\
Then $\exists r>0$ such that $B(x,r)\cap A=\emptyset$. \\
Claim: $B(x,r)\cap\overline A=\emptyset$.  Say $y\in B(x,r)\cap\overline A$. \\
Then $y$ is an accumulation point of $A$.  Since $B(x,r)$ is an open set containing $y$, it would have to intersect $A$.  But we know it doesn't. \\
This proves the claim.
\[ \implies B(x,r)\subseteq\overline A^\Co \implies \overline A^\Co\text{ is open} \implies \overline A \text{ is closed} \]
\item exercise
\end{enumerate}
\defin $A\subseteq X$ is \emph{dense} if $\overline A=X$ \\
\defin $X$ is \emph{separable} if it has a countable dense set \\
e.g., $\Q$ is dense in $\R$ and $\R$ is separable \\
\textbf{Exercise:} Show $\R^n$ is separable for all $n$
\begin{enumerate}
\item $X$ discrete metric space: no proper subset is dense since every set is already closed.
\item If $A$ is closed and dense in $X$, what is $A$? (any metric space)
\[ \mathrel{\underbrace{A=}_\text{\clap{closed}}} {}\overline A{} \mathrel{\underbrace{=X}_\text{\clap{dense}}} \]
\end{enumerate}
\ex $c_0=\set{(x_n)^\infty_{n=1}}{x_n\to0}\subseteq l^\infty=\text{bounded sequences}$ \\
$d(x,y)=\sup_n\abs{x_n-y_n}$ \\
$l^1=\set{(x_n)}{\sum\abs{x_n}<\infty}\subseteq c_0$ \\
Show $l^1$ is dense in $c_0$. \\
Take $x=(x_n)\in c_0$ and consider $B(x,r)$ \\
Pick $N$ such that $\abs{x_n}<r$ for all $n\geq N$ and put $y=(x_1,x_2,\dotsc,x_N,0,0,\dotsc)$ \\
$y\in l^1$
\begin{align*}
d(x,y) &= \sup_n\abs{x_n-y_n} \\
&= \sup_{n>N}\abs{x_n-y_n}\footnotemark \\
&= \sup_{n>N}\abs{x_n} \\
&< r
\end{align*}\footnotetext{since $x_n=y_n$ for all $n\leq N$}%
This proves $x\in\overline{l^1}$.  Therefore $l^1$ is dense in $c_0$.

\defin $\Bdy A = \overline A \cap \overline{A^\Co}$

\begin{enumerate}
\item Ball in $\R^2$: our ``usual'' understanding of boundary
\item $\Bdy \Q\footnotemark = \R$
\item $\Bdy A$, where $A\subseteq X$ discrete metric space:
$\overline A = A$, $\overline{A^\Co}=A^\Co$ \\
therefore $\overline{A}\cap\overline{A^\Co}=A\cap A^\Co=\emptyset$
\end{enumerate}\footnotetext{$\subseteq\R$}
