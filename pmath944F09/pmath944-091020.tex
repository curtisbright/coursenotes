The \LLL-algorithm can be used to find a short vector in a lattice $\Lambda$.  We just put a basis for the lattice in reduced form $\b_1\c \dotsc\c \b_n$.  Then $\b_1$ is a short vector in $\Lambda$.

Let $\alpha_1\c \dotsc\c \alpha_n$ be in $\R$ and let $\epsilon$ be a real number with $0<\epsilon<1$.  How do we produce \emph{efficiently} a positive integer $q$ and integers $p_1\c \dotsc\c p_n$ for which
\[ \abs{q\alpha_i-p_i} < \epsilon \qquad\text{for $i=1\c \dotsc\c n$}, \]
with $1\leq q\leq 2^{n(n+1)/4}\epsilon^{-n}$?

If $\alpha_1\c \dotsc\c \alpha_n$ and $\epsilon$ are in $\Q$ then we can use \LLL\ to find $q$ in polynomial time in terms of the input.  First recall, by Theorem 7, on taking $\epsilon=\frac{1}{Q}$, that such a $q$ exists with
\[ 1 \leq q \leq \epsilon^{-n} . \]

We consider the lattice $\Lambda$ generated by the rows of the matrix
\[ \text{\begin{scriptsize}$n\mathord+1$\end{scriptsize}} \left\{ \vphantom{\begin{pmatrix}
1        & 0 & \cdots & 0 & 0 \\
0        & 1 &        & 0 & 0 \\
\vdots   &   & \ddots &   & \vdots \\
0        & 0 &        & 1 & 0 \\
\alpha_1 & \alpha_2 & \cdots & \alpha_n & \delta 
\end{pmatrix}} \smash{\overbrace{\begin{pmatrix}
1        & 0 & \cdots & 0 & 0 \\
0        & 1 &        & 0 & 0 \\
\vdots   &   & \ddots &   & \vdots \\
0        & 0 &        & 1 & 0 \\
\alpha_1 & \alpha_2 & \cdots & \alpha_n & \delta 
\end{pmatrix}}^{n+1}} \right. \vphantom{\overbrace{\begin{pmatrix}
1        & 0 & \cdots & 0 & 0 \\
0        & 1 &        & 0 & 0 \\
\vdots   &   & \ddots &   & \vdots \\
0        & 0 &        & 1 & 0 \\
\alpha_1 & \alpha_2 & \cdots & \alpha_n & \delta 
\end{pmatrix}}^{n+1}} \qquad\text{where $\delta=2^{-n(n+1)/4}\epsilon^{n+1}$} . \]
%\[ \overbrace{\mathllap{\text{\begin{scriptsize}$n\mathord+1$\end{scriptsize}} \left\{ \vphantom{\begin{pmatrix}
%1        & 0 & \cdots & 0 & 0 \\
%\vdots   &   &        &   & \vdots \\
%0        & 0 & \cdots & 1 & 0 \\
%\alpha_1 & \alpha_2 & \cdots & \alpha_n & \delta 
%\end{pmatrix}} \right. } \begin{pmatrix}
%1        & 0 & \cdots & 0 & 0 \\
%\vdots   &   &        &   & \vdots \\
%0        & 0 & \cdots & 1 & 0 \\
%\alpha_1 & \alpha_2 & \cdots & \alpha_n & \delta 
%\end{pmatrix}}^{n+1} \qquad\text{where $\delta=2^{-n(n+1)/4}\epsilon^{n+1}$} . \]
Note that $d(\Lambda)=\delta$.  By \LLL\ we can find a small non-zero vector $\b$ ($=\b_1$) in the lattice with
\[ \b = \paren{q\alpha_1-p_1,q\alpha_2-p_2,\dotsc,q\alpha_n-p_n,q\delta} \]
where $q$ and $p_1\c \dotsc\c p_n$ are integers.  Note that we may suppose that $q\geq0$ by replacing $\b$ by $-\b$ if necessary.  Further by Proposition 17 iii), we can find $\b$ with
\[ \abs{\b} \leq 2^{n/4} d(\Lambda)^{1/(n+1)} = 2^{n/4}\delta^{1/(n+1)} = 2^{n/4}\cdot 2^{-n/4} \epsilon = \epsilon . \]
Since $\abs{\b}\leq\epsilon$ and $\epsilon<1$ we see that $q\neq0$ since in that case $\abs\b=\abs{\paren{p_1,\dotsc,p_n,0}}\geq1$ since $p_1\c \dotsc\c p_n$ are not all zero as we have supposed $\b\neq\0$.  Thus
\[ 1 \leq q \leq 2^{n(n+1)/4} \epsilon^{-n} . \]

What if we want to find a small linear form with integer coefficients in $\alpha_1\c \dotsc\c \alpha_n$?  Given $\epsilon$ with $0<\epsilon<1$ how do we find efficiently integers $q_1\c \dotsc\c q_n$ and $p$ such that
\[ \abs{q_1\alpha_1+\dotsb+q_n\alpha_n-p} < \epsilon \]
and with
\[ 1 \leq \max_i \abs{q_i} \leq 2^{(n+1)/4} \epsilon^{-1/n} ? \]
Again by Theorem 7 the objective is best possible up to the factor $2^{(n+1)/4}$.

We consider the lattice $\Lambda$ generated by the rows of the matrix
\[ \begin{pmatrix}
1 & 0 & \cdots & 0 \\
\alpha_1 & \delta & & 0 \\
%\vdots & 0 & &        \\
\vdots &        & \ddots &        \\
\alpha_n & 0 &        & \delta
\end{pmatrix} \qquad\text{where $\delta=\paren*{\frac{\epsilon^{1/n}}{2^{1/4}}}^{n+1}$} . \]
A typical vector $\b$ in $\Lambda$ is of the form
\[ \paren{q_1\alpha_1+\dotsb+q_n\alpha_n-p,q_1\delta,q_2\delta,\dotsc,q_n\delta} \]
with $q_1\c \dotsc\c q_n$ and $p$ integers.  By \LLL\ we can find a non-zero vector $\b$ in $\Lambda$ of this form with $\abs\b\leq 2^{n/4} d(\Lambda)^{1/(n+1)}$, and since $d(\Lambda)=\delta^n=\paren*{\frac{\epsilon}{2^{n/4}}}^{n+1}$ we see that
\[ \abs\b \leq 2^{n/4} \cdot \frac{\epsilon}{2^{n/4}} = \epsilon . \]
Further since $\b\neq\0$ and $\epsilon<1$ we have $0<\abs\b<1$ hence $q_1\c \dotsc\c q_n$ are not all zero and so
\[ 0 < \max_i \abs{q_i} . \]

Finally suppose that $\alpha_{ij}$ ($1\leq i\leq n$, $1\leq j\leq m$) are all real numbers and that $\epsilon$ is a real number with $0<\epsilon<1$.  Consider the lattice $\Lambda$ generated by the rows of the matrix
\[ \begin{pmatrix}
1			&		&			&		&		&		\\
			&\ddots	&			&		&		&		\\
			&		&	1		&		&		&		\\
\alpha_{11}	&\cdots	&\alpha_{n1}&\delta	&		&		\\
\vdots		&		&\vdots		&		&\ddots	&		\\
\alpha_{1m}	&\cdots	&\alpha_{nm}&		&		&\delta
\end{pmatrix}\footnote{an $m+n\times m+n$ matrix where $\delta=(2^{-(n+m-1)/4}\cdot\epsilon)^{n/m+1}$} . \]
Note that $d(\Lambda)=\delta^m=(2^{-(n+m-1)/4}\cdot\epsilon)^{n+m}$.

By \LLL\ there is a non-zero vector $\b$ in $\Lambda$ with
\begin{align*}
\abs\b &\leq \delta^{m/(m+n)} 2^{(n+m-1)/4} \\
&= 2^{-(n+m-1)/4}\cdot\epsilon\cdot2^{(n+m-1)/4} = \epsilon .
\end{align*}
We have
\begin{multline*} \b = \paren{q_1\alpha_{11}+q_2\alpha_{12}+\dotsb+q_m\alpha_{1m}-p_1, \\
q_1\alpha_{21}+q_2\alpha_{22}+\dotsb+q_m\alpha_{2m}-p_2,\dotsc,q_1\alpha_{n1}+\dotsb+q_m\alpha_{nm}-p_n,q_1\delta,q_2\delta,\dotsc,q_m\delta}
\end{multline*}
with $q_1\c \dotsc\c q_m$ and $p_1\c \dotsc\c p_n$ integers.  Then
\[ \abs{q_1\alpha_{i1}+\dotsb+q_m\alpha_{im}-p_i}<\epsilon \qquad\text{for $i=1\c \dotsc\c n$} \]
and $\abs{q_j\delta}<\epsilon$ for $j=1\c \dotsc\c m$ so $\abs{q_j}<\delta^{-1}\epsilon=2^{(\frac{n+m-1}{4})(\frac{n+m}{m})}\epsilon^{-n/m}$.  Further, as before the $q_i$s are not all zero.

Theorem 7 tells us that we can make linear forms in the $\alpha_{ij}$s with integer coefficients which are simultaneously close to integers.  \LLL\ gives us an efficient method for finding the associated integer coefficients.  Can we do better than Theorem 7?  Not for real numbers in general, but for algebraic numbers $\alpha_{ij}$ we can say more.  It follows from work of Schmidt that:

\textbf{Theorem 20:} Let $1\c \alpha_1\c \dotsc\c \alpha_n$ be real algebraic numbers which are linearly independent over $\Q$.  Let $\delta>0$.  There are only finitely many $n$-tuples of non-zero integers $q_1\c \dotsc\c q_n$ with
\[ \abs{q_1\dotsm q_n}^{1+\delta} \norm{q_1\alpha_1+\dotsb+q_n\alpha_n} < 1 , \]
where for any real number $x$, $\norm x$ denotes the distance from $x$ to the nearest integer.

Applying Theorem 20 to all finite subsets of $\brace{\alpha_1,\dotsc,\alpha_n}$ we deduce the following:

\cor Let $1\c \alpha_1\c \dotsc\c \alpha_n$ be real algebraic numbers which are linearly independent over $\Q$.  Let $\delta>0$.  There are only finitely many $n+1$-tuples of integers $q_1\c \dotsc\c q_n\c p$ with $q=\max_i\abs{q_i}>0$ for which
\[ \abs{\alpha_1q_1+\dotsb+\alpha_nq_n-p} < \frac{1}{q^{n+\delta}} . \]
Note the special case $n=1$ is Roth's Theorem.
