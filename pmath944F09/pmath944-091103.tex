Consider the lattice $D_4(\sqrt2)$ in $\R^4$.  It has diagram \setlength{\unitlength}{0.5cm}\phantom{$\Biggm|$}\begin{picture}(1.707,0)(0,-0.25)\put(0,0){\circle*{0.2}}\put(1,0){\circle*{0.2}}\put(1.707,0.707){\circle*{0.2}}\put(1.707,-0.707){\circle*{0.2}}\put(0,0){\line(1,0){1}}\put(1,0){\line(1,1){0.707}}\put(1,0){\line(1,-1){0.707}}\end{picture}\phantom{$\Biggm|$} %\setlength{\unitlength}{0.75cm}
and each basis vector has length $\sqrt2$.  In fact $D_4(\sqrt2)$ can be represented as the lattice $\Lambda_1$ in $\R^4$ which is the sublattice of $\Lambda_0$ given by the congruence condition: $(u_1,u_2,u_3,u_4)$ is in the lattice $\iff$ $u_1+u_2+u_3+u_4\equiv0\pmod2$.  One can check that this lattice is generated by
\[ (2,0,0,0)\c  (1,1,0,0)\c  (1,0,1,0)\c  (1,0,0,1) . \]
As a consequence $d(\Lambda_1)=\abs*{\det\paren*{\begin{smallmatrix}2&0&0&0\\1&1&0&0\\1&0&1&0\\1&0&0&1\end{smallmatrix}}}=2$.  Equivalently it is generated by $(1,0,0,1)$, $(1,0,1,0)$, $(1,0,0,-1)$ and $(0,1,1,0)$.  Notice that
\[ \overbrace{\begin{pmatrix}
1 & 0 & 0 & 1 \\
1 & 0 & 1 & 0 \\
1 & 0 & 0 & -1 \\
0 & 1 & 1 & 0
\end{pmatrix}}^B \cdot
\overbrace{\begin{pmatrix}
1 & 1 & 1 & 0 \\
0 & 0 & 0 & 1 \\
0 & 1 & 0 & 1 \\
1 & 0 & -1 & 0
\end{pmatrix}}^{B^{\text{tr}}} = \begin{pmatrix} 2 & 1 & 0 & 0 \\ 1 & 2 & 1 & 1 \\ 0 & 1 & 2 & 0 \\ 0 & 1 & 0 & 2 \end{pmatrix} \]
Thus $\Lambda_1$ is a representation for $D_4(\sqrt2)$.

We now put a sphere of radius $\frac{\sqrt2}{2}=\frac{1}{\sqrt2}$ around each lattice point in $\Lambda_1(D_4(\sqrt2))$.  Notice that any two lattice points in $\Lambda_1$ differ by a vector of length at least $\sqrt2$.  Thus the spheres may touch but they do not overlap in a set of positive volume.  Consider the sphere around $(0,0,0,0)$.

It is surrounded by several spheres which touch it.  They are $(\pm1,\pm1,0,0)$, $(\pm1,0,\pm1,0)$, $(\pm1,0,0,\pm1)$, $(0,\pm1,\pm1,0)$, $(0,\pm1,0,\pm1)$, $(0,0,\pm1,\pm1)$.  Thus the central sphere is surrounded by $\binom{4}{2}\cdot4=24$ spheres which touch it.  Recently (2003) Oleg Musin proved that there is no configuration of 25 spheres of equal radius which touch a central sphere of the same radius without overlap in $\R^4$.

\defin The kissing number $\tau_n$ for $n=1\c 2\c \dotsc$ is defined to be the maximum number of unit spheres in $\R^n$ which can touch a central unit sphere so that their interiors do not overlap.

Thus $\tau_4\geq24$ by the example and $\tau_4\leq24$ by the result of Musin.  Plainly $\tau_1=2$ and the hexagonal packing %figure:circles
\setlength{\unitlength}{0.4cm}
\[ \begin{picture}(6,5.46)(-3,-2.73)
\put(0,0){\circle{2}}
\put(2,0){\circle{2}}
\put(-2,0){\circle{2}}
\put(1,1.73){\circle{2}}
\put(1,-1.73){\circle{2}}
\put(-1,1.73){\circle{2}}
\put(-1,-1.73){\circle{2}}
\end{picture} \]
gives $\tau_2=6$.  It is not so clear what $\tau_3$ is at first glance.  The standard cannonball packing gives $\tau_3\geq12$.  There was a dispute between Newton and Gregory as to whether $\tau_3$ was $12$ or $13$.  The first correct proof that $\tau_3=12$ is due to Schutte and van der Waerden in 1953.

\defin A sphere packing of $\R^n$ is a collection of spheres in $\R^n$ of equal radius whose interiors do not overlap.  If the centres of the spheres occur at the points of a lattice we say that the packing is a lattice packing (of spheres).

Given a sphere packing in $\R^n$ let $\rho$ be the radius of the sphere and define $\Delta$, the packing density, in the following way.  For any real number $x$ let $S_x$ be a sphere of radius $x$ in $\R^n$.  We put
\[ \Delta = \varlimsup_{R\to\infty} \frac{\paren*{\substack{\text{the number of spheres in the}\\\text{collection of radius $\rho$ inside $S^{(0)}_R$}}}\cdot\volume(S_\rho)}{\volume(S^{(0)}_R)}\footnotemark . \]\footnotetext{Let $S^{(0)}_R$ be the sphere of radius $R$ centred at the origin.}%
$\Delta$ measures the ``proportion'' of $\R^n$ covered by the spheres in the sphere packing.

We now define $\Delta_n$ for $n=1\c 2\c \dotsc$ by
\[ \Delta_n = \smashoperator{\sup_{\substack{\text{sphere packing}\\\text{in $\R^n$}}}}\Delta ; \]
here the $\sup$ is taken over all sphere packings in $\R^n$.  Similarly
\[ \Delta_n(L) = \smashoperator{\sup_{\substack{\text{lattice packing}\\\text{in $\R^n$}}}}\Delta ; \]
here the $\sup$ is taken over all sphere packings in $\R^n$ which are lattice packings.  Notice that if $L$ is a lattice then the largest radius $\rho_0$ of spheres in a sphere packing associated with the lattice is $\frac12$ the minimal non-zero distance between points in the lattice.  If we consider the lattice packing of spheres of radius $\rho_0$ around each lattice point of $\Lambda$ then
\[ \Delta(\Lambda) = \frac{\volume S_{\rho_0}}{\text{volume fundamental region of $\Lambda$}} = \frac{\volume S_{\rho_0}}{d(\Lambda)} . \]

Certainly $\Delta_n\geq\Delta_n(L)$ for $n=1\c 2\c \dotsc$.  In fact $\Delta_1=\Delta_1(L)$, $\Delta_2=\Delta_2(L)$.  For $n=2$ the hexagonal lattice yields $\Delta_2$.  We have
\[ \Delta_2 = \Delta_2(L) = \frac{\pi(\frac12\sqrt[4]{4/3})^2}{1} = \frac{\pi}{\sqrt{12}} = 0.9069\ldots . \]

Let us compute the packing density $\Delta$ of $D_4$.  Since the minimal non-zero distance between two lattice points in $D_4(\sqrt2)$ is $\sqrt2$ we may take $\rho_0=\frac12\sqrt2=\frac{1}{\sqrt2}$ and we have
\[ \Delta(D_4) = \frac{\frac{\pi^2}{2}\paren*{\frac{1}{\sqrt2}}^4}{2} = \frac{\pi^2}{16} = 0.6169\ldots . \]
This is the largest lattice packing density known in $\R^4$.

It was proved by Korkine and Zolotareff in 1872 that
\[ \Delta_4(L) = \Delta_4(D_4) . \]
%
%veronika-kl@ya.ru
\vspace{-\baselineskip}