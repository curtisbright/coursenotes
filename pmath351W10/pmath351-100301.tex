%9:00	10
%10:00	13
%11:00	14
%12:00	10
%1:00	12
%2:00	8
%3:00	11
%4:00	15
New Midterm: Tuesday, 16 March, 2010 at 4:00--5:30 PM

Proof of the Bolzano--Weierstrass Theorem (page 165 text) \\
Let $A$ be compact.  Assume, to the contrary that $A$ is not sequentially compact, that there exists a sequence $x_k\in A$ which has no convergent subsequence.

In particular, the sequence has \emph{infinitely many distinct} points $y_1$, $y_2$, $\dotsc$, $y_n$, $\dotsc$.

Claim: $\brace{y_1,y_2,\dotsc,y_n,\dotsc}$ is closed.

\pf Let $a\in A$, $a\notin\brace{y_1,\dotsc,y_n,\dotsc}$.  If $a$ were a limit point of $\brace{y_1,\dotsc,y_n,\dotsc}$ then every neighbourhood of $a$ will meet this set.  Hence, by picking elements\marginpar{figure: $x_{n_1}$, $x_{n_2}$ in neighbourhood of $a$, $n_2>n_1$} %carefully
in the intersection of $D(a,1/n)$ with the set $\brace{y_1,\dotsc,y_n,\dotsc}$, we get a convergent subsequence of $x_k$ which converges to $a$.  This would contradict that $x_k$ has no convergent subsequence. %\\

Therefore $\brace{y_1,\dotsc,y_n,\dotsc}$ is compact. (``closed subsets of a compact space $A$ is compact'').

Claim: Each element of $\brace{y_1,\dotsc,y_n,\dotsc}$ is an \emph{isolated point} of the set, i.e., to each $y_i$, there exists a positive $\delta$ such that $D(y_i,\delta)$ does not meet $\brace{y_1,\dotsc,y_n,\dotsc}$ at any point other than $y_i$.

Consider the open cover of $\brace{y_1,\dotsc,y_n,\dotsc}$
\[ \sC = \set{D(y_i,\delta_i)}{i=1,2,\dotsc} . \]
This $\sC$ has no finite subcover.  It contradicts the compactness of $\brace{y_1,\dotsc,y_n,\dotsc}$.  The above proves that compact $A$ is sequentially compact.

Next, assume that $A$ is sequentially compact.  Let $\sC$ be a given open cover of $A$.

Claim: There exists $r>0$ such that for each $y\in A$, $D(y,r)\subset U$ for some $U\in\sC$. \\
$\ldots$ Read the book.
