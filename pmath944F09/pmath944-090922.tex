\textbf{Theorem 2:} Let $\Lambda$ be a lattice.  The polar lattice of $\Lambda$ consists of the vectors $v^*$ such that $v^*\cdot v$ is an integer for all $v$ in $\Lambda$.  $\Lambda$ is the polar lattice of $\Lambda^*$.
\[ d(\Lambda)d(\Lambda^*) = 1 \]
\pf Let $v_1\c \dotsc\c v_n$ be a basis for $\Lambda$ and let $v^*_1\c \dotsc\c v^*_n$ be a basis for $\Lambda^*$. \\
If $v$ is in $\Lambda$ then there exist integers $a_1\c \dotsc\c a_n$ such that \[v=a_1v_1+\dotsb+a_nv_n\]
while if $v^*$ is in $\Lambda^*$ then there exist integers $b_1\c \dotsc\c b_n$ such that
\[ v^* = b_1v^*_1 + \dotsb + b_nv^*_n . \]
In particular
\[ v^*\cdot v = \sum_{i=1}^n a_i b_i \]
which is an integer.

Now let $w$ be a vector for which $w\cdot v$ is an integer for all $v$ in $\Lambda$.  Then there exist integers $c_1\c \dotsc\c c_n$ such that $w\cdot v_i=c_i$ for $i=1\c \dotsc\c n$. \\
Put $v^*=c_1v^*_1+\dotsb+c_nv^*_n$ so $v^*\in\Lambda^*$.  But then
\[ (w-v^*)\cdot v_i = 0 \qquad\text{for $i=1\c \dotsc\c n$.} \]
But $v_1\c \dotsc\c v_n$ are linearly independent in $\R^n$ and so $w=v^*$ hence $w\in\Lambda^*$.

By what we have just proved we now see that $\Lambda$ is the polar lattice of $\Lambda^*$.  Finally,
\[ \det(v^*_1,\dotsc,v^*_n)\cdot\det(v_1,\dotsc,v_n) = 1, \]
and so
\[ d(\Lambda^*)d(\Lambda) = 1 . \]

Notice that if $w=(y_1,\dotsc,y_n)$ is in $\R^n$ the set of $x=(x_1,\dotsc,x_n)\in\R^n$ for which $x\cdot w=0$ is given by $(x_1,\dotsc,x_n)$ for which
\[ x_1y_1 + \dotsb + x_ny_n= 0 \]
and so it determines a hyperplane in $\R^n$.

\textbf{Proposition 3:} Let $\Lambda$ be a lattice in $\R^n$ and let $u$ be a vector in $\R^n$.  There exist $n-1$ linearly independent vectors $w_1\c \dotsc\c w_{n-1}$ in $\Lambda$ with $u\cdot w_i=0$ for $i=1\c \dotsc\c n-1$ if and only if $u=t\cdot w^*$ with $t\in\R$ and $w^*\in\Lambda^*$.

\pf $\Longrightarrow$: By Corollary 1 of Theorem 1 there is a basis $v_1\c \dotsc\c v_n$ of $\Lambda$ such that
\[ w_i = a_{i1}v_1 + \dotsb + a_{ii}v_i \qquad\text{with}\qquad a_{ij}\in\Z \quad\text{and}\quad a_{ii}\neq0 \]
for $i=1\c \dotsc\c n-1$.  Since $u\cdot w_i=0$ for $i=1\c \dotsc\c n$ we see that $u\cdot v_i=0$ for $i=1\c \dotsc\c n-1$.  Put $u\cdot v_n=t$, for some $t\in\R$.  Observe that if $v^*_1\c \dotsc\c v^*_n$ is a polar basis for $\Lambda^*$ then $u=tv^*_n$ as required.

$\Longleftarrow$: If $w^*=\0$ then $u=\0$ and so $u\cdot w_i=0$ for $i=1\c \dotsc\c n-1$.  Suppose $w^*\neq(0,\dotsc,0)$.  Put $w^*=m\cdot v^*_1$ where $m$ is a positive integer and $v^*_1$ is such that $\frac1k\cdot v^*_1$ is not in $\Lambda^*$ for any integer $k$ with $k\geq2$. ($v^*_1$ is said to be primitive for $\Lambda^*$.)  By Corollary 2 of Theorem 1 we can extend $v^*_1$ to a basis $v^*_1\c v^*_2\c \dotsc\c v^*_n$ of $\Lambda^*$.  Let $v_1\c \dotsc\c v_n$ be a basis for the polar lattice $\Lambda$ of $\Lambda^*$.  Then $v^*_1\cdot v_j=0$ for $j=2\c \dotsc\c n$ and so $w^*\cdot v_j=0$ for $j=2\c \dotsc\c n$ as required.

\textbf{Remark:} It follows from the proof of Proposition 3 that if $w^*\in\Lambda^*$ then we can associate to it a lattice $\Lambda(w^*)$ in $\R^{n-1}$ (with basis $v_2\c \dotsc\c v_n$).

Let $U$ be the unit interval given by
\[ U = \set{t\in\R}{0\leq t<1}, \]
and let $U^n$ be the unit $n$-cube given by
\[ U^n = \set{(x_1,\dotsc,x_n)\in\R^n}{0\leq x_i<1\text{ for }i=1\c \dotsc\c n} . \]
Let $\overline U^n$ denote the closure of $U^n$.  For $\x=(x_1,\dotsc,x_n)\in\R^n$ we denote
\[ \house{\x} = \max_{i=1,\dotsc,n}\abs{x_i} . \]
This is known as the house of $\x$.  If $\x=(x_1,\dotsc,x_n)\in\Lambda_0$ then we say that $\x$ is an integer point.  For any set $T$ in $\R^n$ and $\x$ in $\R^n$ we define $T+\x$ by
\[ T + \x = \set{\y+\x}{\y\in T} . \]
Further for any $\lambda\in\R$ we define $\lambda T$ by
\[ \lambda T = \set{\lambda\y}{\y\in T} . \]
\textbf{Theorem 4 (Blichfeldt, 1914):} Let $P$ be a non-empty set of points in $\R^n$ which is invariant by translation by integer points and has precisely $N$ points in $U^n$.

Let $A$ be a subset of $\R^n$ of positive Lebesgue measure $\mu(A)$.  Then there is an $\x$ in $U^n$ such that $A+\x$ contains at least $N\cdot\mu(A)$ points of $P$.  Further if $A$ is compact then there is an $\x$ in $U^n$ such that $A+\x$ contains more than $N\cdot\mu(A)$ points of $P$.

\pf For any set $S$ in $\R^n$ we let $\upsilon(S)$ be the number of points of $P$ in $S$.  Let $\p_1\c \dotsc\c \p_N$ be the $N$ points of $P$ in $U^n$.  We put
\[ P_i = \set{\p_i+\g}{\g\in\Lambda_0} \]
for $i=1\c \dotsc\c N$.  Since $P$ is invariant by translation by integer points, or equivalently be elements of $\Lambda_0$,
\[ P = \bigcup_{i=1}^N P_i . \]
Further we have $P_i\cap P_j=\emptyset$ for $i\neq j$.  Now for any $S\subseteq\R^n$ let $\upsilon_i(S)$ denote the number of points of $P_i$ in $S$ for $i=1\c \dotsc\c N$.

Let $\chi$ be the characteristic function of $A$\footnote{is $1$ if argument is in $A$}.  Then
\[ \upsilon_i(A+\x) = \sum_{\g\in\Lambda_0}\chi(\p_i+\g-\x) \]
we have
\begin{align*}
\int_{U^n} \upsilon_i (A+\x)\d\x &= \int_{U^n}\sum_{\g\in\Lambda_0}\chi(\p_i+\g-\x)\d\x \\
&= \int_{\R^n}\chi(\z)\d\z \\
&= \mu(A)
\end{align*}
Thus
\[ \int_{U^n} \upsilon(A+\x)\d\x = N\mu(A) . \]
Therefore there is some element $\x$ in $U^n$ such that $\upsilon(A+\x_1)\geq N\mu(A)$ and so $A+\x_1$ contains at least $N\mu(A)$ points of $P$.

If $A$ is compact and $N\mu(A)$ is not an integer there is nothing more to prove.  Suppose $N\mu(A)=h$ for $h\in\Z^+$. \\
For $k=1\c 2\c \dotsc$ we define $A_k$ by
\[ A_k = (1+\tfrac1k)A . \]
By what we have just proved for each positive integer $k$ there is an $\x_k$ in $U^n$ such that
\[ \upsilon(A_k+\x_k) \geq h + 1 . \]
\vspace{-\baselineskip}