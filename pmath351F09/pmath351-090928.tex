Ball $B(x_0,r)=\set{x\in X}{d(x,x_0)<r}$ ($r>0$, $x_0\in X$) \\
$U\subseteq X$ is \emph{open} if $\forall u\in U~\exists B(u,r)\subseteq U$ for some $r>0$

\prop Balls are open sets.

\prop \begin{enumerate}
\item If $U_1$, $U_2$ are open then $U_1\cap U_2$ is open.
\item If $\brace{U_i}_{i\in I}$ are open then $\bigcup_{i\in I}$ is open.
\end{enumerate}
\pf \begin{enumerate}
\item Let $x\in U_1\cap U_2$.  Since $x\in U_i$ and these are open, $\exists r_i>0$ such that $B(x,r_i)\subseteq U_i$.  Let $r=\min(r_1,r_2)>0$ and then $B(x,r)\subseteq B(x,r_1)\subseteq B(x,r_2)\subseteq U_1\cap U_2$ \\
$U_1\cap U_2$ is open
\item If $x\in\bigcup_{i\in I}U_i$ then $\exists i_0\in I$ such that $x\in U_{i_0}$.  That set is open so $\exists r$ such that $B(x,r)\subseteq U_{i_0}\subseteq\bigcup_{i\in I}U_i$ \\
$\implies \bigcup U_i$ is open.
\end{enumerate}

\ex $B(0,\frac1n)$ in $\R^2$.  $\bigcap_{i=1}^\infty B(0,\frac1n)=\brace0$, not open. \\
This shows an infinite intersection of open sets need not be open.

\prop $U$ is open iff $U$ is a union of balls.

\pf ($\Longleftarrow$) Any union of balls is a union of open sets, therefore is open. \\
($\Longrightarrow$) Since $U$ is open, $\forall x\in U~\exists B(x,r_x)\subseteq U$. \\
Claim $U=\bigcup_{x\in U}B(x,r_x)$ \\
RHS $\subseteq U$ as each $B(x,r_x)\subseteq U$ \\
But each $x\in U$ belongs to $B(x,r_x)$, therefore $U\subseteq\text{RHS}$

\prop $\Int U=\bigcup_{\substack{V\subseteq U\\\text{open}}}$: says $\Int U$ is the largest open subset of $U$

\pf Let $x\in\Int U$.  By definition $\exists r>0$ such that $B(x,r)\subseteq U$. \\
$B(x,r)$ is an open set in $U$ therefore $x\in\bigcup_{\substack{V\subseteq U\\\text{$V$ open}}} V \longrightarrow \Int U\subseteq\bigcup_{\substack{V\subseteq U\\\text{$V$ open}}} V$ \\
Pick $x\in\bigcup_{\substack{V\subseteq U\\\text{$V$ open}}} V$.  Then $x\in V$ some $V\subseteq U$, open. \\
So $\exists B(x,r)\subseteq V\subseteq U \implies x\in\Int U\implies\bigcup_{\substack{V\subseteq U\\\text{$V$ open}}} V \subseteq \Int V$

$\Int(A\cup B)\cancel{\overset{?}{=}}\Int A \cup \Int B$ \\
No:\begin{enumerate}
\item $\underbrace{(-1,0]}_A\cup\underbrace{[0,1)}_B$ \\
$\Int(A\cup B) = (-1,1)$ \\
$\Int A = (-1,0)$, $\Int B=(0,1)$
\item $A=\Q$, $B=\R\setminus\Q$ \\
$\Int A=\emptyset=\Int B$ \\
$\Int(A\cup B)=\Int\R=\R$
\end{enumerate}

\defin $A\subseteq X$ is \emph{closed} if $A^\Co=X\setminus A$ is open \\
\ex \begin{enumerate}
\item $\R$: which intervals are closed sets?
\[ [a,b], [a,\infty], (-\infty,a], (-\infty,\infty) \]%
\marginpar{$[a,b)$ is not closed because $(-\infty,a)\cup[b,-\infty)$ is not open as $b$ is not an interior point.}%
\item $X$, $\emptyset$ are both open and closed
\item $\Q\subseteq\R$ is neither open nor closed
\item $(X,d)$, $\brace{x_0}$ is closed \\
\pf Let $z\notin\brace{x_0}$, i.e., $z\neq x_0$\marginpar{figure: line between $x_0$ and $z$} \\
Consider $B(z,d(z,x_0))$.  Verify that $x_n\notin B(z,d(z,x_0))$ \\
That's true since $B(z,d(z,x_0))=\set{y}{d(y,z)<d(z,x_0)}$ and $y=x_0$ does not have that property. \\
Thus $B(z,d(z,x_0))\subseteq\brace{x_0}^\Co$.  Therefore $\brace{x_0}$ is closed.
\item $\set{x}{d(x,x_0)=r_0}$ is closed
\item Discrete space: Every set is closed (and open)
\item $\Z$, $\abs\cdot$, \quad $B(n,r\footnote{$r\leq1$})=\brace n$ \marginpar{figure: $n-1$, $n$, $n+1$ on real line}%
\\ Every set is open and closed.
\end{enumerate}
\prop \begin{enumerate}
\item Any intersection of closed sets is closed.
\item A finite union of closed sets is closed.
\end{enumerate}
\pf \begin{enumerate}
\item Let $U=\bigcap U_i$, $U_i$ closed
\[ U^\Co = \paren*{\bigcap U_i}^\Co = \underbrace{\bigcup \underbrace{U_i^\Co}_{\text{open}}}_{\text{open}} \qquad\text{therefore $U$ is closed} \]
\end{enumerate}
\defin A point $x\in X$ is an \emph{accumulation point\footnote{(cluster point, limit point)} of $U\subseteq X$} if $\forall r>0$, $B(x,r)\cap(U\setminus\brace{x})\neq\emptyset$ \\
(i.e., every ball about $x$ contains a point of $U$ other than $x$) \\
Equivalently: every open set $V$ containing $x$ satisfies \[ V\cap(U\setminus\brace{x})\neq\emptyset . \]
Equivalently, $\forall r>0$, $B(x,r)\cap U$ is infinte.

Take $B(x,r)$: Find $u_1\in B(x,r)\cap (U\setminus\brace{x})$.\marginpar{figure: radii around point $x$ with $u_1$, $u_2$, $u_3$ increasingly closer to $x$} \\
Consider $B(x,d(x,u_1))\owns u_2$, where $u_2\in U\setminus\brace{x}$ \\
($u_2\neq u_1$, since $u_1\notin B(x,d(x,u_1))$) \\
Repeat to find a countably infinite set $\brace{u_i}\subseteq U$, with $u_i\in B(x,r)$.

\ex \begin{enumerate}
\item $U=[0,1)$ in $\R$\marginpar{figure: $[0,1)$ real line} \\
$1$ is an accumulation point of $U$ [but $1$ is not in $U$.] \\
Everything in $U$ is an accumulation point of $U$.  Nothing else.
\item $U=[0,1)\cup\brace{2}$ in $\R$.\marginpar{figure: $[0,1)\cup\brace2$ real line} \\
$2$ is \emph{not} an accumulation point: called \emph{isolated points}.
\end{enumerate}
