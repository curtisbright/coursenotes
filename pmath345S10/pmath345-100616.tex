\defn Let $f_1$, $\dotsc$, $f_r\in E[x_1,\dotsc,x_n]$ be any set of polynomials.  Then $\brace{f_1,\dotsc,f_r}$ is a Gr\"obner basis for $I=(f_1,\dotsc,f_r)$ \emph{iff}
\[ \LT(I) = (\LT(f_1),\dotsc,\LT(f_r)) . \]
In other words, any monomial $m$ that is divisible by $\LT(g)$ for some $g\in I$ is divisible by some $\LT(f_i)$.

\thm If $\LT(I)=(\LT(f_1),\dotsc,\LT(f_r))$ and $f_1$, $\dotsc$, $f_r\in I$, then $I=(f_1,\dotsc,f_r)$. \\
\pf Since $f_1$, $\dotsc$, $f_r\in I$, it follows immediately that $(f_1,\dotsc,f_r)\subset I$.  So it suffices to show $I\subset(f_1,\dotsc,f_r)$.  Let $g\in I$, and divide $g$ by $\brace{f_1,\dotsc,f_r}$.  By the Division Theorem, we get:
\[ g = a_1 f_1 + \dotsb + a_r f_r + t \]
where $t$ is the remainder, whose terms are all \emph{not} divisible by any $(\LT(f_i))$.  But $t\in I$, so $\LT(t)\in\LT(I)=(\LT(f_1),\dotsc,\LT(f_r))$.
This immediately implies $t=0$ so $g\in(f_1,\dotsc,f_r)$. \qed

Do Gr\"obner bases exist?  Yes! \\
\thm Let $I\subset F[x_1,\dotsc,x_n]$ be an ideal.  Then there is a Gr\"obner basis for $I$. \\
\pf Consider $\LT(I)$, which is generated by an infinite collection of monomials:
\[ \m = \set{\LT(f)}{f\in I} \]
Notice that $\LT(I)$ is also generated by the set of leading monomials of elements of $I$:
\[ \L = \set{\LM(f)}{f\in I} \]
The set $\L$ is countably infinite, since each monomial $x_1^{a_1}\dotsm x_n^{a_n}$ corresponding uniquely to $(a_1,\dotsc,a_n)\in\Z^n$.  Therefore, we can enumerate the monomials in $\L$:
\[ m_1\co m_2\co m_3\co \dotsc \]
Define $I_j = (m_1,\dotsc,m_j)$
\[ I_1 \subset I_2 \subset I_3 \subset I_4 \subset \dotsb \]
So by ACC, this chain stabilizes at some finite step $v$, so:
\begin{align*}
\LT(I) &= \bigcup_{j=1}^\infty I_j = I_v \\
&= (m_1,\dotsc,m_v) \\
&= (\LT(f_1),\dotsc,\LT(f_v))
\end{align*}
for some $f_1$, $\dotsc$, $f_v\in I$. \qed

\thm Let $\brace{f_1,\dotsc,f_t}$ be a Gr\"obner basis (for $I=(f_1,\dotsc,f_t)\neq(0)$), $f\in F[x_1,\dotsc,x_n]$.  Then there exists a unique $r\in F[x_1,\dotsc,x_n]$ such that
\[ f = a_1 f_1 + \dotsb + a_t f_t + r \]
for some $a_1$, $\dotsc$, $a_t\in F[x_1,\dotsc,x_n]$, and no term of $r$ is divisible by any $\LT(f_i)$. \\
\pf Say:
\[ a_1 f_1 + \dotsb + a_t f_t + r = a_1' f_1 + \dotsb + a_t' f_t + r' \]
Then:
\[ (a_1-a_1')f_1 + \dotsb + (a_t-a_t')f_t = r'-r \]
%Since
So $\LT(r'-r)\in\LT(I)=(\LT(f_1),\dotsc,\LT(f_t))$.  But $r'$ and $r$ aren't allowed to have any terms divisible by any $\LT(f_i)$, so $r'-r$ has no terms and is therefore $0$.  So $r'=r$. \qed \\
\cor Let $f\in F[x_1,\dotsc,x_n]$ be any polynomial, $I$ any nonzero ideal, $f_1$, $\dotsc$, $f_t$ a Gr\"obner basis for $I$.  Then $f\in I$ \emph{iff} $f$ divided by $\brace{f_1,\dotsc,f_t}$ gives zero remainder. \\
\pf Immediate. \qed

\defn Let $f$, $g\in F[x_1,\dotsc,x_n]$ be any nonzero polynomials.  Then
\[ \S(f,g) = \paren[\Big]{\frac{\LCM}{\LT(f)}}f-\paren[\Big]{\frac{\LCM}{\LT(g)}}g \]
where $\LCM=\LCM(\LM(f),\LM(g))$.
\begin{gather*}
f=3x^2-2 \qquad g = -xy+1 \\
\LT(f)=3x^2 \qquad \LT(g) = -xy \\
\LM(f)=x^2 \qquad \LM(g) = xy \\
\LCM = x^2y \\
\begin{aligned}
\implies \S(f,g) &= \frac{x^2y}{3x^2}(3x^2-2)-\frac{x^2y}{-xy}(-xy+1) \\
&= \tfrac13 y(3x^2-2)-(-x)(-xy+1) \\
&= (x^2y-\tfrac23y)-(x^2y-x) \\
&= x-\tfrac23y
\end{aligned}
\end{gather*}
