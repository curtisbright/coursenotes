\begin{tabular}{ccc}
median & 18.5 & 74\% \\
mean & 17.5 & 70\% \\
% & 1/3 & \\
% & 82\% & \\
% & 1/3 & \\
% & 62\% & \\
% & 1/3 & 
\end{tabular}
%
\[ \text{fields} \subseteq \text{euclidean domains} \subseteq \text{pids} \subseteq \text{ufds} \subsetneq \text{integral domains} \]
\defin $a,b\in R$ integral domain.  $a,b$ irreducibles.  We say $a$ and $b$ are \emph{associate} if $a=bu$ for some unit $u$.

\textbf{Exercises:}
\begin{enumerate}
\item Being associate is an equivalence relation among the irreducibles.
\item If $a$ is irreducible/prime then $au$ is irreducible/prime if $u$ is a unit.
\item $a$ is \emph{irreducible} iff whenever $a=bc$ either $b$ or $c$ is a unit.
\item $a,b$ irreducibles.  $a$ and $b$ are associate $\iff$ $a\mid b$
\end{enumerate}
\lem In a unique factorization domain, $\text{irreducible}=\text{prime}$. \\
\pf Recall $R$ unique factorization domain means $a$ is nonzero nonunit then $a$ is a finite product of primes.
\[ \text{prime}\implies\text{irreducible}\quad \checkmark \]
Conversely let $a$ be an irreducible.  $a=p_1\dotsm p_n$ where $p_i$ are prime. \\
Each $p_i\mid a\implies p_i=au_i$ for some $u_i$. \\
\textbf{Exercise:}~If a product of elements is a unit then so is each factor.

$a=p_iv$, $v$ is a unit
\begin{align*}
\text{cancellation} &\implies v=p_1\dotsm\cancel{p_i}\footnotemark\dotsm p_n \\
&\mathrel{\;{\Longrightarrow}\footnotemark\;} n=1 \\
&\implies\text{$a$ is prime}
\end{align*}\addtocounter{footnote}{-1}\footnotetext{remove $p_i$}\addtocounter{footnote}{1}\footnotetext{by previous exercise}%
\cor There are integral domains that are \emph{not} unique factorization domains. \\
%\pf We have seen an example of an integral domain where $\text{irreducible}\hspace{1.5ex}\not\hspace{-1.5ex}\implies\text{prime}$. \\
\pf We have seen an example of an integral domain where $\text{irreducible}\centernot\implies\text{prime}$. \\
\thm (Unique factorization theorem): \\
$R$ unique factorization domain.  $a$ nonzero nonunit.
\begin{equation*}
\begin{aligned}
a &= p_1 \dotsm p_n \\
a &= q_1 \dotsm q_l
\end{aligned} \qquad \text{where the $p_i$s and $q_j$s are prime}
\end{equation*}
Then $n=l$ and after re-indexing each $p_i$ is associate to $q_i$. \\
\pf By induction on $n$. \\
$n=1$:
\[ p_1 = a = q_1\dotsm q_l \]
$\Longrightarrow$ $l=1$ and $p_1=q_1$ as before \\
$p_1\mid q_1$ $\implies$ $p_1=q_1u$ \\
\[ q_1u = q_1q_2\dotsm q_l \]
$\Longrightarrow$ $u=q_2\dotsm q_l$\footnote{contradiction} $\implies$ $l=1$ $\checkmark$

$n>1$:
\[ p_1\dotsm p_n = a = q_1\dotsm q_l \]
%$p_1\mid\text{LHS}$ $\implies$ $p_1\mid q_i$ or $p_1\mid(q_2\dotsm q_l)$\footnote{$\implies$ $p_1\mid q_2$ or $p_1\mid(q_3\dotsm q_l)$\footnotemark}\footnotetext{$\implies\dotsb$} \\
\[ p_1\mid\text{LHS} \implies
\begin{gathered}
p_1\mid q_1 \\
\text{or} \\
p_1\mid(q_2\dotsm q_l)
\end{gathered} \overset{p_1\nmid q_1}\implies
\begin{gathered}
p_1\mid q_2 \\
\text{or} \\
p_1\mid(q_3\dotsm q_l)
\end{gathered} \overset{p_1\nmid q_2}\implies \dotsb \]
$\implies$ $p_1\mid q_i$ for some $i=1,\dotsc,l$. \\
After re-indexing without loss of generality let $i=1$. \\
$\implies$ $p_1\mid q_1$ $\implies$ $q_1=p_1u$, $u$ unit.
\begin{gather*}
\cancel{p_1} \dotsm p_n = u \cancel{p_1} q_2 \dotsm q_l \\
p_2 \dotsm p_n = u q_2 \dotsm q_l
\end{gather*}
Replacing $q_2$ by an associate (namely $uq_2$) we may assume without loss of generality
\[ p_2\dotsm p_n = q_2\dotsm q_l \]
$\overset{\text{IH}}\Longrightarrow$ $n=l$ and after re-indexing \emph{$p_j$ is associate to $q_j$} $j=2,\dotsc,n=l$.

\ex (non-ufd) \\
$\Z[2i]$ subring of Gaussian integers
\[ \Z[2i] = \set{a+2bi}{a,b\in\Z} \]
$i=\sqrt{-1}$ \\
Fails unique factorization:
\begin{align*}
4 &= 2\cdot 2v \\
4 &= (-2i)\cdot(2i)
\end{align*}
$2,2i\in\Z[i]$ \\
\textbf{Need:}
\begin{enumerate}
\item $2,2i$ are irreducibles
\item $2$ and $2i$ are \emph{not} associate
\end{enumerate}
This leads to two non-associate factorizations of $4$ into irreducibles \\
$\implies$ $\Z[2i]$ \emph{not} unique factorization domain \\
\claim $2$ is irreducible \\
\pf
\begin{align*}
2 &= (a+2bi)(c+2di) \qquad a,b,c,d\in\Z \\
  &= (ac-4bd) + 2(ad+bc)i
\end{align*}
$\implies$ (1) $ad=-bc$ and \\
(2) $ac-4bd=2$ \\
Assume $bd\neq0$.  Then $ac\neq0$. \\
$\implies$ $\sgn(ac)=\text{positive}$ $\implies$ $\sgn(bd)=\text{negative}$ by (1) $\implies$ contradiction (2) \\
$\implies$ $\sgn(bd)=\text{positive}$ $\implies$ $\sgn(ac)=\text{negative}$ by (1) $\implies$ contradiction (2)

\thm (Unique factorization theorem) \\
$R$ ufd.  $a$ nonzero nonunit. \\
$\implies$ $2$ is irreducible $\checkmark$ \\
Similarly $2i$ is irreducible $\checkmark$ \\
Only units in $\Z[i]$ are $1,-1,-i\footnote{not in $R$},i\footnote{not in $R$}$ \\
Only units in $\Z[2i]$ are $1,-1$ \\
$\implies$ $2,2i$ are non-associates.
