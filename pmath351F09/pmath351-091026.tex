\ex $X={\underbrace{\set{(x,\sin\tfrac1x)}{x>0}}_{\equiv E}}\cup\brace{(0,0)}$ \\
Show $X$ is connected, but not path connected.\marginpar{graph of $\sin\frac1x$ for $x>0$} \\
$X=\overline{E}$ \\
\textbf{Proof outline:}\begin{enumerate}
\item $E$ path connected $\implies$ $E$ connected $\implies$\footnote{exercise} $\overline E$ connected
\item $X$ is not path connected
\end{enumerate}
\begin{enumerate}
\item $E$ path connected \\
Let $(x_1,\sin\frac1{x_1}),(x_2,\sin\frac1{x_2})\in E$ ($x_1,x_2>0$)
%
\begin{align*}
\text{Define }f\colon &[0,1]\to E \\
&t\mapsto\paren[\Big]{\underbrace{tx_1+(1-t)x_2}_{>0},\sin\frac{1}{tx_1+(1-t)x_2)}}\in E
\end{align*}
$f$ continuous on $[0,1]$ \\
$f(1)=(x_1,\sin\frac{1}{x_1})$, $f(0)=(x_2,\sin\frac{1}{x_2})$ $\implies$ $E$ is path connected
\item $X$ not path connected \\
Prove no ``path'' joining $(0,0)$ to $(\frac{1}{\pi},0)$ \\
Suppose $f\colon[a,b]\to X$ is a path with $f(a)=(0,0)$, $f(b)=(\frac{1}{\pi},0)$ \\
Claim:
\[ \paren*{\frac{1}{\frac{5\pi}{2}},1},\paren*{\frac{1}{\frac{9\pi}{2}},1},\dotsc,\paren*{\frac{1}{\frac{\pi}{2}+2\pi k},1}\in f[a,b] \]\marginpar{$k\in\N$} \\
Note: $f[a,b]$ is connected as $f$ is continuous and $[a,b]$ is connected.

Suppose without loss of generality $\paren*{\frac{1}{\frac{5\pi}{2}},1}\notin f[a,b]$. \\
Then
\[ f[a,b] = \paren[\Bigg]{\overbrace{f[a,b]\cap\set{(x,y)}{x>\frac{1}{\frac{5\pi}{2}}}}^{\owns(\frac1\pi,0)}} \cup \paren[\Bigg]{\overbrace{f[a,b]\cap\set{(x,y)}{x<\frac{1}{\frac{5\pi}{2}}}}^{\owns(0,0)}} \]
because only $(x,y)\in X$ with $x=\frac{1}{\frac{5\pi}{2}}$ is the point $\paren*{\frac{1}{\frac{5\pi}{2}},1}\notin f[a,b]$
\begin{itemize}
\item this contradicts the fact $f[a,b]$ is connected
\end{itemize}
Also $f[a,b]$ is compact. \\
The sequence $\brace*{\paren*{\frac{1}{\frac{\pi}{2}+2\pi k},1}}_{k=1}^\infty$ is Cauchy and therefore converges as $f[a,b]$ is complete. \\
Hence $(0,1)\in f[a,b]\subseteq X$. \\
But $(0,1)\notin X$ so contradiction.
\end{enumerate}

\textbf{Finite Dimensional Normed Vector Spaces over $\pmb\R$ (or $\pmb\C$)}

\textbf{Norm on a vector space:}
\begin{enumerate}
\item $\norm{v}\geq0$ and $\norm{v}=0$ if and only if $v=0$
\item $\norm{\alpha v}=\abs\alpha\norm{v}$ for all $\alpha$ scalars, $v\in V$
\item $\norm{v_1+v_2}\leq\norm{v_1}+\norm{v_2}$ for all $v_1,v_2\in V$
\end{enumerate}
Norms always give metrics by $d(x,y)=\norm{x-y}$

\ex Space of polynomials on $[0,1]$ of degree $\leq n$
\begin{enumerate}
\item $\inorm{p}=\max_{x\in[0,1]}\abs{p(x)}$
\item $\norm{p}_1=\int_0^1\abs{p(x)}\d x$
\end{enumerate}
\thm Suppose $V$ is a finite dimensional normed vector space over $\R$ with basis $\brace{v_1,\dotsc,v_n}$.  Then there exists constants $A,B>0$ such that for all $(a_1,\dotsc,a_n)\in\R^n$.
\[ A \norm{(a_1,\dotsc,a_n)}_{\R^n} \leq \norm[\Big]{\sum_{i=1}^n a_iv_i}_V \leq B\norm{(a_1,\dotsc,a_n)}_{\R^n} \]
Given any $v\in V$ there exists exactly one $(a_1,\dotsc,a_n)$ such that $v=\sum_1^n a_iv_i$.  Theorem says $\norm{a_1,\dotsc,a_n}_{\R^n}\sim\norm{v}_V$ \\
\pf \begin{align*}
\norm[\Big]{\sum_{i=1}^n a_iv_i}_V &\leq \sum_{i=1}^n \norm{a_iv_i}_V \\
&= \sum_{i=1}^n \abs{a_i}\norm{v_i}_V \\
&\leq\footnote{Cauchy--Schwartz} \paren[\bigg]{\sum_{i=1}^n\abs{a_i}^2}^{1/2}\paren[\bigg]{\sum_{i=1}^n\norm{v_i}^2}^{1/2} \\
&= \norm{(a_1,\dotsc,a_n)}_{\R^n} B \qquad \text{where }B=\paren[\bigg]{\sum_{i=1}^n\norm{v_i}^2}^{1/2}
\end{align*}
Define $F\colon\R^n\to\R$ by
\[ F(a_1,\dotsc,a_n) = \norm[\Big]{\sum_{i=1}^n a_iv_i} \]
Check $F$ is continuous:
\begin{align*}
F(\x) - F(\y) &= \norm[\Big]{\sum_{i=1}^n x_iv_i} - \norm[\Big]{\sum_{i=1}^n y_iv_i} \\
&\leq \norm*{\sum x_iv_i-\sum y_iv_i} + \norm*{\sum y_iv_i} - \norm*{\sum y_iv_i} \\
&= \norm*{\sum(x_i-y_i)v_i}
\end{align*}
Similarly $F(y)-F(x)\leq\norm*{\sum(x_i-y_i)v_i}$
\begin{align*}
\implies \abs{F(x)-F(y)} &\leq \norm*{\sum(x_i-y_i)v_i} \\
&\leq \sum\abs{x_i-y_i}\norm{v_i} \\
&\leq \paren*{\sum\abs{x_i-y_i}^2}^{1/2}\smash{\underbrace{\paren*{\sum\norm{v_i}^2}^{1/2}}_B} \\
&= B\norm{\x-\y}_{\R^n} \\
&= B d(x,y)
\end{align*}
$\Longrightarrow$ $F$ is continuous \\
Restrict $F$ to $S=\set{x\in\R^n}{\norm\x=1}$
\[ F(x) = 0 \iff x = 0 \]
In particular, if $x\in S$ then $F(x)>0$. \\
$S$ is compact.  By Extreme Value Theorem there exists $\delta>0$ such that $F(x)\geq\delta$ for all $x\in S$ \\
Take any $a=(a_1,\dotsc,a_n)\in\R^n\setminus\brace0$ \\
$\frac{a}{\norm{a}_{\R^n}}\in S$. \\
$F\paren[\big]{\frac{a}{\norm{a}}}\geq\delta$.% \\
\begin{align*}
\norm*{\sum a_iv_i}_V &= \norm[\Big]{\norm{a}_{\R^n}\sum\frac{a_i}{\norm{a_i}_{\R^n}}v_i}_V \\
&= \norm{a}_{\R^n} \norm[\Big]{\sum\frac{a_i}{\norm{a}}v_i}_V \\
&= \norm{a}_{\R^n} F\paren[\bigg]{\frac{a}{\norm{a}}} \\
&\geq \norm{a}_{\R^n}\delta
\end{align*}
Take $A=\delta$.
