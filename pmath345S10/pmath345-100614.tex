Long division helps with: \\
Telling if $p(x)\in(q(x))$. \\
Finding $\gcd(p(x),q(x))$.

In many variables: \\
Tell if $p(x_1,\dotsc,x_n)\in(f_1(x_1,\dotsc,x_n),\dotsc,f_r(x_1,\dotsc,x_n))$ \\
Find a ``good'' set of generators for $(f_1,\dotsc,f_r)$.

\eg Divide $x^2y+xy^2+y^2$ by $\brace{xy-1,y^2-1}$. (Use lex order with $x>y$.)
long division
\begin{gather*}
\begin{array}{@{}r@{}r@{}r@{}r@{}r@{}r@{}r@{}r@{}r@{}r@{}r@{}r@{}r@{\qquad}c@{}}
& & & & & & \mathllap{x+y\co1} & & & & & & & \text{Remainder} \\
\cline{2-7} \cline{14-14}
xy-1\co y^2-1\rule{0pt}{1em}& \bigl) & x^2y & {}+{} & xy^2 & {}+{} & y^2 & & & & & & & x\quad y\quad 1 \\
& & x^2y & {}-{} & x & \\
\cline{3-5}
\rule{0pt}{1em}& & & & xy^2 & {}+{} & x & {}+{} & y^2 \\
& & & & xy^2 &{}-{}& y \\
\cline{5-7}
\rule{0pt}{1em}& & & & & & \cancelto{}{x} &{}+{}& y^2 &{}+{}& y \\
& & & & & & & & y^2 &{}-{}& 1 \\ \cline{9-11}
& & & & & & & & & & \cancelto{}{y} &{}+{} & \cancelto{}{1}
\end{array} \\
\therefore~x^2y+xy^2+y^2 = (x+y)\footnote{coefficient of $xy-1$}(xy-1) + (1)\footnote{coefficient of $y^2-1$}(y^2-1) + (x+y+1)\footnote{remainder}
\end{gather*}
%
%\begin{gather*}
%\begin{array}{@{}r@{}r@{}r@{}r@{}r@{}r@{}r@{}r@{}r@{}r@{}r@{}r@{}r@{\qquad}c@{}}
%& & x & {}+{} & y & {}+{} & 1 & & & & & & & \text{Remainder} \\
%\cline{2-7} \cline{14-14}
%xy-1\co y^2-1\rule{0pt}{1em}& \bigl) & x^2y & {}+{} & xy^2 & {}+{} & y^2 & & & & & & & x\quad y\quad 1 \\
%& & x^2y & {}-{} & x & \\
%\cline{3-5}
%\rule{0pt}{1em}& & & & xy^2 & {}+{} & x & {}+{} & y^2 \\
%& & & & xy^2 &{}-{}& y \\
%\cline{5-7}
%\rule{0pt}{1em}& & & & & & \text{\ovalbox{$x$}} &{}+{}& y^2 &{}+{}& y \\
%& & & & & & & & y^2 &{}-{}& 1 \\ \cline{9-11}
%& & & & & & & & & & \text{\ovalbox{$y$}} &{}+{} & \text{\ovalbox{$1$}}
%\end{array} \\
%\therefore~x^2y+xy^2+y^2 = (x+y)\footnote{coefficient of $xy-1$}(xy-1) + (1)\footnote{coefficient of $y^2-1$}(y^2-1) + (x+y+1)\footnote{remainder}
%\end{gather*}
%
\eg Same as before:
%long division
\begin{gather*}
\begin{array}{@{}r@{}r@{}r@{}r@{}r@{}r@{}r@{}r@{}r@{}r@{}r@{}r@{\qquad}c@{}}
& & & & & & \mathllap{x+1\co x} & & & & & & \text{Remainder} \\
\cline{2-7} \cline{13-13}
y^2-1\co xy-1\rule{0pt}{1em}& \bigl) & x^2y & {}+{} & xy^2 & {}+{} & y^2 & & & & & & 2x\quad 1 \\
& & x^2y & {}-{} & x & \\
\cline{3-5}
\rule{0pt}{1em}& & & & xy^2 & {}+{} & x & {}+{} & y^2 \\
& & & & xy^2 &{}-{}& x \\
\cline{5-7}
\rule{0pt}{1em}& & & & & & \cancelto{}{2x} &{}+{}& y^2 \\
& & & & & & & & y^2 &{}-{}& 1 \\ \cline{9-11}
& & & & & & & & & & \cancelto{}{1}
\end{array} \\
x^2y + xy^2 + y^2 = (x+1)\footnote{coefficient of $y^2-1$}(y^2-1) + (x)\footnote{coefficient of $xy-1$}(xy-1) + (2x+1)\footnote{remainder}
\end{gather*}
%\begin{gather*}
%\begin{array}{@{}r@{}r@{}r@{}r@{}r@{}r@{}r@{}r@{}r@{}r@{}r@{}r@{\qquad}c@{}}
%& & x & {}+{} & x & {}+{} & 1 & & & & & & \text{Remainder} \\
%\cline{2-7} \cline{13-13}
%y^2-1\co xy-1\rule{0pt}{1em}& \bigl) & x^2y & {}+{} & xy^2 & {}+{} & y^2 & & & & & & 2x\quad 1 \\
%& & x^2y & {}-{} & x & \\
%\cline{3-5}
%\rule{0pt}{1em}& & & & xy^2 & {}+{} & x & {}+{} & y^2 \\
%& & & & xy^2 &{}-{}& x \\
%\cline{5-7}
%\rule{0pt}{1em}& & & & & & \cancel{2x} &{}+{}& y^2 \\
%& & & & & & & & y^2 &{}-{}& 1 \\ \cline{9-11}
%& & & & & & & & & & \cancel1
%\end{array} \\
%x^2y + xy^2 + y^2 = (x+1)\footnote{coefficient of $y^2-1$}(y^2-1) + (x)\footnote{coefficient of $xy-1$}(xy-1) + (2x+1)\footnote{remainder}
%\end{gather*}
%
\thm Let $f_1$, $\dotsc$, $f_s\in F[x_1,\dotsc,x_n]$ where $F$ is a field, $f_1$, $\dotsc$, $f_s$ not all the zero polynomial.  Then every $f\in F[x_1,\dotsc,x_n]$ can be written as:
\[ f = a_1 f_1 + \dotsb + a_s f_s + r \]
where $a_i$, $r\in F[x_1,\dotsc,x_n]$, every term in $r$ not divisible by \emph{any} $\LT(f_i)$.  If $a_if_i\neq0$, then $\multideg(a_if_i)\leq\multideg(f)$. \\
\pf In Papantonopoulou. \qed

Let $I$ be an ideal of $F[x_1,\dotsc,x_n]$. \\
Define $\LT(I)={}$ideal generated by $\set{\LT(f)}{f\in I}$. \\
\fact If $I=(f_1,\dotsc,f_r)$, then
\[ \LT(I) \neq (\LT(f_1),\dotsc,\LT(f_r)) \]
unless the $f_i$ are carefully chosen.

\defn Let $I=(f_1,\dotsc,f_r)$ be an ideal of $F[x_1,\dotsc,x_n]$.  Then $\brace{f_1,\dotsc,f_r}$ is a Gr\"obner basis for $I$ \emph{iff} $\LT(I)=(\LT(f_1),\dotsc,\LT(f_r))$.
