\thm The following are equivalent
\begin{enumerate}
\item[1.] $X$ is compact
%\item[2.]
\item[3.] Every sequence $X$ has a convergent subsequence (limit in $X$)
\item[4.] $X$ is complete and totally bounded
\end{enumerate}
To finish the proof do (3 $\implies$ 4)
\begin{enumerate}
\item[(i)] Prove $X$ is complete. \\
Let $(x_n)$ be a Cauchy sequence in $X$. \\
By assumption (3), $(x_n)$ has a convergent subsequence.  A Cauchy sequence with a convergent subsequence converges. \\
$\implies$ $X$ is complete.
\item[(ii)] Prove $X$ is totally bounded. \\
Assume not.  Then for some $\epsilon>0$ there is no $\epsilon$-net. \\
Take $x_1\in X$.  Then $\brace{x_1}$ is not an $\epsilon$-net. \\
So there exists $x_2\in X$ such that $d(x_1,x_2)>\epsilon$. \\
Consider $\brace{x_1,x_2}$: not an $\epsilon$-net. \\
So there exists $x_3\in X$ such that $d(x_1,x_2)>\epsilon$ and $d(x_2,x_3)>\epsilon$. \\
Repeat: Get $\brace{x_n}_{n=1}^\infty$ such that $d(x_n,x_j)>\epsilon$ for all $j=1,\dotsc,n-1$, i.e., $d(x_i,x_j)>\epsilon$ for all $i\neq j$. \\
This sequence has no Cauchy subsequence, so no convergent subsequence: contradicting assumption (3).
\end{enumerate}
\ex $\text{Cantor Set}\subseteq[0,1]$.
\begin{itemize}
\item compact, empty interior \\
perfect $\to$ closed set in which every point is an accumulation point.
\end{itemize}
\textbf{Construction:}
$C_0=[0,1]$ \\
$C_1=[0,\frac13]\cup[\frac23,1]$
$C_2=\text{union of $4=2^2$ intervals of length $\frac19=\frac{1}{3^2}$}$\marginpar{figures of $C_0$, $C_1$, $C_2$} \\
$C_n=$ union of $2^n$ closed intervals, each of length $3^{-n}$ with gap between any two intervals $\geq3^{-n}$

$C_n$ is closed $\subseteq[0,1]$, therefore compact. \\
$C_n\subseteq C_{n-1}$ \\
Cantor set $C=\bigcap_{n=1}^\infty C_n$: closed $\subseteq[0,1]$, therefore compact. \\
$0,1\in C$.  $\frac13,\frac23,\frac19,\frac29,\dotsc\in C$: $C$ contains all endpoints of Cantor intervals.

Empty interior: Say $I=(a,b)\subseteq C$. \\
$\implies I \subseteq C_n$ for all $n$. \\
Pick $n$ such that $3^{-n}<b-a=\abs{I}$. \\
But then $I\not\subset C_n$ since the longest intervals in $C_n$ are length $3^{-n}$. \\
$\implies$ contradiction

\textbf{Perfect:} Let $x_0\in C$.  Fix $\epsilon>0$. \\
Pick $n$ such that $3^{-n}<\epsilon$. \\
$x_0\in C_n$ $\implies$ $x_0$ lies in a Cantor interval of step $n$, of length $3^{-n}$. \\
$a,b\in C$\marginpar{$x_0$ between $a$ and $b$, in an interval of length $3^{-n}$} \\
$d(x_0,a), d(x_0,b) \leq 3^{-n} < \epsilon$ \\
Hence $B(x_0,\epsilon)\cap(C\setminus\brace{x_0})$ is non-empty. \\
Since $B(x_0,\epsilon)\cap C\supseteq\brace{a,b}$

\prop A non-empty, perfect set $E$ in $\R^k$ is uncountable. \\
\pf $E$ must be infinite since it has accumulation points. \\
Assume $E=\brace{x_n}_{n=1}^\infty$ (i.e., $E$ is countably infinite) \\
Put $k_1=1$. \\
Look at $B(x_{k_1},1)=B(x_1,1)\equiv V_1$: open set containing $x_1$. \\
Since $x_1$ is an accumulation point of $E_1$ there exists $e\in V_1\setminus\brace{x_1}$, $e\in E$ \\
Pick least integer $k_2>k_1$ such that $x_{k_2}\in V_1\cap E$, $x_{k_2}\neq x_{k_1}$ \\
Pick $V_2$ open, contains $x_{k_2}$ and satisfies $\overline{V_2}\subseteq V_1$ and $x_{k_1}\notin\overline{V_2}$.\marginpar{figure: $x_{k_1}$ in $V_1$ and $x_{k_2}$ in $V_2$} \\
(e.g., $V_2=B(x_{k_2},r)$ where $r=\frac12\min(d(x_{k_1},x_{k_2}),1-d(x_{k_1},x_{k_2}))$)

Consider $V_2\cap E\setminus\brace{x_{k_2}}$: non-empty \\
Pick minimal $k_3$ such that $x_{k_3}\in V_2 \cap E\setminus\brace{x_{k_2}}$.

By construction $k_3>k_2$.\marginpar{$x_{k_2}\notin\overline{V_3}$} \\
Assume we have chosen $x_{k_n}\in E\cap V_{n-1}\setminus\brace{x_{k_{n-1}}}$ with $k_n>k_{n-1}$ and minimal; open sets $V_n\owns x_{k_n}$. \\
$\overline{V_n}\subset V_{n-1}$ and $x_{k_{n-1}}\notin\overline{V_n}$. \\
As $x_{k_n}$ is an accumulation point of $E$, we can choose $k_{n+1}$ minimal such that $x_{k_{n+1}}\in V_n\cap E\setminus\brace{x_{k_n}}$.  Then $k_{n+1}>k_n$. \\
Get $V_{n+1}$ open such that $\overline{V_{n+1}}\subset V_n$ and $x_{k_n}\notin\overline{V_{n+1}}$
\begin{align*}
\text{Put } K_n &= \overline{V_n}\cap E\footnotemark \\
&\subseteq V_{n-1} \cap E \subseteq \overline{V_{n-1}}\cap E = K_{n-1}
\end{align*}\footnotetext{non-empty, closed}%
so $K_1\supseteq K_2\supseteq \dotsb$ \\
\[ K_n \subseteq K_1 \subseteq \overline{B(x_0,1)}\footnote{compact in $\R^k$} . \]
Since nested, have FIP.  By characterization of compactness (2), $\bigcap_{n=1}^\infty K_n \neq \emptyset$.

Now, $x_1\notin\overline{V_2}$, therefore $x_1\notin\bigcap K_n$; $x_2\notin V_1$, therefore $x_2\notin\bigcap K_n$.  $x_{k_2}\notin\overline{V_3}$, therefore $x_{k_3}\notin\bigcap K_n$.  $x_{2+1}\notin V_2$, \ldots; $x_{k_j}\in\overline{V_{j+1}}$, therefore $x_{k_j}\notin\bigcap K_n$. \\
$\implies$ $x_j\notin\bigcap K_n$, for any $j$, and $K_n\subseteq E$. \\
Therefore $\bigcap K_n=\emptyset$: contradiction.
