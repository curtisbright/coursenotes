\cor \textbf{Weierstrass $M$-test}

Let $f_n\colon X\to\R$.  If there exists a sequence $M_k$ such that $\abs{f_k(x)}\leq M_k$ for all $x\in X$ and for all $k$ and if $\sum_1^\infty M_k$ converges, then $\sum_{k=1}^\infty f_k$ converges uniformly.

\ex
\[ f_k(x) = \frac{\sin kx}{k^2} \qquad \abs{f_k(x)}\leq\frac{1}{k^2} \qquad 0\leq\sum\frac{1}{k^2}<\infty \]
\negthickspace$\implies\sum\frac{\sin kx}{k^2}$ is a continuous function. \\
\pf Let $S_N(x)=\sum_1^N f_k(x)$.  Show $\brace{S_N}$ converges uniformly.  It's enough to prove $\brace{S_N}$ is uniformly Cauchy.
\[ \abs{S_N-S_M(x)} = \abs[\bigg]{\sum_{N+1}^M f_k(x)} \leq \sum_{k=N+1}^M\abs[\big]{f_k(x)} \leq \sum_{k=N+1}^M M_k \to0\text{ as }M>N\to\infty \]
\negthickspace$\implies\brace{S_N}$ is uniformly Cauchy.

\textbf{Dini's Theorem:} Suppose $K$ is compact and $f_n\colon K\to\R$ converges pointwise to $f$.  If $f_n,f$ are continuous and $f_{n+1}(x)\leq f_n(x)$ for all $n$, for all $x\in K$, then $f_n\to f$ uniformly. \\
\pf Let $g_n=f_n-f$ \\
$g_n$ is continuous \\
$g_n\to0$ pointwise \\
$g_n(x)\geq g_{n+1}(x)$ \\
$g_n\geq0$ since $f(x)\leq f_n(x)$ as $f_n(x)$ decreases

Prove $g_n\to0$ uniformly to conclude $f_n\to f$ uniformly. \\
Let $\epsilon>0$.  Find $N$ such that $\abs{g_n(x)}<\epsilon$ for all $n\geq N$ and for all $x\in K$, \\ $\iff$ $0\leq g_n(x)\leq\epsilon$ for all $n\geq N$ and for all $x\in K$. \\
Since $g_n\to0$ pointwise, for all $t\in K$ there exists $N_t$ such that $0\leq g_n(t)<\frac{\epsilon}{2}$ for all $n\geq N_t$. \\
In particular, $g_{N_t}(t)<\frac{\epsilon}{2}$.

Because $g_{N_t}$ is continuous at $t$ so there exists $\delta_t>0$ such that if $d(t,x)<\delta_t$ then $\abs{g_{N_t}(t)-g_{N_t}(x)}<\frac{\epsilon}{2}$. \\
The balls $B(t,\delta_t)$, $t\in K$ are an open cover of the compact set $K$.  Take a finite subcover say $B(t_1,\delta_{t_1}),\dotsc,B(t_L,\delta_{t_L})$.

If $x\in K$ there exists $i$ such that $x\in B(t_i,\delta_{t_i})$
%\[ \implies d(x,t_i)<\delta_{t_i} \implies \abs{g_{N_{t_i}}(t_i)-g_{N_{t_i}}(x)} < \frac{\epsilon}{2} \]
%\begin{align*}
%\implies \abs{g_{N_{t_i}}(x)} &\leq \abs{g_{N_{t_i}}(x)-g_{N_{t_i}}(t_i)} + \abs{g_{N_{t_i}}(t_i)} \\
%&< \frac{\epsilon}{2} + \frac{\epsilon}{2} = \epsilon
%\end{align*}
\begin{gather*}
\implies d(x,t_i)<\delta_{t_i} \implies \abs{g_{N_{t_i}}(t_i)-g_{N_{t_i}}(x)} < \frac{\epsilon}{2} \\
\begin{aligned}
\implies \abs{g_{N_{t_i}}(x)} &\leq \abs{g_{N_{t_i}}(x)-g_{N_{t_i}}(t_i)} + \abs{g_{N_{t_i}}(t_i)} \\
&< \frac{\epsilon}{2} + \frac{\epsilon}{2} = \epsilon
\end{aligned}
\end{gather*}
Take $N=\max(N_{t_1},\dotsc,N_{t_L})$.

Let $n\geq N$ and $x\in K$.  Get $t_i$ as before.
%\[ 0\leq g_n(x) \mathbin{\mathord{\leq}\footnotemark} g_N(x) \mathbin{\mathord\leq\footnotemark} g_{N_{t_i}}(x)<\epsilon \]
\[ 0\leq g_n(x) \mathbin{\mathord{\leq}\footnotemark} g_N(x) \leq g_{N_{t_i}}(x)<\epsilon \]\footnotetext{by $g_n$ decreasing}%
This is uniform convergence. \\
\textbf{Examples:}
\begin{enumerate}
\item See need $K$ compact \\
$f_n(x)=\frac{1}{nx+1}$ on $K=(0,1]$ \\
$f_n(x)\to0$\footnotemark\ pointwise \\
$f_{n+1}(x)\leq f_n(x)$ \\
$f_n, f$ continuous

$f_n(1/n)=1/2$ for all $n$ so there does not exist $N$ such that for all $n\geq N$ and for all $x\in(0,1]$, $\abs{f_n(x)}<1/2$.
\item $f_n(x)=x^n$ on $[0,1]$ \\
Everything satisfied except continuity of $f$.
\item\marginpar{graph of $f_n(x)$: peak of height $n$ at $x=1/n$}
$f_n\to0$ pointwise \\
$f_n(1/n)=n$ so convergence is not uniform \\
$f_n$ are not decreasing pointwise.
\end{enumerate}\footnotetext{$=f$}

\textbf{Function Spaces}
$ C(X) = \text{continuous functions $f\colon X\to\R$} $ vector spaces \\
$ C_b(X) = \text{continuous, bounded functions $f\colon X\to\R$} $ subspaces \\
When $X$ is compact $C(X)=C_b(X)$ \\
$C(\R)\setminus C_b(\R)$: $f(x)=x$

Define $\norm f=\sup_{x\in X}\abs{f(x)}$ when $f\in C_b(X)$ \\
``$\sup$ norm'' or ``uniform'' norm (exercise) \\
$\abs{f(x)}\leq\norm f$ for all $x\in X$ \\
Defines a metric on $C_b(x)$ by $d(f,g)=\norm{f-g}$ %\\

Ball $B(f,r)$:\marginpar{figure: $g$ within a $\epsilon$-tube of $f$}
\marginpar{figure: $d(f,g)=\norm{f-g}$}

Take $f_n,f\in C_n(X)$ \\
Recall $f_n\to f$ uniformly means for all $\epsilon>0$ there exists $N$ such that $\abs{f_n(x)-f(x)}\leq\epsilon$ for all $n\geq N$ and for all $x\in X$.
\begin{align*}
&\iff \sup_{x\in X}\abs{f_n(x)-f(x)}\leq\epsilon\quad\forall n\geq N \\
&\iff \norm{f_n-f}\leq\epsilon\quad\forall n\geq N \\
&\iff d(f_n,f)\leq\epsilon\quad\forall n\geq N \\
&\iff f_n\to f \text{ in metric space $C_b(x)$}
\end{align*}
$\brace{f_n}$ in $C_b(x)$ is Cauchy if and only if $\brace{f_n}$ is uniformly Cauchy

\thm $C_b(X)$ is a complete metric space \\
\pf Suppose $\brace{f_n}$ in $C_b(X)$ is a Cauchy sequence.  Then $\brace{f_n}$ is uniformly Cauchy and so it converges uniformly to some $f\in C(X)$. \\
Get $N$ such that $\abs{f(x)-F_N(x)}\leq1$ for all $x\in X$
\begin{align*}
&\implies \abs{f(x)}\leq 1 + \abs{f_N(x)} \leq 1 + \norm{f_N} \\
&\implies \norm f = \sup_{x\in X}\abs{f(x)} \leq 1 + \norm{f_N} < \infty \\
&\implies f \in C_b(X)
\end{align*}
Hence $f_n\to f$ in uniform norm. \\
Therefore $C_b(X)$ is complete.

$C_b(X)$ is a \emph{complete normed vector space}, i.e., a \emph{Banach space}.