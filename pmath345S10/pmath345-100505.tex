\begin{small}\defn A ring is a set $R$ with 2 operations $+\colon R\times R\to R$, $\cdot\colon R\times R\to R$ satisfying for all $a$, $b$, $c\in R$:
\begin{enumerate}[label=(\arabic*)]
\item $(a+b)+c=a+(b+c)$
\item $a+b=b+a$
\item There is $0\in R$ such that $a+0=a$ $\forall a\in R$
\item There is $-a\in R$ such that $a+(-a)=0$
\item $a\cdot(b\cdot c)=(a\cdot b)\cdot c$
\item $a\cdot b=b\cdot a$
\item There is $1\in R$ such that $a\cdot 1=1\cdot a=a$ for all $a\in R$
\item $a\cdot(b+c)=a\cdot b+a\cdot c$ \\
$(a+b)\cdot c=a\cdot c+b\cdot c$
\end{enumerate}\end{small}%
\thm (Subring Theorem) \\
Let $R$ be a ring.  $S\subset R$ any subset.  Then $S$ is a subring of $R$ \emph{iff}:
\begin{enumerate}[label=(\arabic*)]
\item $0$, $1\in S$
\item If $a$, $b\in S$ then $a-b\in S$
\item If $a$, $b\in S$ then $ab\in S$
\end{enumerate}
\pf Forwards is trivial. \\
Backwards: Assume $S$ satisfies (1), (2), and (3) from the theorem.  We need to check that $+$ and $\cdot$ are well defined from $S\times S\to S$, and we need to check (1)--(8).

The fact that $\cdot$ is from $S\times S\to S$ is precisely (3).  For $+$, first note that (1) means that $0$, $1\in S$.  By (2), we find $0-1=-1\in S$.  Thus, if $a$, $b\in S$, then by (3), $(-1)\cdot b\in S$ so since $(-1)\cdot b=-b$, we get $-b\in S$.
\begin{small}\begin{gather*}
\begin{aligned}
(-1)\cdot b + b &= (-1+1)\cdot b \\
&= 0\cdot b \\
&= 0
\end{aligned} \\
\begin{aligned}
\text{follows from: } 0\cdot b &= (0+0)\cdot b \\
&= 0\cdot b + 0\cdot b \\
\implies -0\cdot b + 0\cdot b &= -0\cdot b + 0\cdot b + 0\cdot b \\
\implies 0 &= 0\cdot b
\end{aligned}
\end{gather*}\end{small}%
We want to show that $a+b\in S$.  Well, $-b\in S$, so $a-(-b)\in S$ by (2), so $a+b\in S$. \\
\begin{small}(1), (2), (5), (6), (8): Trivial for $S$ \\
(3), (7): By (1) \\
(4): Already done\end{small}\qed
%We want to show that $a+b\in S$.  Well, $-b\in S$, so $a-(-b)\in S$ by (2), so $a+b\in S$. \qed

\eg Prove $\Z[\sqrt{17}]=\set{a+b\sqrt{17}}{a,b\in\Z}$ is a subring of $\R$. \\
\soln $\Z[\sqrt{17}]\subset\R$ clearly.  By Subring Theorem:
\begin{enumerate}[label=(\arabic*)]
\item $0=0+0\sqrt{17}\in\Z[\sqrt{17}]$ \\
$1=1+0\sqrt{17}\in\Z[\sqrt{17}]$
\item $a+b\sqrt{17}\in\Z[\sqrt{17}]$ \\
$c+d\sqrt{17}\in\Z[\sqrt{17}]$ \\
$\implies(a+b\sqrt{17})-(c+d\sqrt{17})=(a-c)+(b-d)\sqrt{17}\in\Z[\sqrt{17}]$
\item Similarly, $(a+b\sqrt{17})(c+d\sqrt{17})=(ac+17bd)+(ad+bc)\sqrt{17}\in\Z[\sqrt{17}]$ so we're done.
\end{enumerate}
\defn Let $R$ be a ring, $r\in R$ any element.  Then:

$r$ is a zero divisor \emph{iff} $ra=0$ for some $a\in R$, $a\neq0$, provided $r\neq0$.  $r$ is a unit \emph{iff} there is an element $1/r\in R$ such that $r(1/r)=1$. \\
$r$ is nilpotent \emph{iff} $r^n=0$ for some positive integer $n$ ($r\neq0$).

\defn A ring $R$ is called an (integral) domain \emph{iff} it contains no zero divisors.

A ring $R$ is a field \emph{iff} every nonzero element is a unit. \\
A ring $R$ is reduced \emph{iff} it contains no nilpotent elements.

$\Z/4\Z$ is not reduced: $2^2=0$, $2\neq0$ \\
$\Z/6\Z$ is reduced, but not a domain: $2\cdot3=0$, $2,3\neq0$ \\
$\Z/7\Z$ is a field: every nonzero element is a unit: $1\cdot1=1$, $2\cdot4=1$, $3\cdot5=1$, $6\cdot6=1$

$\Z$ is a domain that's not a field. \\
\thm Let $R$ be a ring, $r\in R$ any element.  Then $r$ cannot be both a zero divisor and a unit. \\
\pf Say $r$ is a unit.  Then $r\cdot(1/r)=1$.  If $r$ is also a zero divisor, then $ra=0$ for some $a\neq0$, so:
\begin{align*}
ar(1/r) &= a \\
\implies 0 &= a
\end{align*}
Bad! \qed

\defn Let $R$, $S$ be rings.  Their direct sum is the ring $R\oplus S$.  The elements of $R\oplus S$ are the elements of $R\times S$, and the $+$ and $\cdot$ are:
\begin{align*}
(r_1,s_1)+(r_2,s_2) &= (r_1+r_2,s_1+s_2) \\
(r_1,s_1)(r_2,s_2) &= (r_1r_2,s_1s_2)
\end{align*}
\thm $R\oplus S$ is a ring. \\
\pf Dull.%
\begin{flalign*}
&&0 &\leftrightarrow (0,0) && \\
&&1 &\leftrightarrow (1,1) &&\qed
\end{flalign*}
%
$(1,0)\cdot(0,1)=(0,0)$ \\
If $R$, $S$ are nonzero, then $0\neq1$, so $R\oplus S$ is not an integral domain.
