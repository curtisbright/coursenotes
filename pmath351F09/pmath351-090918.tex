$2^A = \set{f}{A\to\brace{0,1}}$ \\
If $A$ has $n$ elements then $\abs{\P(A)}=2^n$ and $\abs{2^A}=2^n$

\thm $\abs{2^A}=\abs{\P(A)}$ for all sets $A$ \\
\pf Need to construct bijection $g\colon\P(A)\to2^A$ \\
Define $g(B)=1_B$ \\
$B\subseteq\P(A)$ i.e., $B\subseteq A$ \\
where $1_B(x) = \begin{cases}
1 & \text{$1$ if $x\in B$} \\
0 & \text{$0$ if $x\notin B$}
\end{cases}$ \\
$1_B \in 2^A$ \\
Check $g$ is 1--1 and onto. \\
First, if $B\neq C$ then $1_B\neq1_C$ so $g(B)\neq g(C)\implies\text{$g$ is 1--1}$ \\
\textbf{Onto:} Take $f\in2^A$ \\
Put $B=\set{x\in A}{f(x)=1}\implies f(x)=1_B(x)$ \\
Therefore $g(B)=f$ where $g$ is a bijection.

\textbf{Schroeder--Bernstein Theorem} \\
If $\abs A\leq\abs B$ and $\abs B\leq\abs A$ then $\abs A=\abs B$. \\
\pf Given injections $f\colon A\to B$ and $g\colon B\to A$.
\marginpar{figure: $D^\Co=g(f(D)^\Co)$ and $D=(g(f(E)^\Co))^\Co$}%
%
%Define $Q\colon\P(A)\to\P(A)$ by $E\mapsto(g(f(E)^\Co))^\Co$. \\
\begin{align*}
\text{Define } Q\colon & \P(A) \to \P(A) \\
& E \mapsto (g(f(E)^\Co))^\Co
\end{align*}
Want to find a set $D$ such that $Q(D)=D$. \\
First, if $E\subseteq F$ then $Q(E)\subseteq Q(F)$ because $f(E)\subseteq f(F)\implies f(E)^\Co\supseteq f(F)^\Co$ \\
$\implies g(f(E)^\Co)\supseteq g(f(F)^\Co)\implies\underbrace{(g(f(E)^\Co))^\Co}_{Q(E)}\subseteq\underbrace{(g(f(F)^\Co))^\Co}_{Q(F)}$ \\
Let $\D=\set{E\subseteq A}{E\subseteq Q(E)}$. \\
Take $D=\bigcup_{E\in\D} E$ \\
If $E\in\D$ then $E\subseteq D$ \\
$\implies Q(E)\subseteq Q(D)$ \\
Also $E\subseteq Q(E)\subseteq Q(D)$ for all $E\in\D$ \\
hence $D=\bigcup_{E\in\D} E \subseteq Q(D)$. \\
So $D\subseteq Q(D)\implies Q(D)\subseteq Q(Q(D))$ \\
therefore $Q(D)\in\D$. \\
So $Q(D)\subseteq D$. \\
Hence $Q(D) = D$ \\
i.e., $D=(g(f(D)^\Co))^\Co$ or $D^\Co=g(f(D)^\Co)$. \\
Now define $h\colon A\to B$ as follows:
\[ h(x) = \begin{cases}
f(x) & \text{if $x\in D$} \\
g^{-1}(x) & \text{for $x\in D^\Co$ and this is well defined because $D^\Co\subseteq\Range g$}
\end{cases} \]
%Hence $h$ is a bijection and $\abs A=\abs B$. \\
If $x\in D^\Co$ then $x\in g(f(D)^\Co)$. \\
$h$ is 1--1 since both $f|_D$ and $g^{-1}|_{D^\Co}$ are 1--1 and similarly is onto by construction. \\
Hence $h$ is a bijection and $\abs A=\abs B$.

\textbf{Consequences}
\begin{enumerate}
\item If $A_1\subseteq A_2\subseteq A_3$ and $\abs{A_1}=\abs{A_3}$ then also $\abs{A_2}=\abs{A_3}$. \\
\pf $\underbrace{A_2 \overset{\text{inj}}{\hookrightarrow} A_3}_{\text{embedding}}\implies\abs{A_2}\leq\abs{A_3}$
\[ \underbrace{A_3 \overset{\text{bij}}{\to} A_1 \overset{\text{inj}}{\hookrightarrow} A_2}_f \]
$f\colon A_3\to A_2$ is an injection $\implies\abs{A_3}\leq\abs{A_2}$ \\
By S--B, $\abs{A_3}=\abs{A_2}$.
\item $\abs{(0,1)}=\abs{[0,1)}=\abs\R$ \marginpar{figure: arctan}%
\\$[0,1)\subseteq[0,1)\subseteq\R$. \\
So enough to prove $(0,1)$ and $\R$ have same cardinality. \\
Let $f(x)=\arctan x$ by $f\colon\R\underset{\text{bij}}{\to}(-\frac{\pi}{2},\frac{\pi}{2})\overset{\text{lin}}{\underset{\text{bij}}{\to}}(0,1)$
\marginpar{figure: alternate definition of $f$, line between point $(0,1)$ and $r\in\R$, intersects circle with centre $(0,1)$ and radius $1$ at $f(r)$}%
\item $\abs\R=\abs{2^\N}$, another proof that $\R$ is uncountable. \\
Show $\abs{[0,1)}=\abs{2^\N}$. \\
Given $r\in[0,1)$ write its binary representation
\[ r = .a_1a_2a_3\ldots \qquad\text{(where $a_i=0$ or $1$)} \]
Define $f_r(n)=a_n$.  Then $f_r\colon\N\to\brace{0,1}$, i.e., $f_r\in2^\N$.% \\
%Define $\Phi\colon[0,1)\to2^\N$ by $r\mapsto f_r$. \\
\begin{align*}
\text{Define } \Phi\colon & [0,1) \to 2^\N \\
& r \mapsto f_r
\end{align*}
$\Phi$ is 1--1 because $r_1\neq r_2$, then there exists $n$ such that $n$th digits are different, so $f_{r_1}(n)\neq f_{r_2}(n)\implies f_{r_1}\neq f_{r_2}$.

\emph{But} $\Phi$ is \emph{not} onto because of non-uniqueness of binary representation.
%
%Define $\Lambda\colon2^\N\to[0,1)$ by $f\mapsto.0f(1)0f(2)0f(3)\ldots$ \\
\begin{align*}
\text{Define } \Lambda\colon & 2^\N \to [0,1) \\
& f \mapsto .\mspace{1mu}0\mspace{1mu}f(1)\mspace{1mu}0\mspace{1mu}f(2)\mspace{1mu}0\mspace{1mu}f(3)\mspace{1mu}\ldots
\end{align*}
$\Lambda$ is 1--1, since one of the binary representations of a number with two forms ends with a tail of $1$s, \emph{and} $\Lambda(f)$ never has a tail of $1$s.

Therefore, by Schroeder--Bernstein, $\abs{2^\N}=\abs{\R}$.
\end{enumerate}
