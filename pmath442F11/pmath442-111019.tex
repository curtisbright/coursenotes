\thm (Primitive Element) Let $K/F$ be a finite, separable field extension.  Then $K=F(\alpha)$ for some $\alpha\in K$. \\
\pf First, note that is enough to show that $K=F(\alpha)$ \emph{iff} $K/F$ has finitely many subextensions.  To see this, assume we had proven that $K=F(\alpha)$ \emph{iff} $K$ has finitely many $F$-subfields.  Then since $K/F$ is separable, there is a Galois extension $L/F$ with $K\subset L$.  By the Fundamental Theorem, $L$ has only finitely many $F$-subfields, so $K$ also has only finitely many $F$-subfields.  By our presumed fact, $K=F(\alpha)$ for some $\alpha\in K$.

\textbf{Forwards: }Assume $K=F(\alpha)$, and let $E\subset K$ be an $F$-subfield.  Let $p(x)\in F[x]$ be the monic minimal polynomial for $\alpha/F$.  Let $p(x)=p_1(x)\dotsm p_n(x)$ be a factorization of $p(x)$ into monic irreducibles in $E[x]$.  Let $E'$ be the $F$-field generated by the coefficients of the $p_i(x)$.  Note that $K=E(\alpha)=E'(\alpha)$ and $\alpha$ has the same minimal polynomial over $E$ and $E'$, so $[K:E]=[K:E']$, and hence $E=E'$ (since $E'\subset E$).

\textbf{Backwards: }Assume $K$ has only finitely many $F$-subfields. \\
\textbf{Case I: }$F$ is infinite.  Then it is enough to show that for any $\alpha$, $\beta$ in $K$, $F(\alpha,\beta)=F(\gamma)$ for some $\gamma\in K$.  Since $F$ is infinite, and since $K$ has only finitely many $F$-subfields there exist $c_1$, $c_2\in F$ such that $F(\alpha+c_1\beta)=F(\alpha+c_2\beta)$ \& $c_1\neq c_2$.  
\begin{align*}
\text{Then } \beta &= \frac{(\alpha+c_1\beta)-(\alpha+c_2\beta)}{c_1-c_2}\in F(\alpha+c_1\beta) \\
\text{and } \alpha &= (\alpha+c_1\beta)-c_1\beta \in F(\alpha+c_1\beta)
\end{align*}
so we may take $\gamma=\alpha+c_1\beta$. \\
\textbf{Case II: }$F$ finite, so $K$ finite.  By the classification of finite abelian groups, $K^*=K\setminus\brace0\cong(\Z/n\Z)\times\dotsb\times(\Z/n\Z)$ with $n_i\divides n_{i+1}$ for all $i<r$.  If $r\geq2$, then there are at least $n_1^2$ elements of $K^*$ with order dividing $n_1$.  This corresponds to at least $n_1^2$ different roots of $x^{n_1}-1$.  This is a problem if $n_1>1$, so we deduce that $r=1$ \& $K^*$ is cyclic.

So $K=F(\alpha)$ where $\alpha$ is a generator of the cyclic group $K^*$. \qed

Let's compute $\Gal(\Q(\zeta_n)/\Q)$.
\[ \zeta_n=\text{primitive $n$th root of unity} \]
\begin{align*}
\text{Well, } [\Q(\zeta_n):\Q] &= \phi(n) \\
&= \#(\Z/n\Z)^* \\
&= \#\set{a\in\brace{1,\dotsc,n}}{\gcd(a,n)=1}
\end{align*}
We will find $\phi(n)$ automorphisms of $\Q(\zeta_n)/\Q$, which will imply that $\Q(\zeta_n)/\Q$ is Galois.

Let $\zeta_n(x)=\text{$n$th cyclotomic polynomial}$.  The roots of $\zeta_n(x)$ are the primitive $n$th roots of unity.  They are all powers of $\zeta_n$, so $\Q(\zeta_n)$ is the splitting field for $\zeta_n(x)$ over $\Q$, and so $\Q(\zeta_n)/\Q$ is Galois.

\claim $\Gal(\Q(\zeta_n)/\Q)\cong(\Z/n\Z)^*$ %via $\sigma\mapsto$.
\begin{align*}
\text{via } \sigma &\stackrel\psi\mapsto \frac{\log\sigma(\zeta_n)}{\log\zeta_n} \\
&= a\text{, where $\sigma(\zeta_n)=\zeta_n^a$}
\end{align*}
\textbf{Proof of claim: }It is easy to check that $\psi$ is a homomorphism.  If $\psi(\sigma)=1$, then $\sigma(\zeta_n)=\zeta_n\implies\sigma=\id$, so $\psi$ is 1--1.  Since $\#\Gal(\Q(\zeta_n)/\Q)=\#(\Z/n\Z)^*=\phi(n)$, we see that $\psi$ is onto. \qed\ claim
