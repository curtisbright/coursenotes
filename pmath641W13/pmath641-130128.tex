Let $[K:\Q]=n$ and let $\sigma_1$, $\dotsc$, $\sigma_n$ be the embeddings of $K$ in $\C$ which fix $\Q$.  Let $\alpha\in K$.  The trace of $\alpha$ from $K$ to $\Q$, $T_\Q^K(\alpha)$ is given by $T_\Q^K(\alpha)=\sigma_1(\alpha)+\dotsb+\sigma_n(\alpha)$.

The norm $N_\Q^K(\alpha)$ is given by
\[ N_\Q^K(\alpha) = \sigma_1(\alpha)\dotsm\sigma_n(\alpha) . \]
Note $T_\Q^K$ is additive on $K$ since for $\alpha$, $\beta\in K$
\[ T_\Q^K(\alpha+\beta) = T_\Q^K(\alpha) + T_\Q^K(\beta) \]
and also
\[ N_\Q^K(\alpha\beta) = N_\Q^K(\alpha) N_\Q^K(\beta) . \]
Since the embeddings $\sigma_i$ fix elements of $\Q$, for $r\in\Q$ we have
\[ T_\Q^K(r\alpha) = \sigma_1(r\alpha) + \dotsb + \sigma_n(r\alpha) = r(\sigma_1(\alpha)+\dotsb+\sigma_n(\alpha)) = r T_\Q^K(\alpha) \]
and
\[ N_\Q^K(r\alpha) = r^n N_\Q^K(\alpha) . \]
Also note $\Q(\alpha)$ is contained in $K$ so we can consider $N_\Q^{\Q(\alpha)}(\alpha)$ and $T_\Q^{\Q(\alpha)}(\alpha)$.  These are coefficients in the minimal polynomial $\alpha$.

$\implies N_\Q^{\Q(\alpha)}(\alpha)$ and $T_\Q^{\Q(\alpha)}(\alpha)$ are in $\Q$ and are in $\Z$ if $\alpha$ is an algebraic integer.

\textbf{Theorem 19:} Let $K$ be a finite extension of $\Q$.  Let $\alpha\in K$ and let $l=[K:\Q(\alpha)]$.  Then
\[ T_\Q^K(\alpha) = l T_\Q^{\Q(\alpha)}(\alpha) \]
and
\[ N_\Q^K(\alpha) = (N_\Q^{\Q(\alpha)}(\alpha))^l . \]
\pf Each of the embeddings of $\Q(\alpha)$ in $\C$ which fix $\Q$ extend to $l$ distinct embeddings of $K$ in $\C$ which fix $\Q$ by Theorem 4.  The result follows.

\textbf{Theorem 20:} Let $K$ be a finite extension of $\Q$ and let $\alpha\in A\cap K$.
\[ \text{$\alpha$ is a unit in $A\cap K$} \iff N_\Q^K(\alpha) = \pm 1 . \]
\pf
\begin{itemize}
\item[$\Rightarrow$] Since $\alpha$ is a unit there is a $\beta\in A\cap K$ with $\alpha\beta=1$.  Thus $N_\Q^K(\alpha\beta)=\N_\Q^K(1)=1$.  But $N_\Q^K(\alpha\beta)=N_\Q^K(\alpha)N_\Q^K(\beta)$ and since $\alpha$, $\beta\in\A\cap K$ we see that $N_\Q^K(\alpha)$, $N_\beta^K\in\Z$.  Hence $N_\Q^K(\alpha)=\pm1$.
\item[$\Leftarrow$] Suppose $N_\Q^K(\alpha)=\pm1$.  Then let $\sigma_1(\alpha)=\alpha$, $\sigma_2(\alpha)$, $\dotsc$, $\sigma_n(\alpha)$ be the images of $\sigma_i$.

Thus
\[ \alpha((-1)^t\sigma_2(\alpha)\dotsm\sigma_n(\alpha)) = 1 \]
where $t\in\brace{0,1}$.  But $\beta=(-1)^t\sigma_2(\alpha)\dotsm\sigma_n(\alpha)$ is in $\A\cap K$ since $\beta=\frac{1}{\alpha}\in K$ and $\sigma_i(\alpha)$ is an algebraic integer for $i=2$, $\dotsc$, $n$ hence $\beta\in\A$.  Thus
\[ \beta \in \A \cap K . \]
\end{itemize}
Theorem 20 $\implies$ The set of units in $\A\cap K$ is a group under multiplication hence a subgroup of $\C$.  What happens in $A\cap\Q(\sqrt{D})$ when $D$ is a squarefree integer with $D\neq1$?

What is the unit group? \\
If $D\nequiv1\pmod4$ then to determine the unit group we must find all elements $l+m\sqrt{D}$ with $l$, $m\in\Z$ for which
\[ N_\Q^{\Q(\sqrt{D})}(l+m\sqrt{D}) = \pm 1 \tag{1} \]
hence for which $(l+m\sqrt{D})(l-m\sqrt{D})=\pm1\implies l^2-Dm^2=\pm1$.  If $D\equiv1\pmod4$ then we must also consider $\frac{l+m\sqrt{D}}{2}$ with $l$ and $m$ odd integers.  Hence
\[ N_\Q^{\Q(\sqrt{D})}\paren[\Big]{\frac{l+m\sqrt{D}}{2}} = \frac{l^2-Dm^2}{4} = \pm 1 \implies l^2-Dm^2 = \pm4 . \tag{2} \]
\textbf{Theorem 21:} Let $D$ be a squarefree negative integer.  The units in $\A\cap\Q(\sqrt{D})$ are $\pm1$ unless $D=-1$ in which case the units are $\pm1$, $\pm i$ or $D=-3$ in which case the units are $\pm1$, $\frac{\pm1\pm\sqrt{-3}}{2}$.  Since $D$ is negative we need only consider
\[ l^2 - Dm^2 = + 1 \text{ in (1)} \]
and
\[ l^2 - Dm^2 = + 4 \text{ in (2)} . \]
If $-D\neq1$ or $-3$ then the only solution of~(1) in integers $l$ and $m$ is given by $l=\pm1$, $m=0$.  Similarly if $D\equiv1\pmod4$ and $D\neq-3$ there are no solutions of~(2) with $l$ odd.  If $D=-1$ then~(1) has the solutions $l=\pm1$, $m=0$ and $l=0$, $m=\pm1$.

If $D=-3$ and $l$ and $m$ are odd then the solutions $(l,m)$ are given by $(\pm1,\pm1)$.  Further if $D=-3$ then~(1) has only the solutions $l=\pm1$, $m=0$ in integers $l$, $m$.

\textbf{Theorem 22:} Let $D$ be a squarefree integer larger than $1$.  There is a unit $\epsilon$ in $\Q(\sqrt{D})$ larger than $1$ with the property that the group of units in $\Q(\sqrt{D})$ is
\[ \set{(-1)^j\epsilon^k}{j,k\in\Z} . \]