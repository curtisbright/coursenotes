\textbf{Riemann Zeta Function}
\[ \zeta(z) = \sum_{n=1}^\infty\frac{1}{n^z} \qquad \text{for $\Re(z)>1$} \]
For $z=x+iy$
\[ \abs{n^z} = \abs{e^{z\log n}} = \abs{e^{x\log n+iy\log n}} = e^{x\log n} = n^x = n^{\Re(z)} . \]
\begin{align*}
\abs[\Big]{\sum_{n=1}^\infty\frac{1}{n^z}} &\leq \sum_{n=1}^\infty\abs[\Big]{\frac{1}{n^z}} \\
&= \sum_{n=1}^\infty \frac{1}{n^{\Re(z)}}
\end{align*}
This converges by the Integral Test when $\Re(z)>1$.
\begin{align*}
\text{(Indeed, } \int_1^\infty\frac{1}{t^x}\d t &= \brack[\Big]{\frac{t^{-x+1}}{-x+1}}_1^\infty \\
&= \brack[\Big]{\frac{-1}{(x-1)t^{x-1}}}_1^\infty \\
&= \frac{1}{x-1} \text{ for $x>1$)}
\end{align*}
Thus $\sum_{n=1}^\infty\frac{1}{n^z}$ converges absolutely for $\Re(z)>1$ so $\zeta(x)=\sum_{n=1}^\infty\frac1{n^z}$ is well defined for $\Re(z)>1$

\thm (Euler Product Formula)
\[ \zeta(z) = \sum_{n=1}^\infty\frac1{n^z} = \prod_p\frac{1}{1-\frac{1}{p^z}} = \prod_{i=1}^\infty\frac{1}{1-\frac{1}{p_i^z}} \text{ where $p_i$ is the $i$th prime} \]
for $\Re(z)>1$. \\
\pf The $l$th partial product is
\begin{align*}
P_l &= \prod_{i=1}^l \frac{1}{1-\frac{1}{p_i^z}} \\
&= \prod_{i=1}^l\paren{1+\frac1{p_i^z}+\frac1{p_i^{2z}}+\dotsb} \qquad \text{(since $\abs[\Big]{\frac1{p^z}}=\frac{1}{p^{\Re(z)}}\leq\frac1{2^1}=\frac12$)} \\
&= \prod_{i=1}^l\sum_{k_i=0}^\infty \frac{1}{p_i^{k_iz}} \\
&= \sum_{k_1\geq0,\dotsc,k_l\geq0}\frac{1}{p_1^{k_1z}p_2^{k_2z}\dotsm p_l^{k_lz}} \text{ (since the infinite series converge absolutely)}\footnote{\[
\paren[\Big]{\sum_i a_i}\paren[\Big]{\sum_j b_j} = \sum_{i,j}a_ib_j
\] when we have absolute convergence} \\
&= \sum_{n\in A_l}\frac{1}{n^z} \text{ where $A_l$ is the set of positive integers whose prime factors are in $\brace{p_1,\dotsc,p_l}$} \\
\abs{\zeta(z)-P_l(z)} &= \abs[\Big]{\sum_{n\in\Z^+}\frac{1}{n^z}-\sum_{n\in A_l}\frac{1}{n^z}} \\
&= \abs[\Big]{\sum_{n\notin A_l}\frac{1}{n^z}} \\
&\leq \sum_{n\notin Z_l}\frac{1}{n^x} \text{ where $x=\Re(z)>1$} \\
&\leq \sum_{n>p_l}\frac{1}{n^x} \to 0 \text{ as $l\to\infty$ since $\sum\frac1{n^x}$ converges and $p_l\to\infty$}
\end{align*}
\eg\begin{align*}
\zeta(z) \sum_{n=1}^\infty \frac{\mu(n)}{n^z} &= \sum_{k=1}^\infty \frac{1}{k^z} \sum_{l=1}^\infty \frac{\mu(l)}{l^z} \\
&= \sum_{k,l} \frac{\mu(l)}{(kl)^z} \text{ (by abs convergence)} \\
&= \sum_n\sum_{d\div n}\frac{\mu(d)}{n^z} \\
&= \sum_n\frac{1}{n^z}\sum_{d\div n}\mu(d) \\
&= \sum_n\frac{1}{n^z}\underbrace{I(n)}_{\text{$=1$ only when $n=1$}} \\
&= \frac{1}{1^z} = 1
\end{align*}
$\therefore\zeta(z)\neq0$ for all $\Re(z)>1$ and $\frac{1}{\zeta(z)}=\sum_{n=1}^\infty\frac{\mu(n)}{n^z}$

\ex Let $f\colon\Z^+\to\C$.  Suppose $\sum_{n=1}^\infty f(n)$ converges absolutely.  Show that if $f$ is multiplicative then
\[ \sum_{n=1}^\infty f(n) = \prod_p (1+f(p)+f(p^2)+f(p^3)+\dotsb) \]
and if $f$ is completely multiplicative then
\[ \sum_{n=1}^\infty f(n) = \prod_p\frac{1}{1-f(p)} \]
\ex Apply this formula to $f(n)=\frac{\mu(n)}{n^z}$.

Calculation of $\zeta(2n)$ for $n\in\Z^+$.
\[ \zeta(2n) = \sum_{n=1}^\infty \frac{1}{k^{2n}} \]
\textbf{Euler's Informal Calculation} \\
For a polynomial with $n$ distinct roots $\alpha_i$
\[ f(x) = a_n \prod (x-\alpha_i) \]
If the $\alpha_i\neq0$
\[ f(x) = a_0 \prod\paren[\Big]{1-\frac{x}{\alpha_i}} \]
\[ f(x) = f(0) \prod_i\paren[\Big]{1-\frac{x}{\alpha_i}} \]
By analogy,
\begin{align*}
\frac{\sin x}{x} &= 1 \prod_{0\leq k\in\Z}\paren[\Big]{1-\frac{x}{kx}} \\
&= \prod_{k=1}^\infty \paren[\Big]{1-\frac{x^2}{(k\pi)^2}}
\end{align*}
\eg when $x=\frac\pi2$,
\begin{align*}
\frac2\pi &= \prod\paren[\Big]{1-\frac{\pi^2/4}{(k\pi)^2}} \\
&= \prod\paren[\Big]{1-\frac{1}{4k^2}} = \prod\paren[\Big]{\frac{4k^2-1}{4k^2}} \\
&= \prod\paren[\Big]{\frac{2k-1}{2k}\cdot\frac{2k+1}{2k}} \\
&= \frac12\cdot\frac32\cdot\frac34\cdot\frac54\cdot\frac56\cdot\frac76\dotsm
\end{align*}
This is Wallis' Formula.
\begin{gather*}
\frac{\sin x}{x} = \prod_{k=1}^\infty\paren[\Big]{1-\frac{x^2}{(k\pi)^2}} \\
1 - \frac16x^2 + \frac1{24}x^2 - \dotsb = 1 - \sum_{k=1}^\infty\frac1{(k\pi)^2}x^2 + \dotsb \\
\therefore \sum_{k=1}^\infty \frac{1}{k^2} = \frac{\pi^2}{6} \\
\log\frac{\sin x}{x} = \sum_{k=1}^\infty\log\paren[\Big]{1-\frac{x^2}{(k\pi)^2}} \\
\frac{x\cos x-\sin x}{x^2}\cdot\frac{x}{\sin x} = \sum_{k=1}^\infty\frac{-\frac{2x}{(k\pi)^2}}{1-\frac{x^2}{(k\pi)^2}} \\
x\cot x - 1 = -2\sum_{k=1}^\infty \frac{\frac{x^2}{(k\pi)^2}}{1-\frac{x^2}{(k\pi)^2}}
\end{gather*}
\begin{align*}
x\cot x &= 1 - \sum_{k=1}^\infty\sum_{n=1}^\infty\paren[\Big]{\frac{x^2}{(k\pi)^2}}^n \\
&= 1 - 2\sum_{n=1}^\infty \sum_{k=1}^\infty \frac{x^{2n}}{(k\pi)^{2n}} \\
&= 1 - \frac{2}{\pi^{2n}} \sum_{n=1}^\infty \underbrace{\paren[\Big]{\sum_{k=1}^\infty\frac{1}{k^{2n}}}}_{\zeta(2n)} x^{2n} \\
\therefore \zeta(n) &= \sum_{k=1}^\infty \frac1{k^{2n}} \\
&= -\frac{\pi^{2n}}{2}\paren{\text{coefficient of $x^{2n}$ in $x\cot x$}}
\end{align*}
To find this coefficient write $x\cot x=\sum_{n=0}^\infty c_{2n}x^{2n}$
\begin{gather*}
x\cot x = \frac{\cos x}{\frac{\sin x}{x}} \\
\frac{\sin x}{x}\cdot x\cot x = \cos x \\
(1-\frac1{3!}x^2+\frac1{5!}x^4-\dotsb)(c_0+c_2x^2+c_4x^4+\dotsb) = 1 - \frac{1}{2!}x^2 + \frac{1}{4!}x^4 - \dotsb \\
c_0 = 1 \\
c_2 - \frac1{3!}c_0 = -\frac1{2!} \qquad c_2 = -\frac12 + \frac16 = -\frac13 \\
c_4 - \frac1{3!}c_2 + \frac1{5!}c_0 = \frac1{4!} \qquad c_4 = \frac1{24}-\frac1{18}-\frac1{120} = \frac{15-20-3}{360} = \frac{-8}{360} = \frac{-1}{45} \\
\zeta(2) = -\frac{\pi^2}{2}\paren[\Big]{-\frac{1}{3}} = \frac{\pi^2}{6} \\
\zeta(4) = -\frac{\pi^4}{2}\paren[\Big]{-\frac{1}{45}} = \frac{\pi^4}{90}
\end{gather*}
