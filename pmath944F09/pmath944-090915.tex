Cam Stewart MC 5051

\textbf{Books:} \\
Geometry of Numbers: J. W. S. Cassels \\
Geometry of Numbers: Lekkerkerker \\
Sphere Packings, Lattices and Groups: Conway \& Sloane

Let $\v_1$, $\dotsc$, $\v_n$ be linearly independent vectors in $\R^n$.  The set
\[ \Lambda = \set{ a_1\v_1+\dotsb+a_n\v_n }{ (a_1,\dotsc,a_n)\in\Z^n } \]
is said to be a lattice with basis $\v_1$, $\dotsc$, $\v_n$.

Note that since $\v_1\c \dotsc\c \v_n$ are linearly independent, each element of $\Lambda$ has a unique representation as a linear combination of $\v_1\c \dotsc\c \v_n$.  Further, the coefficients in the representation are integer.

Observe that the basis $\v_1\c \dotsc\c \v_n$ is not uniquely determined by $\Lambda$.  In particular, let $A$ be an $n\times n$ matrix with integer entries and determinant $\pm1$ and put
\[ A \begin{pmatrix}\v_1\\\vdots\\\v_n\end{pmatrix} = \begin{pmatrix}\w_1\\\vdots\\\w_n\end{pmatrix} \]
We claim that $\w_1\c \dotsc\c \w_n$ is also a basis for $\Lambda$.  Certainly $\w_1\c \dotsc\c \w_n$ are linearly independent vectors in $\R^n$.  Secondly note that
\[ \begin{pmatrix}\v_1\\\vdots\\\v_n\end{pmatrix} = A^{-1}\begin{pmatrix}\w_1\\\vdots\\\w_n\end{pmatrix} \qquad\text{and}\qquad A^{-1}=\frac{1}{\det(A)}\adj A \]
Recall that the $i,j$-th entry of $\adj A$ is the cofactor of $a_{j,i}$.  But the cofactor is an integer and $\det(A)=\pm1$ so $\v_i$ can be expressed as an integer linear combination of $\w_1\c \dotsc\c \w_n$.  Thus every element of $\Lambda$ is an integer linear combination of $\w_1\c \dotsc\c \w_n$.  Thus $\w_1\c \dotsc\c \w_n$ is also a basis for $\Lambda$.

Suppose now that $\v_1\c \dotsc\c \v_n$ is a basis for $\Lambda$ and that $\w_1\c \dotsc\c \w_n$ is also a basis for $\Lambda$.  We'll now show that they are related as above.  In particular since $\v_1\c \dotsc\c \v_n$ is a basis we can express $\w_i$ for $i=1\c \dotsc\c n$ as an integer linear combination of $\v_1\c \dotsc\c \v_n$.  Thus there is an $n\times n$ matrix $A$ with integer entries such that
\[ A \begin{pmatrix}\v_1\\\vdots\\\v_n\end{pmatrix} = \begin{pmatrix}\w_1\\\vdots\\\w_n\end{pmatrix} \]
Similarly, there is an $n\times n$ matrix $B$ with integer entries such that
\[ B \begin{pmatrix}\w_1\\\vdots\\\w_n\end{pmatrix} = \begin{pmatrix}\v_1\\\vdots\\\v_n\end{pmatrix} \]
Therefore
\[ AB \begin{pmatrix}\w_1\\\vdots\\\w_n\end{pmatrix} = \begin{pmatrix}\w_1\\\vdots\\\w_n\end{pmatrix} \]
hence $AB=I$ so $\det A\cdot\det B=1$.  But $\det A$ and $\det B$ are integers so $\det A=\pm1$.

We are now in a position to define the determinant $d(\Lambda)$ of $\Lambda$.  Let $\v_1\c \dotsc\c \v_n$ be a basis for $\Lambda$.  We put
\[ d(\Lambda) = \abs{\det(\v_1,\dotsc,\v_n)} . \]
Here $(\v_1,\dotsc,\v_n)$ represents the matrix obtained by writing the $\v_i$s with respect to the standard basis $(1,0,\dotsc,0)$, $(0,1,0,\dotsc,0)$, $\dotsc$, $(0,\dotsc,0,1)$ in $\R^n$.

Notice that $d(\Lambda)$ does not depend on the choice of basis $\v_1\c \dotsc\c \v_n$ since if $\w_1\c \dotsc\c \w_n$ is another basis for $\Lambda$ then there is a matrix $A$ with $\det(A)=\pm1$ such that
\[ A \begin{pmatrix}\v_1\\\vdots\\\v_n\end{pmatrix} = \begin{pmatrix}\w_1\\\vdots\\\w_n\end{pmatrix} \]
and we see that
\[ \abs{\det(\w_1,\dotsc,\w_n)} = \abs{\det(A)}\cdot\abs{\det(\v_1,\dotsc,\v_n)} = \abs{\det(\v_1,\dotsc,\v_n)} \]
\remark Since $\v_1\c \dotsc\c \v_n$ are linearly independent we see that $d(\Lambda)>0$.

The simplest lattice in $\R^n$ is $\Lambda_0$ where $\Lambda_0$ is generated by $(1,0,\dotsc,0)\c (0,1,0,\dotsc,0)\c \dotsc\c (0,\dotsc,0,1)$.  Then $d(\Lambda_0)=1$.

Let $\Lambda$ and $\Lambda_1$ be lattices in $\R^n$.  If $\Lambda_1\subseteq\Lambda$ then $\Lambda_1$ is said to be a sublattice of $\Lambda$.  Note that if $\w_1\c \dotsc\c \w_n$ are linearly independent vectors in a lattice $\Lambda$ in $\R^n$ then they generate a sublattice $\Lambda_1$ of $\Lambda$ and there is a matrix $A$ with integer entries such that
\[ A \begin{pmatrix}\v_1\\\vdots\\\v_n\end{pmatrix} = \begin{pmatrix}\w_1\\\vdots\\\w_n\end{pmatrix} . \]
Let $D=\abs{\det(A)}$ and note that $D$ is a positive integer.  Further
\[ D = \frac{\abs{\det(\w_1,\dotsc,\w_n)}}{\abs{\det(\v_1,\dotsc,\v_n)}} = \frac{\abs{\det(\w_1,\dotsc,\w_n)}}{d(\Lambda)} = \frac{d(\Lambda_1)}{d(\Lambda)} \]
where $\Lambda_1$ is the lattice generated by $\w_1\c \dotsc\c \w_n$.  $D$ is known as the index of $\Lambda_1$ in $\Lambda$.

Suppose that $\Lambda$ is a lattice in $\R^n$ and $\Lambda_1$ is a sublattice of $\Lambda$ of index $D$.  Let $\v_1\c \dotsc\c \v_n$ be a basis for $\Lambda$ and $\w_1\c \dotsc\c \w_n$ be a basis for $\Lambda_1$.  Then we have a matrix $A$ with integer entries and $\abs{\det(A)}=D$ such that
\[ A \begin{pmatrix}\v_1\\\vdots\\\v_n\end{pmatrix} = \begin{pmatrix}\w_1\\\vdots\\\w_n\end{pmatrix} . \]
Thus
\[ \begin{pmatrix}\v_1\\\vdots\\\v_n\end{pmatrix} = \frac{1}{\det A} \adj A \begin{pmatrix}\w_1\\\vdots\\\w_n\end{pmatrix} , \]
and so
\[ \begin{pmatrix}D \v_1\\\vdots\\D \v_n\end{pmatrix} = \frac{D}{\det A} \adj A \begin{pmatrix}\w_1\\\vdots\\\w_n\end{pmatrix} \]
hence $D\v_i$ is an integer linear combination of $\w_1\c \dotsc\c \w_n$ for $i=1\c \dotsc\c n$.  In particular $D\v_i\in\Lambda_1$ for $i=1\c \dotsc\c n$.

\textbf{Theorem 1:} Let $\Lambda_1$ be a sublattice of the lattice $\Lambda$ in $\R^n$.
%alpha list
\begin{enumerate}
\item[A)] If $\v_1\c \dotsc\c \v_n$ is a basis for $\Lambda$ then there is a basis $\w_1\c \dotsc\c \w_n$ of $\Lambda_1$ such that
\begin{equation}
\begin{aligned}
\w_1 &= a_{11}\v_1 \\
\w_2 &= a_{21}\v_1 + a_{22}\v_2 \\
&\eqvdots \\
\w_n &= a_{n1}\v_1 + \dotsb + a_{nn}\v_n
\end{aligned}
\label{star090915}
\end{equation}
where
\begin{enumerate}
\item[i)] the $a_{ij}$s are integers
\item[ii)] $a_{ii}>0$ for $i=1\c \dotsc\c n$
\item[iii)] $0\leq a_{ij} < a_{jj}$ for $1\leq j < i \leq n$.
\end{enumerate}
\item[B)] If $\w_1\c \dotsc\c \w_n$ is a basis $\Lambda_1$ then there is a basis $\v_1\c \dotsc\c \v_n$ for $\Lambda$ such that \eqref{star090915} holds with
\begin{enumerate}
\item[i)] the $a_{ij}$s are integers
\item[ii)] $a_{ii}>0$ for $i=1\c \dotsc\c n$
\item[iii)$'$] $0\leq a_{ij} < a_{ii}$ for $1\leq j < i \leq n$.
\end{enumerate}
\end{enumerate}

\textbf{Proof:}
\begin{enumerate}
\item[A)] Let $D$ be the index of $\Lambda_1$ in $\Lambda$.  For each $i$ with $1\leq i\leq n$ there exist vectors
\[ \w_i = a_{i1}\v_1 + \dotsb + a_{ii}\v_i \]
in $\Lambda_1$ with $a_{ij}\in\Z$ and $a_{ii}>0$ since $D\v_i\in\Lambda_1$.  We choose $\w_i$ for $i=1\c \dotsc\c n$ in such a way that $a_{ii}$ is positive and as small as possible.  Since $\w_1\c \dotsc\c \w_n$ are in $\Lambda_1$ we have $b_1\w_1+\dotsb+b_n\w_n$ in $\Lambda_1$ for any integers $b_1\c \dotsc\c b_n$.

We claim that $\w_1\c \dotsc\c \w_n$ forms a basis for $\Lambda_1$.

If not then there is a vector $\z$ in $\Lambda_1$ which is not of the form $b_1\w_1+\dotsb+b_n\w_n$ with $b_1\c \dotsc\c b_n$ integers.  Then there exist integers $c_1\c \dotsc\c c_n$ such that $\z=c_1\v_1+\dotsb+c_n\v_n$.  We now choose $\z$ in $\Lambda_1$ for which the representation has $c_{i+1}=\dotsb=c_n=0$ with $i$ minimal.  In particular $\z=c_1\v_1+\dotsb+c_i\v_i$.

Let $c_i=qa_{ii}+r$ with $0\leq r<a_{ii}$.  Then
\[ \z-q\w_i = (c_1-qa_{ii})\v_1+\dotsb+r\v_i . \]
Note that $\z-q\w_i\in\Lambda_1$ and is an integer linear combination of $\v_1\c \dotsc\c \v_i$.  Further note that $r\neq0$ since $i$ is minimal.  But this contradicts the minimal choice of $a_{ii}$.  Thus $\w_1\c \dotsc\c \w_n$ forms a basis for $\Lambda_1$.

It remains to check that iii) holds.  To obtain iii) we replace $\w_i$ by $\w'_i$ for $i=1\c \dotsc\c n$ where
\[ \w'_i = b_{i1}\w_1 + \dotsb + b_{i,i-1}\w_{i-1} + \w_i , \]
with the $b_{ij}$s integers to be chosen.  Note that $\w'_1\c \dotsc\c \w'_n$ is a basis for $\Lambda_1$ and that
\[ \w'_i = a'_{i1}\v_1 + \dotsb + a'_{ii}\v_i \]
with $a'_{ii}=a_{ii}$ for $i=1\c \dotsc\c n$.  Further for $j<i$ we have
\[ a'_{ij} = b_{ij}a_{jj} + b_{i,j+1}a_{j+1,j} + \dotsb + b_{i,i-1}a_{i-1,j} + a_{ij} . \]
For each $i$ we now choose $b_{i,i-1}\c b_{i,i-2}\c \dotsc\c b_{i,1}$ in that order so that $0\leq a'_{ij}<a_{jj}=a'_{jj}$ as required.
\end{enumerate}