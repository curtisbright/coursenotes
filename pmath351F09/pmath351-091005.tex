\defin $A\subseteq X$ is \emph{compact} if every open cover of $A$ has a finite subcover.

e.g., $\R$ not compact: $\set{(-n,n)}{n\in\N}$ is an open cover with no finite subcover.

e.g., $(0,1)$ not compact: $\set{(1/n,1-1/n)}{n=2,3,\dotsc}$

e.g., $X$ any metric space \\
$A=\brace{a_1,\dotsc,a_N}$ any finite set is compact \\
\pf Let $\brace{G_\alpha}$ be an open cover of $A$ \\
For each $j=1,\dotsc,N$ there exists $G_{\alpha_j}$ from the collection such that $a_j\in G_{\alpha_j}$.  Then $G_{\alpha_1},\dotsc,G_{\alpha_N}$ are a finite subcover of $A$.

e.g., $X$ discrete metric space.  Then $A\subseteq X$ is compact if and only if $A$ is finite.
\begin{itemize}
\item Saw on Friday that infinite sets in discrete metric space are not compact: just take $\set{B(a,1)}{a\in A}$
\end{itemize}

\textbf{Characterization of compactness in $\pmb\R^n$:} \\
\thm For $A\subseteq\R^n$ the following are equivalent:
\begin{enumerate}
\item[(1)] $A$ is compact
\item[(2)] $A$ is closed and bounded\footnote{(1) and (2): Heine--Borel}
\item[(3)] Every sequence from $A$ has a convergent subsequence with the limit in $A$\footnote{(1) and (3): Bolzano--Weierstrass}
\end{enumerate}
Heine--Borel Theorem does not hold true in general metric spaces.

\prop Compact sets in metric spaces are alwasys closed. \\
\pf Let $K$ be a compact set.  Want to prove $K^\Co$ is open. \\
Let $x\in K^\Co$. \\
For all $y\in K$ there exists $r_y>0$ such that
\[ B(x,r_y) \cap B(y,r_y) = \emptyset \]
Consider $\set{B(y,r_y)}{y\in K}$: open cover of $K$ \\
$K$ is compact so there exists a finite subcover, \\
i.e., there exists $B(y_1,r_{y_1}),\dotsc,B(y_N,r_{y_N})$ such that
\[ \bigcup_{j=1}^N B(y_j,r_{y_j}) \supseteq K . \]
Let $r=\min(r_{y_1},\dotsc,r_{y_N})>0$. \\
Claim $B(x,r)\cap K=\emptyset$. \\
Say $z\in B(x,r)\cap K$.  Then there exists $j\in\brace{1,\dotsc,N}$ such that $z\in B(y_j,r_{y_j})$.  So $z\in B(x,r)\cap B(y_j,r_{y_j})$, but $B(x,r)\subseteq B(x,r_{y_j})$, i.e., $z\in B(x,r_{y_j})\cap B(y_j,r_{y_j})=\emptyset$ by construction. \\
Contradiction.  Hence $B(x,r)\subseteq K^\Co$ $\implies$ $K^\Co$ is open $\iff$ $K$ is closed.

\prop Closed subsets of compact sets are compact. \\
\pf Let $F$ be a closed subset of compact set $X$. \\
Take an open cover $\brace{G_\alpha}$ of $F$. \\
Then the collection of sets $G_\alpha$ together with the open set $F^\Co$ is an open cover of $X$.\footnote{$\bigcup G_\alpha\cup F^\Co\supseteq F\cup F^\Co=X$} \\
Let $G_{\alpha_1},\dotsc,G_{\alpha_N},(F^\Co)$\footnote{(because $X$ is compact)} be a finite subcover of $X$. \\
Then $G_{\alpha_1},\dotsc,G_{\alpha_N}$ must cover $F$. \\
So the open cover $\brace{G_\alpha}$ of $F$ has a finite subcover. \\
Hence $F$ is compact.

\prop Compact sets (in metric spaces) are bounded. \\
\pf Let $K$ be compact set and let $x_0\in K$. \\
Consider all balls $B(x_0,n)$, $n=1,2,3,\dotsc$ \\
If $k\in K$ then $d(x_0,k)<n_0$ for some large enough integer $n_0$ \\
i.e., $k\in B(x_0,n_0)$.  Therefore
\[ k \in \bigcup_{n=1}^\infty B(x_0,n) \]
\[ \implies K \subseteq \bigcup_{n=1}^\infty B(x_0,n) \]
Hence $\set{B(x_0,n)}{n=1,2,\dotsc}$ is an open cover of $K$. \\
Since $K$ is compact there must be a finite subcover, say $B(x_0,n_1),\dotsc,B(x_0,n_L)$. \\
Say $n_L=\max(n_1,\dotsc,n_L)$ \\
Then $B(x_0,n_L)\supseteq B(x_0,n_j)$ for $j=1,2,\dotsc,L$ \\
$\implies K \subseteq B(x_0,n_L) = \bigcup_1^L B(x_0,n_j)$ \\
Hence $K$ is bounded.

\defin \emph{$\epsilon$-net}: for $A\subseteq$ metric space $X$ is a finite set $x_1,\dotsc,x_n\in X$ such that every element of $A$ has distance at most $\epsilon$ from at least one $x_j$. \\
i.e., for all $a\in A$ there exists $j\in\brace{1,\dotsc,n}$ such that $d(a,x_j)\leq\epsilon$. \\
If take $\epsilon'>\epsilon$ then $\bigcup_{j=1}^n B(x_j,\epsilon')\supseteq A$.

\defin Say $A$ is \emph{totally bounded} if for all $\epsilon>0$ there exists $\epsilon$-net for $A$. \\
e.g., $X$ discrete metric space. \\
There is a $1$-net (consisting of one element) \\
But no $1-\epsilon$ net if $X$ is infinite. \\
So if $X$ is infinite it is not totally bounded.

\prop Totally bounded $\implies$ bounded. \\
\pf Take a $1$-net for the totally bounded set $A$, say $x_1,\dotsc,x_k$. \\
$\implies \bigcup_{j=1}^k B(x_j,3/2) \supseteq A$ \\
Take $B(x_1,\underbrace{\max_{j=1,\dotsc,k} d(x_1,x_j)+1+3/2}_r) \supseteq B(x_j,3/2)$ for all $j$. \\
Then $A\subseteq B(x_1,r)$
