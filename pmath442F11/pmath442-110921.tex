Cyclotomic extensions \\
Let $n$ be an integer, $\zeta_n\in\C$ a primitive root of unity; \textit{i.e.,} $\zeta_n=(e^{2\pi i/n})^a$ for some integer $a$ prime to $n$.  The $n$th cyclotomic extension of $\Q$ is $\Q(\zeta_n)$.  Note that this is independent of $a$.
\[ \begin{array}{c|c|c}
n & \Q(\zeta_n) & \text{degree over $\Q$} \\ \hline
1 & \Q & 1 \\
2 & \Q & 1 \\
3 & \Q(\zeta_3) = \Q(\sqrt{-3}) & 2 \\
4 & \Q(i) & 2 \\
5 & & 4 \\
6 & \Q(\sqrt{-3}) & 2 \\
\vdots & & \vdots \\
n & & \phi(n)
\end{array} \]
\defn The group $\mu_n$ is the group of $n$th roots of unity with respect to multiplication. \\
We have $\mu_n\cong C_n$ (or $\Z/n\Z$), with generator $e^{2\pi i/n}$, via:
\[ e^{2\pi i a/n} \mapsto a \bmod n \]
Note $\Q(\zeta_n)=\Q(\mu_n)$. \\
Note that if $d\divides n$, then $\mu_d\subset\mu_n$. \\
\defn The $n$th cyclotomic polynomial is
\begin{align*}
x^n-1 &= \prod_{\alpha\in\mu_n}(x-\alpha)=\prod_{a=1}^n(x-e^{2\pi ia/n}) \\
\phi_n(x) &= \prod_{(a,n)=1}(x-e^{\pi ia/n})
\end{align*}
Note that $x^n-1=\prod_{d\divides n}\phi_d(x)$ \\
Note $\phi_n(x)$ has degree $\phi(n)={}$\# integers prime to $n$ between $0$ and $n$.

\thm $\phi_n(x)\in\Z[x]$, and is primitive.
\pf By induction on $n$.  If $n=1$, $\phi_n(x)=x-1$ and we're done. \\
Now assume $\phi_k(x)\in\Z[x]$ for all $k<n$, and consider $\phi_n(x)$.  We have
\begin{align*}
x^n-1 &= \prod_{d\divides n}\phi_d(x) \\
&= \phi_n(x) \prod_{\substack{d\divides n\\d\neq n}}\phi_d(x)
\end{align*}
Since $x^n-1$, $\phi_d(x)\in\Z[x]$ for $d<n$, we deduce $\phi_n(x)\in\Q[x]$.  Since $\Z$ is a UFD and since $\prod\phi_d(x)$ is primitive (by Gauss' Lemma), we conclude by Gauss' Lemma that $\phi_n(x)\in\Z[x]$.  $\phi_n(x)$ is primitive because it's monic. \qed

\thm $\phi_n(x)$ is irreducible over $\Q$. \\
\pf By Gauss' Lemma, it suffices to show that $\phi_n(x)$ is irreducible over $\Z$.  Assume $\phi_n(x)=f(x)g(x)$ for irreducible $f(x)$ over $\Q$, $f(x)$, $g(x)\in\Z[x]$.  Let $\zeta_n$ be come primitive $n$th root of unity.  Note that if $p$ is prime, $p\ndivides n$, then $\phi_n(\zeta_n^p)=0$.  $f(\zeta_n)=0$ \\
Since $x^n-1$ is separable, so is $\phi_n(x)$, so there are 2 cases: \\
Case I: $g(\zeta_n^p)=0$ for some prime $p$.  Then $\zeta_n$ is a root of $g(x^p)$.  Since $f(\zeta_n)=0$ and $f$ is irreducible, we get
\[ g(x^p) = f(x)h(x) \]
for some $h(x)\in\Z[x]$.  Reducing mod $p$:
\begin{align*}
g(x^p) &\equiv f(x)h(x) \bmod p \\
\implies g(x)^p &\equiv f(x)h(x) \bmod p
\end{align*}
so $\gcd(f,g)\nequiv1\bmod p$. \\
So $\phi_n(x)=f(x)g(x)$ has a multiple root mod $p$.  But this is impossible, since $\phi_n(x)\divides x^n-1$ and $x^n-1$ is separable mod $p$ (since $p\ndivides n$).  So we are in: \\
Case II: $g(\zeta^p_n)\neq0$ for all primes $p\ndivides n$.  In this case, $g(\zeta_n^a)$ for all $a$ prime to $n$.  Since $g\divides\phi_n(x)$, this means $g(x)$ is constant and $\phi_n(x)$ is irreducible. \qed

So $\zeta_n$ has minimal polynomial $\phi_n(x)$ over $\Q$.  Since $\deg(\phi_n(x))=\phi(n)$, we conclude:
\[ [\Q(\zeta_n):\Q] = \phi(n) \]
%
If $n=p$ is prime, then $\phi_p(x)=\frac{x^p-1}{x-1}=x^{p-1}+x^{p-2}+\dotsb+x+1$.
