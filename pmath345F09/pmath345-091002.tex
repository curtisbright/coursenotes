$n\geq2$, $\Z/n\Z=\set{a+n\Z}{a\in\Z}=\Z_n$ \\
$\Z/1\Z=0+1\Z$ trivial \\
In general, $R/R$ is the trivial ring. \\
$\Z/0\Z=\set{a+(0)}{a\in\Z}$
\[ a + (0) = \brace{a+0} = \brace{a} \]
\textbf{Exercise:} $\Z/0\Z\approx\Z$ by $\Z/0\Z\to\Z$, $a+(0)\mapsto a$ \\
In general, $R/(0)\approx R$ in the canonical way.  That is
\begin{align*}
\phi\colon & R/(0) \to R \\
& a + (0) \mapsto a
\end{align*}
is a bijective homomorphism. \\
\ex $\R[x]$
\begin{align*}
I &= (x^2+1)\R[x] \\
  &= \set{(x^2+1)P}{P\in\R[x]}
\end{align*}
Consider $\R[x]/I$
\[ (x+I)^2 = x^2 + I \]
since $x^2+1\in I$
\[ x^2 + I = -1 + I = -(1+I) = -1_{R/I} \]
In $\R[x]/I$, $(x+I)$ is a square root of $-1$.

\lem $R$ commutative ring, $I$ ideal of $R$.
\[ \underbrace{a + I = b + I}_\text{inside $R/I$} \iff a-b\in I . \]
\pf $a+I=b+I$, so
\begin{align*}
a\in b+I &\implies a = b + x \qquad\text{for some $x\in I$} \\
&\implies a - b = x \in I
\end{align*}
If $a-b\in I$, so $a-b=x$, for some $x\in I$.
\begin{gather*}
\implies a = b + x \in b + I \\
\implies a \in a + I \\
\implies (a+I)\cap(b+I) \neq \emptyset \\
\implies a + I = b + I .
\end{gather*} %\qed
%
\begin{align*}
\text{Also } \phi\colon & \R \to \R[x]/I \\
& r \mapsto r + I
\end{align*}
is an embedding.

\pf Clearly a homomorphism, \\
Suppose $r+I=0_{R/I}$, i.e., $r\in\ker(\phi)$ \\
$r+I=0+I$ \\
$\implies r\in I$ \\
But in $I$ the only constant polynomial is $0$.  Therefore $r=0$. %\qed

\textbf{Aside:} The above argument works for any integral domain $R$.  That is,
\[ \phi\colon R \to R[x]/(x^2+1)\R[x] \]
is an embedding and in $R[x]/I$, $(x+I)^2=-1$.

Identify $\R$ with its image in $\R[x]$.
\begin{gather*}
\text{\llap{$\C \mathrel{{\approx}\footnotemark}{}$}} \R[x]/I\footnotemark \\ % interesting bug here?
| \\
\R
\end{gather*}\addtocounter{footnote}{-1}\footnotetext{we will see this}\addtocounter{footnote}{1}\footnotetext{in here $-1$ has a square}%
%
\textbf{Notation:} In any ring $R$, by (a) we mean $aR$, the ideal generated by $a$ in $R$, $a\in R$.

\textbf{First isomorphism theorem:} $R$, $T$ commutative rings.  $\phi\colon R\to T$ homomorphism. \\
$\Im(\phi)$\marginpar{$\im\phi\coloneqq\phi(R)$} is isomorphic to $R/\Im(\ker\phi)$.

\pf
\begin{align*}
\text{Define } \psi\colon & R/\ker\phi \to \Im\phi \\
& a + \ker\phi \mapsto \phi(a)
\end{align*}
Note if $b+\ker\phi=a+\ker\phi$ then by lemma $a-b\in\ker\phi$ \\
$\phi(a-b)=0$ \\
$\implies \phi(a)-\phi(b)=0$ \\
$\implies \phi(a)=\phi(b)$ \\
So $\psi$ is well-defined. \\
Let's write $\overline a=a+\ker\phi$.
\begin{align*}
\psi(\overline a+\overline b) &= \psi(\overline{a+b}) \text{ by definition of $+$ in $R/\ker\phi$} \\
&=\phi(a+b) \text{ by definition of $\psi$} \\
&=\phi(a)+\phi(b)=\psi(\overline a)+\psi(\overline b)
\end{align*}
Similarly $\psi(\overline a\overline b)=\psi(\overline a)\psi(\overline b)$. \\
And $\phi(\overline1)=\phi(1)=1$. \\
So $\psi$ is a homomorphism. \\
\textbf{Surjective:} $x\in\Im\phi$
\begin{align*}
x &= \phi(a)\text{ for some $a\in R$} \\
&= \psi(\overline a) \in \Im\psi
\end{align*}
therefore $\psi$ is surjective \\
\textbf{Injective:} $x\in\ker(\psi)$. $\psi(x)=0$. \\
$x\in R/\ker\phi$ so $x=\overline a$ for some $a\in R$. \\
$\phi(a)=\psi(\overline a)=0$ \\
therefore $a\in\ker\phi$

\ex $\phi\colon\R[x]\to\C$ \\
the ``evaluation at $i$'' map, \\
i.e., $\phi(P)\coloneqq P(i)\in\C$ \\
\textbf{Check:} $\phi$ is a homomorphism.

$\ker\phi={?}$ \\
Suppose $P\in\ker\phi$. \\
So $P(i)=0$. \\
That is $i$ is a root of $P$. \\
In $\C[x]$, $(x-i)$ is a factor, $(x+i)$ is a factor \\
since $P$ is \emph{actually} real. \\
$\implies$ $(x+i)(x-i)=x^2+1$ is a factor \\
therefore $(x^2+1)$ is a factor of $P$ in $\R[x]$. \\
i.e., $P\in(x^2+1)=(x^2+1)\R[x]$

Conversely if $Q\in(x^2+1)$ \\
then $Q=(x^2+1)Q'$ \\
so $Q(i)=0\cdot Q'(i) = 0$. \\
$\implies$ $Q\in\ker\phi$. \\
therefore $\ker\phi=(x^2+1)$. \\
What is $\Im\phi=?$ \\
Let $a+bi\in\C$. $a,b\in\R$ \\
$a+bi=P(i)$ \qquad $P=a+bx\in\R[x]$ \\
therefore $\phi$ is surjective. \\
Hence $\C\approx\R[x]/(x^2+1)$. \\
Moreover this isomorphism is given by
\begin{align*}
\phi\colon &  \R[x]/(x^2+1) \to \C \\
& P+(x^2=1) \mapsto P(i)
\end{align*}