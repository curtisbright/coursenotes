\thm (Chinese Remainder) Let $F$ be a field, $p(x)$, $q(x)\in F[x]$ coprime polynomials.  Then the function:
\[ \phi\colon F[x] / (pq) \to F[x] /(p) \oplus F[x] / (q) \]
given by
\[ (a(x)\bmod pq) \mapsto (a(x)\bmod p,a(x)\bmod q) \]
is an isomorphism. \\
\pf (Continued) To show that $\phi$ is onto, we first note that since $F[x]$ is a PID, and since $p$, $q$ are coprime, we get $(p(x),q(x))=(1)$.  In other words, there are $a(x)$, $b(x)\in F[x]$ such that
\[ a(x) p(x) + b(x) q(x) = 1 . \]
Now let $f(x)$, $g(x)\in F[x]$ be any polynomials.  We want to find $h(x)\in F[x]$ such that
\begin{align*}
h(x) &\equiv f(x) \bmod p \\
h(x) &\equiv g(x) \bmod q
\end{align*}
Let $h(x)=f(x)b(x)q(x)+g(x)a(x)p(x)$.  Then
\begin{align*}
h(x) &\equiv f(x) \bmod p \\
\mathllap{\text{and}\qquad} h(x) &\equiv g(x) \bmod q
\end{align*}
So $\phi(h(x)\bmod pq)=(f(x)\bmod p,g(x)\bmod q)$, as desired. \qed

In light of the CRT, to understand $F[x]/(f(x))$, it suffices to understand
\[ F[x] / (p(x)^a) \]
for irreducible polynomials $p(x)$.  We will study $F[x]/(p(x))$ for irreducible $p(x)$.  Note that $F[x]/(p(x))$ is a field \emph{iff} $p(x)$ is irreducible in $F[x]$.

Linear Algebra over general fields. \\
\textbf{Non-definition:} A vector space over a field $F$ is a set $V$ of ``vectors'' that you can add, subtract, and multiply by scalars in a sensible way.

Spanning, linear independence, basis, dimension, linear transformation, kernel, range, eigenstuff\ldots they all have the same definitions and properties over a general field as they do over, say, $\R$.

Note that if $F$ is a field and $R$ is any ring with $F\subset R$, then $R$ is an $F$-vector space.

In particular, $F[x]/(p(x))$ is an $F$-vector space.
\begin{align*}
F &\hookrightarrow F[x]/(p) \\
\alpha &\mapsto (\alpha\bmod p)
\end{align*}
\thm Let $F$ be a field, $p(x)\in F[x]$ any polynomial. If $p(x)=0$, then $\dim_F F[x]/(p(x))=\infty$.  Otherwise, $\dim_F F[x]/(p(x))=\deg(p(x))$. \\
\pf If $p(x)=0$, then $F[x]/(0)=F[x]$, which contains the infinite linearly independent set $\brace{1,x,x^2,x^3,\dotsc}$.  Now assume $p(x)\neq0$.  Then by the Division Theorem, for any $f(x)\in F[x]$, we can write:
\[ f(x) = q(x)p(x) + r(x) \]
where $q(x)$, $r(x)\in F[x]$, and $\deg(r(x))<\deg(p(x))$.  Better yet, $r(x)$ is unique!

So $F[x]/(p(x))$ is in 1--1 correspondence with $\set{r(x)}{\deg(r)<\deg(p)}$.  Furthermore, this correspondence respects addition and scalar multiplication, but not multiplication (unless you reduce the result mod $p(x)$ again).

In particular, $F[x]/(p(x))$ is isomorphic as an $F$-vector space to:
\[ V = \set{r(x)}{\deg(r(x))<\deg(p(x))} \]
A basis for $V$ is
\[ \brace{1,x,x^2,\dotsc,x^{\deg p-1}} \]
so $\dim_F F[x]/(p(x))=\deg(p(x))$ as desired. \qed

\eg $\dim_\Q \Q[x]/(x^2-1)=2$%\footnote{
\[ (a+bx)(c+dx) = (ac+bd)+(ad+bc)x \]
%} \\
Basis: $\brace{1,x}$ \\
\eg $\dim_\Q \Q[x]/(x^2-2)=2$%\footnote{
\[ (a+bx)(c+dx) = (ac+2bd)+(ad+bc)x \]
%} \\
Basis: $\brace{1,x}$. \\
These two rings are \emph{not} isomorphic, but the two $\Q$-vector spaces are.
