\textbf{Fermat Primes} $F_n=2^{2^n}+1$ \\
\eg Show that $F_n=F_0F_1F_2\dotsm F_{n-1}+2$, for $n\geq1$. \\
\soln $F_0=2^{2^0}+1=2^1+1=3$ \\
$F_1=2^{2^1}+1=2^2+1=5=F_0+2$ \\
Fix $n\geq1$ and suppose that $F_n=F_0F_1F_2\dotsm F_{n-1}+2$.
\begin{align*}
F_{n+1}-2 &= 2^{2^{n+1}}-1 \\
&= (2^{2^n})^2-1 \\
&= (2^{2^n}+1)(2^2-1) \\
&= F_n(F_n-2) \\
&= F_n(F_0F_1\dotsm F_{n-1}) \\
&= F_0F_1\dotsm F_n \text{, so} \\
F_{n+1} &= F_0 F_1 \dotsm F_n + 2 .
\end{align*}
\eg Show that for $k\neq l$, we have $\gcd(F_k,F_l)=1$. \\
\pf Say $0\leq k< l$. \\
Then $F_l=F_0F_1\dotsm F_k\dotsm F_{l-1}+2$ \\
Recall that if $a=bq+r$, then $\gcd(a,b)=\gcd(b,r)$
\begin{align*}
\therefore \gcd(F_l,F_k) &= \gcd(F_{k}q+2,F_k)\text{, with }q=\frac{F_0F_1\dotsm F_{l-1}}{F_k} \\
&= \gcd(F_k,2) \\
&= 1 \text{ or } 2
\end{align*}
and $\gcd(F_k,F_l)\neq2$, since $F_k$ and $F_l$ are odd.

\textbf{Mersenne Primes} \\
\defn A \emph{Mersenne prime} is a prime of the form $2^k-1$ with $k\in\Z^+$.  We have:
\[ \begin{array}{c|cccccccccc}
k & 1 & 2 & 3 & 4 & 5 & 6 & 7 & 8 & 9 & 10 \\ \hline
2^k-1 & 1 & \ovalbox{3} & \ovalbox{7} & 15 & \ovalbox{31} & 63 & \ovalbox{127} & 255 & 511 & 1023
\end{array} \]
It appears that $2^k-1$ is prime $\iff$ $k$ is prime. \\
\eg Show that if $2^k-1$ is prime, then $k$ is prime. \\
Indeed, show that if $a^k-1$ is prime with $a\geq2$ and $k\geq1$, then $a=2$ and $k$ is prime. \\
\soln If $a\geq3$ then
\[ (a^k-1) = (a-1)(a^{k-1}+a^{k-2}+\dotsb+1) \]
and $1<a-1<a^k-1$, so $a^k-1$ is not prime. \\
$\therefore$ we need $a=2$. \\
If $k$ is not prime, say $k=rs$ with $1<r$, $s$, then:
\[ (2^k-1) = (2^{rs}-1) = (2^r-1)((2^r)^{s-1}+(2^r)^{s-2}+\dotsb+1) \]
and $1<2^r-1<2^{rs}-1$.

\defn The $n$th Mersenne number is $M_n=2^n-1$. \\
\eg Show that $M_{11}$ is composite. \\
\soln $M_{11}=2^{11}-1=2047$ \\
$\floor{\sqrt{2047}}=45$, so we test every $p\leq45$ to see if it is a factor. \\
We try $2$, $3$, $5$, $7$, $11$, $13$, $17$, $19$.  They are not factors. \\
But ($23$ works): $\therefore 2047=23\cdot89$.

\eg Show that for $k$, $l\in\Z^+$, if $\gcd(k,l)=1$, then $\gcd(M_k,M_l)=1$. \\
\soln Suppose $\gcd(M_k,M_l)\neq1$, say $\gcd(M_k,M_l)=d>1$ \\
Then $d\div M_k$, $d\div 2^k-1$, $2^k-1=0\bmod d$, $2^k=1\bmod d$ \\
Let $m$ be the smallest positive integer such that $2^m=1\bmod d$. \\
Note that $m>1$, since $2^1=2\neq1\bmod d$, since $d>1$.

Note that $m\div k$, since if we write $k=mq+r$ with $0\leq r<m$ then
\[ 1 = 2^k = 2^{mq+r} = (2^m)^q\cdot2^r = 1^q\cdot2^r = 2^r \]
so $r=0$ by the choice of $m$.

\emph{Similarly}, $m\div l$. \\
\emph{So}, $m\div\gcd(k,l)$, $\therefore\gcd(k,l)\neq1$. \qed

\textbf{Perfect Numbers} \\
\defn A \emph{perfect number} is a positive integer $n\in\Z^+$ which is the sum of its proper divisors.
\[ n = \sum_{d\div n,d\neq n}d = \sigma(n) - n \]
That is, $\sigma(n)=2n$. \\
The first few Mersenne primes are
\[ \begin{array}{c|cccc}
k & 2 & 3 & 5 & 7 \\ \hline
M_k=2^k-1 & 3 & 7 & 31 & 127
\end{array}
\]
The first few perfect numbers are
\begin{align*}
6 &= 2\cdot3 = 1+2+3 \\
28 &= 4\cdot7 = 1+2+4+7+14 \\
496 &= 16\cdot31 = 1+2+4+8+16+31+62+124+248 
\end{align*}
It appears that the perfect numbers are the number of the form $2^{p-1}M_p$, where $M_p$ is a Mersenne prime. \\
\remark It is not known whether there exist any odd perfect numbers. \\
\thm Every even perfect number is of the form $n=2^{p-1}M_p$, for some Mersenne prime $p$. \\
\pf Suppose $n=2^{p-1}M_p$, where $M_p$ is a Mersenne prime. \\
(Recall: $\sigma(\prod p_i^{k_i})=\prod\sigma(p_i^{k_i})$, $\sigma(p^k)=1+p+p^2+\dotsb+p^k=\frac{p^{k+1}-1}{p-1}$)
\begin{align*}
\text{Then } \sigma(n) &= \sigma(2^{p-1}M_p) \\
&= \sigma(2^{p-1})\sigma(M_p) \text{, since $M_p$ is an odd prime} \\
&= (1+2+2^2+\dotsb+2^{p-1})(1+M_p) \text{, since $M_p$ is prime} \\
&= (2^p-1)(1+M_p) \\
&= 2^p + 2^p M_p - 1 - M_p = 2^p M_p = 2 (2^{p-1}M_p) = 2n .
\end{align*}
Conversely, suppose $n$ is an even perfect number, say $n=2^{p-1}q$, where $p\geq2$, and $q$ is odd.
\begin{align*}
\text{Then } \sigma(n) = 2n &\implies \sigma(2^{p-1})\sigma(q) = 2^p q \\
&\implies (2^p-1)\sigma(q) = 2^p q
\end{align*}
$\therefore 2^p-1\div q$, say $q=(2^p-1)k$,
\begin{align*}
\text{so } (2^p-1)\sigma(q) &= 2^p(2^p-1)k \\
\sigma(q) &= 2^p k
\end{align*}
Note that $k\div q$ and $q\div q$ and $k+q=k+(2^p-1)k=2^pk=\sigma(q)$ \\
$\therefore k$ and $q$ are \emph{all} of the divisors of $q$ \\
$\therefore k=1$ and $q$ is prime, and we have
\[ q = (2^p-1)k = 2^p-1 = M_p . \]
