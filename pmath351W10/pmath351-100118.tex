If $\set{\T_i}{i\in I}$ is a non-empty family of topologies on $X$, then $\bigcap_{i\in I}\T_i$ is a top (on $X$) \\
\pf
\begin{enumerate}%(i)
\item $\emptyset\in\T_i$ for each $i\in I$, as each $\T_i$ is a top.  So $\emptyset\in\bigcap_{i\in I}\T_i$.  Similarly, $X\in\bigcap_{i\in I}\T_i$.
\item We shall show that if $A$ and $B$ are in $\bigcap_{i\in I}\T_i$, then $A\cap B\in\bigcap_{i\in I}\T_i$.  For each $i\in I$, $A\in\T_i$ and $B\in\T_i$ by definition of intersection.  Since $\T_n$ is a topology, $A\cap B\in\T_i$.  So $A\cap B\in\bigcap_{i\in I}\T_i$.
\item Let $A_j\in\bigcap_{i\in I}\T_i$ for each $j\in J$.  Then, for each $i\in I$, $A_j\in\T_i$ for each $j\in J$.  As $\T_i$ is a topology, $\bigcup_{j\in J}A_j\in\T_i$.  As $i\in I$ is arbitrary, $\bigcup_{j\in J}A_j\in\bigcap_{i\in I}\T_i$.  This shows that $\bigcap_{i\in I}\T_i$ is \emph{closed} under arbitrary union.
\end{enumerate}
\prop Let $X$ be a non-empty set.  Let $\S$ be any given family of subsets of $X$ (i.e., $\S\subset\P(X)$).  Then there exists a topology $\T_0$ on $X$ such that (1) $\T_0\supset\S$ (2) if $\T$ is a topology on $X$ and $\T\supset\S$, then $\T_0\subset\T$.  So, $\T_0$ is the smallest topology on $X$ which contains $\S$. %\\

\pf Consider $\G=\set{\T}{\text{$\T$ is a topology on $X$, $\T\supset\S$}}$.  Clearly, the discrete topology, $\P(X)$, contains $\S$ and so it is an element of $\G$.  Thus $\G\neq\emptyset$. \\
Now $\T_0\overset{\text{def}}{=}\bigcap\G$ is a topology on $X$ by the previous theorem.  Since each $\T\in\G$ clearly contains $\T_0$, this shows that (2) holds.

\defn We call $\T_0$ the topology \emph{generated} by $\S$.

\eg Let $X=\brace{a,b,c,d}$.  Let $\S=\brace{\brace{a},\brace{b},\brace{c,d}}$. \\
Then the topology generated by $\S$ is
\[ \T_0 = \brace{\brace{a},\brace{b},\brace{c,d},\emptyset,X,\brace{a,b},\brace{a,c,d},\brace{b,c,d}} \]
\prop Let $\S\subset\P(X)$ be given.  Let $\B$ be obtained from $\S$ by taking all possible finite intersections of members of $\S$.  (Then $\B$ is closed under finite intersection.)  Next, let $\sC$ be obtained from $\B$ by taking all possible arbitrary union of members of $\B$.  Then $\sC$ is not just closed under arbitrary union, it is still closed under finite intersection.  (Exercise.) \\
In particular, $\sC=\T_0$. \\
\remark By first taking arbitrary union of members of $\S$ then further by taking finite intersections, we don't always get $\T_0$.
