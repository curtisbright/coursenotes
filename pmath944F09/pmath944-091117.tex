TALKS Tue--Fri Dec 1--4.

Recall $q=e^{2\pi iz}$.  We define the theta function of a lattice $\Lambda$ in $\R^n$ by
\[ \theta_\Lambda(z) = \sum_{\x\in\Lambda} q^{(\x\cdot\x)/2} . \]
If $B$ has integer entries, determinant $1$ and $\x\cdot\x\equiv0\pmod2$ for all $\x\in\Lambda$ then $\theta_\Lambda$ is a modular form of weight $n/2$.

Let $\paren*{\begin{smallmatrix}a&b\\c&d\end{smallmatrix}}\in\SL(2,\Z)$ so that $a$, $b$, $c$ and $d$ are integers with $ad-bc=1$.  $\SL(2,\Z)$ is a group which acts on the upper half plane $H=\set{z\in\C}{\Im(z)>0}$ by, for each $g=\paren*{\begin{smallmatrix}a&b\\c&d\end{smallmatrix}}\in\SL(2,\Z)$ we put $gz=\frac{az+b}{cz+d}$.  Let $k$ be an integer.  We say that a meromorphic function $f\colon H\to\C$ is said to be weakly modular of weight $2k$ if
\[ f(z) = (cz+d)^{-2k}f\paren*{\frac{az+b}{cz+d}}, \qquad \text{for all $\paren*{\begin{smallmatrix}a&b\\c&d\end{smallmatrix}}\in\SL(2,\Z)$} . \]
Note that if $g=\paren*{\begin{smallmatrix}a&b\\c&d\end{smallmatrix}}=\paren*{\begin{smallmatrix}1&1\\0&1\end{smallmatrix}}$ then $gz=z+1$ and so if $f$ is weakly modular of weight $2k$, $f(z+1)=f(z)$ and so $f$ can be expressed in terms of $q=e^{2\pi iz}$.  In particular $f$ determines a function $\tilde f(q)$ where
\[ \tilde f\colon \set{q\in\C}{0<\abs q<1} \to \C . \]
$\tilde f$ is meromorphic on the punctured dish $\set{q\in\C}{0<\abs q<1}$ and if it extends to a meromorphic function on all of the disc then we say that $f$ is a modular function.  If $\tilde f$ is holomorphic on $\set{q\in\C}{0<\abs q<1}$ and extends to a holomorphic function on $\set{q\in\C}{\abs q<1}$ then we say that $f$ is a modular form.

The space of modular forms of weight $2k$ ($k\geq0$), forms a vector space $M_{2k}$ over $\C$ of dimension $d_k$ where
\[ d_k = \begin{cases}
%\brack*{\frac{k}{6}}, & k\equiv1\pmod6\text{, }k\geq0 \\
%\brack*{\frac{k}{6}}+1, & k\not\equiv1\pmod6\text{, }k\geq0
\floor*{\frac{k}{6}}, & k\equiv1\pmod6\text{, }k\geq0 \\
\floor*{\frac{k}{6}}+1, & k\not\equiv1\pmod6\text{, }k\geq0
\end{cases} \]
%Here $[x]$ denotes the greatest integer less than or equal to $x$.
Here $\floor{x}$ denotes the greatest integer less than or equal to $x$.

The lattice $\Lambda=E_8(\sqrt2)$ determines $\theta_{E_8(\sqrt2)}(z)$ which is a modular form of weight $4$.  Thus $\theta_{E_8(\sqrt2)}(z)$ lies in $M_4$ a vector space of dimension $1$ over $\C$.  Now $E_2(z)=1+240\sum_{n=1}^\infty\sigma_3(n)q^n$ is in $M_4$.  We have
\[ \theta_{E_8(\sqrt2)}(z) = \sum_{m=0}^\infty r_\Lambda(m)q^m \]
where $r_\Lambda(m)$ counts the number of vectors $\x$ in $\Lambda=E_8(\sqrt2)$ for which $\x\cdot\x=2m$.  Thus $E_2(z)=\theta_{E_8(\sqrt2)}(z)$.

Associated to each lattice $\Lambda$ in $\R^n$ is $\Aut(\Lambda)$, the group of symmetries of the lattice which fix the origin or equivalently the set of isometries (distance preserving maps) of $\R^n$ which fix the origin and take the lattice to itself.  For each lattice $\Lambda$ in $\R^n$, $\Aut(\Lambda)$ is a finite group.  Each element of the group can be represented by an orthogonal matrix.

The automorphism group of the hexagonal lattice $A_2$
%honeycomb lattice figure
\setlength{\unitlength}{0.6cm}
\[ %\begin{picture}(4.3,4.3)(-2.15,-1.9)
\begin{picture}(6.45,5.58)(-3.22,-2.79)
\put(0,0){\circle*{0.1}}
\put(2.15,0){\circle*{0.1}}
\put(-2.15,0){\circle*{0.1}}
\put(1.07,1.86){\circle*{0.1}}
\put(1.07,-1.86){\circle*{0.1}}
\put(-1.07,1.86){\circle*{0.1}}
\put(-1.07,-1.86){\circle*{0.1}}
\path(2.15,0)(1.07,1.86)(-1.07,1.86)(-2.15,0)(-1.07,-1.86)(1.07,-1.86)(2.15,0)(-2.15,0)
\path(-1.07,1.86)(1.07,-1.86)
\path(-1.07,-1.86)(1.07,1.86)
\dashline{0.15}(2.15,0)(3.22,0)
\dashline{0.15}(-2.15,0)(-3.22,0)
\dashline{0.15}(1.07,1.86)(1.61,2.79)
\dashline{0.15}(1.07,-1.86)(1.61,-2.79)
\dashline{0.15}(-1.07,1.86)(-1.61,2.79)
\dashline{0.15}(-1.07,-1.86)(-1.61,-2.79)
\end{picture} \]
is generated by a rotation of $\frac{\pi}{3}$ and a reflection around the line determined by any non-zero vector.  Thus it is isomorphic to the Dihedral group $D_6$.

The automorphism group of $E_8$ ($E_8(\sqrt2)$) is a group of order $2^{14}\cdot3^5\cdot5^2\cdot7$ and it permutes the 240 vectors of minimal length transitively.

We'll now construct an astonishing combinatorial object called the Leech lattice.  It was found by Leech in 1965 and it was described by him in a paper in the Canadian Journal of Math in 1967.  It is a lattice $L$ in $\R^{24}$ with determinant $1$, the associated inner product matrix $B$ has integer entries.  The polar lattice $L^*$ of $L$ is $L$, in other words $L$ is self-dual.  Further if $\x\in L$ and $\x\neq\0$ then
\[ \x\cdot\x \geq 4 . \]
We'll now construct the Leech lattice $L$ following Leech and Milnar.

Let $\F_2^{24}$ be the 24 dimensional vector space over the field $\F_2=\brace{0,1}$ of two elements.

\textbf{Proposition 28: }There exists a 12 dimensional subspace $S$ of $\F_2^{24}$ with the following property.  For every non-zero vector $\s=(s_1,\dotsc,s_{24})$ in $S$ the number of coordinates which are $1$ is at least 8 and is congruent to $0\bmod4$.  Further $(1,1,\dotsc,1)$ is in $S$.

To prove this we'll realize $S$ as the span of the rows of a $12\times24$ matrix over $\F_2$ which we shall construct.  Let $A$ denote a symmetric $11\times11$ matrix whose first row is
\[ 1\,1\,1\,0\,1\,1\,0\,1\,0\,0\,0 \]
and whose remaining rows are obtained by permuting the rows cyclically to the left so
\[ A = \begin{pmatrix}
1 & 1 & 1 & 0 & 1 & 1 & 0 & 1 & 0 & 0 & 0 \\
1 & 1 & 0 & 1 & 1 & 0 & 1 & 0 & 0 & 0 & 1 \\
1 & 0 & 1 & 1 & 0 & 1 & 0 & 0 & 0 & 1 & 1 \\
0 & 1 & 1 & 0 & 1 & 0 & 0 & 0 & 1 & 1 & 1 \\
1 & 1 & 0 & 1 & 0 & 0 & 0 & 1 & 1 & 1 & 0 \\
1 & 0 & 1 & 0 & 0 & 0 & 1 & 1 & 1 & 0 & 1 \\
0 & 1 & 0 & 0 & 0 & 1 & 1 & 1 & 0 & 1 & 1 \\
1 & 0 & 0 & 0 & 1 & 1 & 1 & 0 & 1 & 1 & 0 \\
0 & 0 & 0 & 1 & 1 & 1 & 0 & 1 & 1 & 0 & 1 \\
0 & 0 & 1 & 1 & 1 & 0 & 1 & 1 & 0 & 1 & 0 \\
0 & 1 & 1 & 1 & 0 & 1 & 1 & 0 & 1 & 0 & 0
\end{pmatrix} \]

One may check that each pair of rows have exactly three columns consisting of two $1$s.

Next let $B$ be the symmetric $12\times12$ matrix obtained by adjoining a first row of the form $(011\cdots1)$ to $A$ and completing the first column to be $(011\cdots1)^\text{tr}$ also.  Thus
\[ B = \begin{pmatrix}
0 & 1 & 1 & \cdots & 1 \\
1 \\
\vdots & & \raisebox{1ex}{\rule{1ex}{0ex}\rlap{$A$}} & & \\
1
\end{pmatrix} \]
Since any two rows of $A$ have exactly three columns of the form $\begin{pmatrix}1\\1\end{pmatrix}$ we see that
\[ B^2 = B B^\text{tr} = I . \footnote{over $\F_2$} \]
We now put
\[ C = \paren*{\begin{array}{@{}c|c@{}} I_{12} & B \end{array}} \qquad \text{a $12\times24$ matrix} \]
We claim that $S$ is the subspace of $\F_2^{24}$ generated by the rows of $C$.
