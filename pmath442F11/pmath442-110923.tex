Let $K/F$ be a field extension.  Then $\Aut_F(K)$ is the set of $F$-algebra isomorphisms $\phi\colon K\to K$. \\
\eg $\Aut_K(K)=\brace{1}\footnote{$\id$}$ \\
(An automorphism is an isomorphism of an object with itself.) \\
\eg $\Aut_\R(\C)=\brace{1,\sigma}$ where $\sigma$ is complex conjugation. \\
\eg $\Aut_\Q(\Q(\sqrt2))=\brace{1,\sigma}$ where $\sigma(a+b\sqrt2)=a-b\sqrt2$. \\
\eg If $\sqrt{D}\notin F$, then $\Aut_F(F(\sqrt D))=\brace{1,\sigma}$, where $\sigma(a+b\sqrt{D})=a-b\sqrt{D}$.
\begin{align*}
i^2 = -1 &\implies \sigma(i^2)=\sigma(-1) \\
&\implies \sigma(i)^2 = -1
\end{align*}
\thm Let $p(x)\in F[x]$ be any polynomial, $E/F$ an extension, $\sigma\in\Aut_F(E)$.  If $\alpha\in E$ is a root of $p(x)$, then so is $\sigma(\alpha)$. \\
\pf Let $p(x)=a_0+a_1x+\dotsb+a_nx^n$ for $a_i\in F$.  Then:
\begin{align*}
a_0 + a_1\alpha + \dotsb + a_n\alpha^n &= 0 \\
\implies \sigma(a_0+\dotsb+a_n\alpha^n) &= 0 \\
\implies \sigma(a_0) + \dotsb + \sigma(a_n)\sigma(\alpha)^n &= 0 \\
\implies a_0 + \dotsb + \sigma(\alpha)^n &= 0 \\
\implies p(\sigma(\alpha)) &= 0 \qed
\end{align*}
Since $\sigma$ is 1--1, it follows that it permutes the roots of $p(x)$. \\
\eg $\Aut_\Q(\Q(\sqrt[3]{2}))=\brace1$, because $\sigma(\sqrt[3]{2})^3=2\implies\sigma(\sqrt[3]{2})=\sqrt[3]{2}$ since $\Q(\sqrt[3]{2})\subset\R$.

\thm Let $\S\subset\Aut_F(K)$ be any subset.  Let $E=\set{\alpha\in K}{\sigma(\alpha)=\alpha\text{ for all }\sigma\in\S}.$ \\
($E$ is called the fixed field of $\S$.) \\
Then $E$ is a field. \\
\pf It suffices to show $0$, $1\in E$ (clear) and that $E$ is closed under $+$, $-$, $\cdot$, and $\div$.  Thus, pick any $a$, $b\in E$.  Then for all $\sigma\in\S$, $\sigma(a)=a$ \& $\sigma(b)=b$, so $\sigma(a+b)=\sigma(a)+\sigma(b)$, and similarly for the rest. \qed

\thm Let $T\subset K$ be any subset.  Let $H=\set{\sigma\in\Aut_F(K)}{\sigma(\alpha)=\alpha\text{ for all }\alpha\in T}$. \\
Then $H$ is a subgroup of $\Aut_F(K)$. \\
\pf It suffices to show $1\in H$ (clear) and $H$ closed under composition and inversion.  This is easy:
\[ \sigma_1\in H\co\sigma_2\in H\implies\sigma_i(\alpha)=\alpha\text{ for }i=1\co2 \]
so $\sigma_1^{-1}(\alpha)=\alpha$ and $\sigma_1(\sigma_2(\alpha))=\sigma_1(\alpha)=\alpha$ \qed

\begin{center}\begin{tabular}{c@{\,\,}c|@{\,\quad}c}
$\Aut_F(K)$ & & $K/F$ \\ \hline
$\S$ & \makebox[0.75mm][l]{$\longrightarrow$} & fixed field, $F\subset E\subset K$ \\
fixing automorphisms $H$ subgroup & \makebox[0.75mm][l]{$\longleftarrow$} & $T$
\end{tabular}\end{center}
Notice that the fixed field of $\S$ is the same as the fixed field of the subgroup generated by $\S$. \\
Notice also that if $T\subset K$ is any subset, then the automorphisms fixing $T$ are the same as the automorphisms fixing $F(T)$.

In particular, if $\alpha\in K$ is any element, then the $F$-algebra homomorphisms of $K$ fixing $\alpha$ are precisely the $F$-algebra homomorphisms fixing $F(\alpha)$.

For instance, $\sigma\in\Aut_\Q(\C)$ fixes $\sqrt2$ \emph{iff} it fixes $\Q(\sqrt2)$.

If $H_1\subset H_2$, then $\fix(H_2)\subset\fix(H_1)$.  If $E_1\subset E_2$, then $H_2\footnote{$\Aut_{E_2}(K)$}\subset H_1\footnote{$\Aut_{E_1}(K)$}$.
\begin{center}\begin{tabular}{c|c}
$\Aut_\R(\C)$ & $\C/\R$ \\ \hline
$\brace1$ & $\C/\R$ \\
$\brace{1,\sigma}$ & $\R/\R$
\end{tabular}\end{center}
For which field extensions $K/F$ is this correspondence a bijection? \\
\ans Splitting fields.  Almost.
