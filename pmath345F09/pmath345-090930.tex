\vspace{-\baselineskip}
\begin{align*}
e &= (f+f) \\
(1+e)^{-1} &\overset{\times}{=} (1-f)\footnotemark \\
&= (1-ef)
\end{align*}\footnotetext{$f$ is unique $\iff$ $2=1+1$ is not a zero divisor}%
%
\ex Any $R$, $(0)$ trivial ideal $=\brace0$

\ex $\phi\colon R\to S$ homomorphism of rings \\
$\ker\phi$ is an ideal of $R$.

\pf
\[ \begin{gathered}
\phi(a) = 0 \\
\phi(b) = 0
\end{gathered} \implies \phi(a+b)=\phi(a)+\phi(b)=0 \]
$\ker\phi\neq0$ since $0\in\ker\phi$ \\
$a\in\ker\phi$, $r\in R$, $\phi(ra)=\phi(r)\phi(a)=\phi(r)0=0$ \\
Similarly $\phi(ar)=0$ $\longrightarrow$ $ar,ra\in\ker\phi$

\ex What are the ideals of $\Z$? \\
Suppose $I\neq(0)$ ideal in $\Z$. \\
$\implies$ $I$ has positive elements (since $a\in I\implies -a\in I$) \\
Let $c$ be the \emph{least} positive integer \emph{in $I$}. \\
Let $J=c\Z\coloneqq\set{ca}{a\in\Z}=\brace{\text{integers divisible by $c$}}$

\textbf{Check:} $J$ is an ideal ``ideal generated by $c$'' \\
$J\subseteq I$ since $c\in I$, all $ca\in I$

\textbf{Claim:} $J=I$. \\
\pf Suppose not. \\
There is $a\in I\setminus J$. \\
If $-a\in J$ then $-(-a)=a\in J$. \\
But $a\notin J$, so $-a\notin J$. \\
But $-a\in I$.  So $-a\in I\setminus J$. \\
$I\setminus J$ has a positive integer. \\
Let $b$ be the \emph{least} positive integer in $I\setminus J$. \\
$\implies b=qc+r$ where $q\in\Z$, $0<r<c$. \\
$r=b-qc=b+(-q)c\in I$ since $b,c\in I$, therefore $r\in I$. \\
Note $b\geq c$ by choice of $c$. \\
$\implies$ $r<c\leq b$, therefore $r<b$ \\
And $0<r<c$, $c\nmid r$ $\implies$ $r\notin J$. \\
Contradiction to minimal choice of $b$.

Every ideal in $\Z$ is of the form $c\Z$ for some $c\geq0$.

\defin $R$ commutative ring.  A \emph{principal ideal} is one of the form
\[ cR \coloneqq \set{ca}{a\in R} \]
where $c\in R$.

(Exercise: $cR$ is the smallest ideal containing $c$.)

$R$ is a \emph{principal ideal domain} (pid) if it is an integral domain and \emph{every} ideal of $R$ is principal. \\
So $\Z$ is a pid.

$R$ commutative ring.  $I$ an ideal of $R$.  $a\in R$, $\overline{a}\coloneqq a+I\coloneqq\set{a+b}{b\in I}\subseteq R$.

\textbf{residue $a\bmod I$} \\
$R/I \coloneqq \set{\overline a}{a\in R}$.

\textbf{Quotient of $R$ modulo $I$} \\
Elements of $R/I$ are called \emph{cosets} of $I$.

\lem If $a,b\in R$, either $\overline a=\overline b$ or ${\overline a}\cap{\overline b}=\emptyset$.

\pf Suppose $z\in\overline a\cap\overline b$.
\[ \begin{gathered}
z = a + x \\
z = b + y
\end{gathered} \qquad\text{for some $x,y\in I$} \]
$\Longrightarrow$ $a=b+(y-x)$ \\
Hence for any $u\in I$,
\begin{align*}
a + u &= b + \underbrace{(y-x) + u}_{\text{in $I$}} \\
&\in b + I = \overline b
\end{align*}
therefore $\overline a\subseteq\overline b$.  Similarly $\overline b\subseteq\overline a$.

\textbf{Note:} If $a\in R$ then $a\in\overline a = a + I$ \\
Hence $R$ is partitioned into disjoint cosets of $I$.\marginpar{figure: $I$ subset of $R$}%
\\ (Possibly \emph{infinite} partitioning of $R$).

\prop $R/I$ is a commutative ring with:
\begin{align*}
0 &= 0 + I \\
1 &= 1 + I \\
(a+I) + (b+I) &= (a+b) + I \\
(a+I)(b+I) &= (ab)+I
\end{align*}
\pf Need to prove that $+$ and $\times$ on $R/I$ are well-defined operations.

\textbf{Note:} A coset $a+I$ is not uniquely represented by this notation.  In fact if $b\in a+I$ then $a+I=b+I$. (by the lemma) \\
(conversely $a+I=b+I\implies b\in a+I$).

Every element of a coset represents that coset. \\
$+$ should depend only on the cosets \emph{not} on the representatives. \\
\textbf{need:} If $a+I=a'+I$ \\
$b+I=b'+I$ \\
then $(a+b)+I=(a'+b')+I$. \\
\pf
\begin{align*}
a' + I = a + I &\implies a' \in a + I \\
&\implies a' = a + x \text{ for some $x\in I$} \\
b' + I = b + I &\implies b'\in b + I \\
&\implies b' = b + y \text{ for some $y\in I$} \\
\implies (a'+b') &= (a+b) + \underbrace{(x+y)}_\text{in $I$} \\
&\in (a+b) + I \\
\text{therefore } (a'+b')+I &= (a+b) + I
\end{align*}
Similarly check $\times$ is well-defined. \\
\textbf{Check: $R/I$ is a commutative ring.}

\ex Consider $\Z$ and the ideal $n\Z=\set{na}{a\in\Z}$, $n\geq2$ \\
\textbf{Check:} $\Z/n\Z=\Z_n$ \\
$a\in\Z$. $\res(a)=a+n\Z$ \\
$\Z_n$ is the quotient of $\Z \bmod n\Z$ \\
\textbf{missing:} $n=0$, $n=1$, $0\Z=(0)$, $\Z/(0)=\set{a+(0)=\brace a}{a\in\Z}$ \\
$\Z/1\Z$ trivial ring