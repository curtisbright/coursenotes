%((10) + (11))

We'll prove that $\mu_2=\sqrt[4]{4/3}$.  We first note that this is a lattice $\Lambda$ in $\R^2$ with $d(\Lambda)=1$ and $\mu(\Lambda)=\sqrt[4]{4/3}$.  We take the basis vectors for $\Lambda$ to be $\paren[\big]{\sqrt[4]{4/3},0}$, $\paren[\big]{\frac{1}{2}\sqrt[4]{4/3},\sqrt[4]{3/4}}$.  Observe that $d(\Lambda)=1$ and that both generating vectors have length $\sqrt[4]{4/3}$ and that this is the minimal distance between two distinct vectors in $\Lambda$:%\marginpar{figure: honeycomb lattice}
%\[ \begin{picture}(4.3,4.3)(-2.15,-1.9)
%\put(0,0){\circle*{0.1}}
%\put(0,0){\line(3,-5){1.07}}
%\put(0,0){\line(3,5){1.07}}
%\put(2.15,0){\circle*{0.1}}
%\put(2.15,0){ $(\sqrt[4]{4/3},0)$}
%\put(-2.15,0){\circle*{0.1}}
%\put(-2.15,0){\line(3,5){1.07}}
%\put(-2.15,0){\line(3,-5){1.07}}
%\put(-2.15,0){\line(1,0){4.3}}
%\put(1.07,1.86){\circle*{0.1}}
%\put(1.07,1.86){\line(3,-5){1.07}}
%\put(1.07,1.86){ $(\frac{1}{2}\sqrt[4]{4/3},\sqrt[4]{3/4})$}
%\put(1.07,-1.86){\circle*{0.1}}
%\put(1.07,-1.86){\line(3,5){1.07}}
%\put(-1.07,1.86){\circle*{0.1}}
%\put(-1.07,1.86){\line(3,-5){1.07}}
%\put(-1.07,1.86){\line(1,0){2.15}}
%\put(-1.07,-1.86){\circle*{0.1}}
%\put(-1.07,-1.86){\line(3,5){1.07}}
%\put(-1.07,-1.86){\line(1,0){2.15}}
%%\put(-2.15,-2.15){\framebox(4.3,4.3){}}
%\end{picture} \]
\[ \begin{picture}(4.3,4.3)(-2.15,-1.9)
\put(0,0){\circle*{0.1}}
\put(2.15,0){\circle*{0.1}}
\put(2.15,0){ $\paren[\big]{\sqrt[4]{4/3},0}$}
\put(-2.15,0){\circle*{0.1}}
\put(1.07,1.86){\circle*{0.1}}
\put(1.07,1.86){ $\paren[\big]{\frac{1}{2}\sqrt[4]{4/3},\sqrt[4]{3/4}}$}
\put(1.07,-1.86){\circle*{0.1}}
\put(-1.07,1.86){\circle*{0.1}}
\put(-1.07,-1.86){\circle*{0.1}}
%\put(-2.15,0){\line(1,0){4.3}}
%\put(-1.07,-1.86){\line(1,0){2.15}}
%\put(-1.07,1.86){\line(1,0){2.15}}
%\path(-2.15,0)(-1.07,1.86)(0,0)(1.07,1.86)(2.15,0)(1.07,-1.86)(0,0)(-1.07,-1.86)(-2.15,0)
\path(2.15,0)(1.07,1.86)(-1.07,1.86)(-2.15,0)(-1.07,-1.86)(1.07,-1.86)(2.15,0)(-2.15,0)
\path(-1.07,1.86)(1.07,-1.86)
\path(-1.07,-1.86)(1.07,1.86)
\end{picture} \]

This is the maximum for suppose that $\Lambda'$ is a lattice in $\R^2$ with $d(\Lambda')=1$ for which $\mu(\Lambda')>\sqrt[4]{4/3}$.  Then, without loss of generality, we may suppose that a basis for $\Lambda'$ is of the form $(a,0)$, $(b,1/a)$ with $a>0$.  Further, by adding an appropriate multiple of $(a,0)$ to $(b,1/a)$ we may suppose that $\abs b\leq\frac{a}{2}$.

Furthermore we may suppose that $a=\mu(\Lambda')$.  If $a>\sqrt[4]{4/3}$ then $3a^4>4$ so $\frac{3}{4}a^2>\frac{1}{a^2}$.  But then $(b,\frac{1}{a})$ is closer to the origin than $(a,0)$ since $b^2+\frac{1}{a^2}<\frac{a^2}{4}+\frac{3}{4}a^2=a^2$, and this is a contradiction.

The first few extremal lattices can be represented by graphs.  The graphs are Dynkin diagrams which arise in the study of Lie groups.  A graph will consist of $n$ points which correspond to generators of the lattice.  Each of the generators will be of the same length.  If two generators are not connected by an edge they are orthogonal.  If they are connected by an edge then the angle between them is $60^\circ$ or $\frac{\pi}{3}$.  Finally we normalize the length of the generators so that the determinant of the lattice is $1$.

Here are the graphs associated with extremal lattices for $n=2\c \dotsc\c 8$.
%figures
\begin{gather*}
\begin{picture}(1,2)(0,-1)
\put(0,0){\circle*{0.2}}
\put(1,0){\circle*{0.2}}
\put(0,0){\line(1,0){1}}
\put(0.5,-0.5){\makebox(0,0){$A_2$}}
\end{picture}
\qquad
\begin{picture}(2,2)(0,-1)
\put(0,0){\circle*{0.2}}
\put(1,0){\circle*{0.2}}
\put(2,0){\circle*{0.2}}
\put(0,0){\line(1,0){2}}
\put(1,-0.5){\makebox(0,0){$A_3$}}
\end{picture}
\qquad
\begin{picture}(1.707,2)(0,-1)
\put(0,0){\circle*{0.2}}
\put(1,0){\circle*{0.2}}
\put(1.707,0.707){\circle*{0.2}}
\put(1.707,-0.707){\circle*{0.2}}
\put(0,0){\line(1,0){1}}
\put(1,0){\line(1,1){0.707}}
\put(1,0){\line(1,-1){0.707}}
\put(0.854,-0.5){\makebox(0,0){$D_4$}}
\end{picture}
\qquad
\begin{picture}(2.6,2)(-0.4,-1)
\put(0,0){\circle*{0.2}}
\put(1,0){\circle*{0.2}}
\put(2,0){\circle*{0.2}}
\put(1,1){\circle*{0.2}}
\put(0,0){\line(1,0){2}}
\put(1,0){\line(0,1){1}}
\put(-0.4,0){\makebox(0,0){$(=$}}
\put(2.2,0){\makebox(0,0){$)$}}
%\put(-0.4,-1){\framebox(2.6,2){}}
\end{picture}
\qquad
\begin{picture}(2.707,2)(0,-1)
\put(0,0){\circle*{0.2}}
\put(1,0){\circle*{0.2}}
\put(2,0){\circle*{0.2}}
\put(2.707,0.707){\circle*{0.2}}
\put(2.707,-0.707){\circle*{0.2}}
\put(0,0){\line(1,0){2}}
\put(2,0){\line(1,1){0.707}}
\put(2,0){\line(1,-1){0.707}}
\put(1.353,-0.5){\makebox(0,0){$D_5$}}
\end{picture}
\qquad
\begin{picture}(3.6,2)(-0.4,-1)
\put(0,0){\circle*{0.2}}
\put(1,0){\circle*{0.2}}
\put(2,0){\circle*{0.2}}
\put(3,0){\circle*{0.2}}
\put(2,1){\circle*{0.2}}
\put(0,0){\line(1,0){3}}
\put(2,0){\line(0,1){1}}
\put(-0.4,0){\makebox(0,0){$(=$}}
\put(3.2,0){\makebox(0,0){$)$}}
\end{picture}
\\
\begin{picture}(4,2)(0,-1)
\multiput(0,0)(1,0){5}{\circle*{0.2}}
\put(2,1){\circle*{0.2}}
\put(0,0){\line(1,0){4}}
\put(2,0){\line(0,1){1}}
\put(2,-0.5){\makebox(0,0){$E_6$}}
\end{picture}
\qquad
\begin{picture}(5,2)(0,-1)
\multiput(0,0)(1,0){6}{\circle*{0.2}}
\put(2,1){\circle*{0.2}}
\put(0,0){\line(1,0){5}}
\put(2,0){\line(0,1){1}}
\put(2.5,-0.5){\makebox(0,0){$E_7$}}
\end{picture}
\qquad
\begin{picture}(6,2)(0,-1)
\multiput(0,0)(1,0){7}{\circle*{0.2}}
\put(2,1){\circle*{0.2}}
\put(0,0){\line(1,0){6}}
\put(2,0){\line(0,1){1}}
\put(3,-0.5){\makebox(0,0){$E_8$}}
\end{picture}
\end{gather*}

These lattices give the values of $\mu$ which I indicated were the extremal values.  The difficult task is to prove they are extremal.

We'll look more closely at the lattices associated with these diagrams.  Let $\b_1\c \dotsc\c \b_n$ be basis vectors in such a lattice.  We'll assume initially that each vector is of length $\sqrt2$.  Notice that the inner product $\b_i\cdot\b_j=\abs{\b_i}\abs{\b_j}\cos{\theta_{ij}}$ where $\theta_{ij}$ is the angle between the vectors $\b_i$ and $\b_j$.  Thus if the angle is $60^\circ$ then $\b_i\cdot\b_j=2\cos{\frac{\pi}{3}}=1$.

Notice that if
\[ B = (\b_i\cdot\b_j)_{\substack{i=1,\dotsc,n\\j=1,\dotsc,n}} \qquad\text{then the}\qquad \det(B) = d(\Lambda)^2 . \]
To see this let $e_1\c \dotsc\c e_n$ be the standard basis in $\R^n$ and put $\b_i=\sum_{j=1}^n B_{ij}e_j$ with $B_{ij}\in\R$.  Then
\[ B = ( ( B_{ij} )^\text{tr} ( B_{ij} ) ) \]
and so
\[ \det(B) = (\det(B_{ij}))^2 = d(\Lambda)^2 . \]

Next we observe that each non-zero vector in $\Lambda$ has length at least $\sqrt2$.  To see this suppose that $\b=k_1\b_1+\dotsb+k_n\b_n$ is in $\Lambda$ with $k_1\c \dotsc\c k_n$ integers, not all zero.  Then
\begin{align*}
\b\cdot\b &= (k_1\b_1+\dotsb+k_n\b_n)\cdot(k_1\b_1+\dotsb+k_n\b_n) \\
&= \sum_{i=1}^n \sum_{j=1}^n k_i k_j (\b_i\cdot\b_j) \\
&= 2(k_1^2+\dotsb+k_n^2) + 2\smashoperator{\sum_{\substack{i<j\\\text{$i$ and $j$ connected}\\\text{by an edge}}}}k_ik_j .
\end{align*}
This quantity is an even integer and so the length of $\b$ is at least $\sqrt2$.

Therefore to determine $\mu(\Lambda)$ in each example it suffices to compute $\det B$ and then normalize the length of the vectors so that $d(\Lambda)=1$.
\[ \begin{picture}(1,0)(0,-0.25)\put(0,0){\circle*{0.2}}\put(1,0){\circle*{0.2}}\put(0,0){\line(1,0){1}}\put(0.5,-0.5){\makebox(0,0){$A_2$}}\end{picture} \qquad B = \begin{pmatrix} 2 & 1 \\ 1 & 2 \end{pmatrix} \qquad\text{and}\qquad \det B = 3 . \]%\marginpar{figure of $A_2$}%
Thus it suffices to divide our basis vectors by $\sqrt[4]{3}$ and then $\mu(\Lambda_2)=\frac{\sqrt2}{\sqrt[4]3}=\sqrt[4]{4/3}$. $\checkmark$

\[ \begin{picture}(2,0)(0,-0.25)\put(0,0){\circle*{0.2}}\put(1,0){\circle*{0.2}}\put(2,0){\circle*{0.2}}\put(0,0){\line(1,0){2}}\put(1,-0.5){\makebox(0,0){$A_3$}}\end{picture} \qquad B(A_3(\sqrt2)) = \begin{pmatrix} 2 & 1 & 0 \\ 1 & 2 & 1 \\ 0 & 1 & 2 \end{pmatrix} \qquad \text{and so $\det B=6-2=4$} . \]%\marginpar{figure of $A_3$}%
We must then divide $\b_1$, $\b_2$ and $\b_3$ by $\sqrt[6]4$ and so the minimal length of a vector in $\Lambda_3$ is $\frac{\sqrt2}{\sqrt[6]4}=2^{1/6}$.

\[ \begin{picture}(1.707,0)(0,-0.25)\put(0,0){\circle*{0.2}}\put(1,0){\circle*{0.2}}\put(1.707,0.707){\circle*{0.2}}\put(1.707,-0.707){\circle*{0.2}}\put(0,0){\line(1,0){1}}\put(1,0){\line(1,1){0.707}}\put(1,0){\line(1,-1){0.707}}\put(0.854,-0.5){\makebox(0,0){$D_4$}}\end{picture} \qquad B(D_4(\sqrt2)) = \begin{pmatrix} 2 & 1 & 0 & 0 \\ 1 & 2 & 1 & 1 \\ 0 & 1 & 2 & 0 \\ 0 & 1 & 0 & 2 \end{pmatrix} \qquad \text{and $\det B=4$} . \]%\marginpar{figure of $D_4$}%
Thus we must divide each vector $\b_i$ by $4^{1/8}=2^{1/4}$ and so $\mu(\Lambda_4)=\frac{\sqrt2}{2^{1/4}}=2^{1/4}$.

Let us look more closely at $D_4(\sqrt2)$.  We claim that the lattice is the same as the lattice of vectors in $\R^4$ of the form $(u_1,u_2,u_3,u_4)$ with the $u_i$s integers and $u_1+u_2+u_3+u_4\equiv0\pmod2$.  What are the shortest vectors in the above lattice?  They are
\[ (\pm1,\pm1,0,0)\c  (\pm1,0,\pm1,0)\c  (\pm1,0,0,\pm1)\c  (0,\pm1,0,\pm1)\c  (0,\pm1,\pm1,0)\c  (0,0,\pm1,\pm1) . \]
One can check that the lattice is generated by
\[ (1,0,0,1)\c  (1,0,1,0)\c  (1,0,0,-1) \text{ and } (0,1,1,0) . \]
Notice that
\[ \begin{pmatrix} 1 & 0 & 0 & 1 \\ 1 & 0 & 1 & 0 \\ 1 & 0 & 0 & -1 \\ 0 & 1 & 1 & 0 \end{pmatrix}
\begin{pmatrix} 1 & 1 & 1 & 0 \\ 0 & 0 & 0 & 1 \\ 0 & 1 & 0 & 1 \\ 1 & 0 & -1 & 0 \end{pmatrix}
=
\begin{pmatrix} 2 & 1 & 0 & 0 \\ 1 & 2 & 1 & 1 \\ 0 & 1 & 2 & 0 \\ 0 & 1 & 0 & 2 \end{pmatrix} . \]
%\vspace{-\baselineskip}