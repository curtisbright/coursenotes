\textbf{Lattices, $\Lambda$ in $\R^n$} \\
Let $\alpha_1$, $\dotsc$, $\alpha_n$ be linearly independent vectors over $\R$ in $\R^n$.  The set of points
\[ \Lambda = \set{m_1\alpha_1+\dotsb+m_n\alpha_n}{m_i\in\Z\co i=1,\dotsc,n} , \]
is known as a lattice.  The lattice is said to be generated by $\alpha_1$, $\dotsc$, $\alpha_n$.  Notice that if $(v_{ij})$ is a matrix with integer entries and $\det(v_{ij})=\pm1$ and we put
\[ \alpha'_i = \sum_{j=1}^n v_{ij}\alpha_j \]
then $\alpha'_1$, $\dotsc$, $\alpha'_n$ is also a basis for $\Lambda$.

Put $d(\Lambda)=\abs{\det(\alpha_1,\dotsc,\alpha_n)}$.  Then $d(\Lambda)$ does not depend on the choice of generators $\alpha_1$, $\dotsc$, $\alpha_n$ for $\Lambda$ since
\[ \det(\alpha_1,\dotsc,\alpha_n) = \pm \det(\alpha_1',\dotsc,\alpha_n') \]
whenever $\alpha_1'$, $\dotsc$, $\alpha_n'$ also generate $\Lambda$.

For generators $\alpha_1$, $\dotsc$, $\alpha_n$ of $\Lambda$ we can define an associated fundamental parallelogram $P$ in $\R^n$ given by
\[ P = \set{\theta_1\alpha_1+\dotsb+\theta_n\alpha_n}{0\leq\theta_i<1\text{ for }i=1,\dotsc,n} . \]
Notice that every element $\beta$ in $\R^n$ has a unique representation in the form
\[ \beta = \lambda + \gamma , \]
with $\lambda\in\Lambda$ and $\gamma\in P$.

Note also that $\mu(P)$ the Lebesgue measure or volume of $P$ is just
\[ \mu(P) = d(\Lambda) . \]
\remark Since $\alpha_1$, $\dotsc$, $\alpha_n$ are linearly independent over $\R$, $d(\Lambda)>0$. \\
\eg Let $\Lambda$ be the lattice in $\R^n$ generated by $e_1$, $\dotsc$, $e_n$ where
\[ e_j = (0,\dotsc,0,\overset{\mathclap{\text{$j$th position}}}{1},0,\dotsc,0) \]
\[ \Lambda_0 = \set{(m_1,\dotsc,m_n)}{m_i\in\Z\text{ for }i=1,\dotsc,n} . \]
$d(\Lambda_0)=1$

\textbf{Theorem 57:} (Blichfeldt's Theorem) Let $m$, $n\in\Z^+$.  Let $\Lambda$ be a lattice in $\R^n$.  Let $S$ be a set in $\R^n$ with Lebesgue measure $\mu(S)$.  Suppose that either $\mu(S)>md(\Lambda)$ or $S$ is compact and
\[ \mu(S) \geq md(\Lambda) \]
then there exist distinct points $x_1$, $\dotsc$, $x_{m+1}$ in $S$ with with $x_i-x_j\in\Lambda$ for $1\leq i,j\leq m$. \\
\pf Let $\alpha_1$, $\dotsc$, $\alpha_n$ generate $\Lambda$ and let $P$ be the fundamental parallelogram associated with $\alpha_1$, $\dotsc$, $\alpha_n$.

For each $\lambda\in\Lambda$ we define $R(\lambda)$ to be the set of points $v\in P$ such that
\[ \lambda + v \in S . \]
We then have
\[ \sum_{\lambda\in\Lambda}\mu(R(\lambda)) = \mu(S) > md(\Lambda) = m\mu(P) . \]
Therefore there is a point $v_0\in S$ which is associated with $m+1$ distinct lattice points $\lambda_1$, $\dotsc$, $\lambda_{m+1}$.  We now take $x_i=v_0+\lambda_i$ for $i=1$, $\dotsc$, $m+1$.  But then
\[ x_i - x_j = \lambda_i - \lambda_j \in \Lambda \]
as required.

Suppose now that $S$ is compact and
\[ \mu(S) = md(\Lambda) . \]
Let $\epsilon_1$, $\epsilon_2$, $\dots$ be a sequence of \emph{positive} real numbers with $\lim_{r\to\infty}\epsilon_r=0$.  Then
\[ \mu((1+\epsilon_r)S) > \mu(S) = md(\Lambda) . \]
Thus there exist points $x_{1,r}$, $\dotsc$, $x_{m+1,r}$ in $(1+\epsilon_r)S$ for which
\[ u_r(i,j) = x_{i,r} - x_{j,r} \in \Lambda \qquad \text{for $1\leq i,j\leq m+1$.} \]
%for $1\leq i,j\leq m+1$.
Since $S$ is compact we can extract a subsequence and so suppose that $\lim_{r\to\infty}x_{i,r}=x'_i$ for $i=1$, $\dotsc$, $m+1$ with $x'_i\in S$.  Notice that since $\Lambda$ is discrete the $u_r(i,j)$'s are all the same for $r$ sufficiently large.  Therefore $x_1'$, $\dotsc$, $x_{m+1}'$ are in $S$ and
\[ x'_i - x'_j \in \Lambda \qquad \text{for $1\leq i,j\leq m+1$.} \]
%for $1\leq i,j\leq m+1$.
%But then
%\[ x_i - x_j = \lambda_i - \lambda_j \in \Lambda \]
%as required.  Suppose now that $S$ is compact and
%\[ \mu(S) = md(\Lambda) . \]