\defn Let $L/F$ be an extension, $\alpha\in L$ any element.  Then $\alpha$ is solvable in radicals over $F$ \emph{iff} $\alpha\in K$ for some field $K$ such that
\[ F = K_0 = K_1 \subset K_2 \subset \dotsb \subset K_n = K \]
where $K_i=K_{i-1}(\sqrt[r_i]{a_i})$ for some $a_i\in K_{i-1}$, and $r_i\in\Z_{>0}$, $\Char F\ndivides r_i$.

We say $p(x)\in F[x]$ non-constant is solvable in radicals \emph{iff} all its roots are.  We call an extension like $K/F$ a solvable extension.

\thm Let $\alpha\in K$ be solvable in radicals over $F$.  Then $\alpha$ is contained in a Galois solvable extension. \\
\pf First, adjoin all the $r_i$th roots of unity to $f$;
\[
\xymatrix{ & \Q(\sqrt[4]{2},i)\ar@{-}[d]\ar@{-}[ld] \\
\Q(\sqrt[4]{2})\ar@{-}[d] & \Q(i,\sqrt2)\ar@{-}[d]\ar@{-}[ld] \\
\Q(\sqrt2)\ar@{-}[d] & \Q(i)\ar@{-}[ld] \\
\Q & } \]
this is an extension of solvable form.  Next, notice that to compute the Galois closure of $K$ over $F$, one need only adjoin elements of the form $\sqrt[r_i]{b_i}$ for some elements $b_i\in K_{i-1}$, although there may be several of them for each $i$. \qed

\defn A group $G$ is solvable \emph{iff} there is a set of subgroups
\[ \brace1 = G_0 \subset G_1 \subset \dotsb \subset G_n = G \]
such that $G_{i-1}$ is a normal subgroup of $G_i$, with $G_i/G_{i-1}$ an abelian group.

Say $G$ is a group, $N\subset G$ a normal subgroup.  Then $G/N$ is abelian \emph{iff} for all $g$, $h\in G$, we have $ghg^{-1}h^{-1}\in N$.

\defn The commutator of $g$ \& $h$ is $[g,h]=ghg^{-1}h^{-1}$.  The commutator subgroup of $G$ is the subgroup of $G$ generated by the commutators of $G$.  It's denoted $[G,G]$. \\
Notice that $G/N$ is abelian \emph{iff} $[G,G]\subset N$.  Also notice that $[G,G]$ is a normal subgroup of $G$, because for any homomorphism $f$ (like, say, conjugation by $\sigma$), $f(ghg^{-1}h^{-1})=f(g)f(h)f(g)^{-1}f(h)^{-1}=[f(g),f(h)]$.

We can construct the commutator series of $G$: \\
$G^{(0)}=G$ \\
$G^{(i)}=[G^{(i-1)},G^{(i-1)}]$ \\
So $G^{(0)}\supset G^{(1)}\supset G^{(2)}\supset\dotsb$ and $G^{(i)}/G^{(i-1)}$ is abelian!  If $G^{(n)}=\brace1$ for some $n$, then $G$ is solvable.  Conversely, if $G$ is finite, then if $G^{(n)}\neq\brace1$ for all $n$, then $G$ is not solvable.

\thm Let $G$ be a finite solvable group.  Then any subgroup or quotient group of $G$ is also solvable. \\\newcommand{\embeds}{\hookrightarrow}
\pf Say $H$ is a subgroup of $G$, and say $G_0=\brace1\subset G_1\subset\dotsb\subset G_n=G$ satisfy $G_i/G_{i-1}$ abelian.  Let $H_i=H\cap G_i$.  Then $H_i$ is a normal subgroup of $H_{i+1}$ and $H_{i+1}/H_i\embeds G_{i+1}/G_i$, so $H_{i+1}/H_i$ is abelian.  Since $H_0\subset G_0=\brace1$, we conclude that $H$ is solvable.

Similarly, if $N$ is a normal subgroup of $G$ \& $q\colon G\to G/N$ is the ``reduce mod $N$'' homomorphism, then the chain
\[ q(G_0) \subset q(G_1) \subset \dotsb \subset q(G_n) \]
shows that $G/N$ is solvable. \qed
