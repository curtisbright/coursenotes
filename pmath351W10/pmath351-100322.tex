\textbf{The Stone Weierstrass Theorem} (generalisation of Weierstrass approximation theorem)

Let $A$ be a compact metric space, $\B\subset C(A,\R)$ \\
Assuming that $\B$ satisfies:
\begin{enumerate}
\item[i)] $\B$ is an algebra, i.e., $f$, $g\in\B\implies f+g\in\B$, $fg\footnote{pointwise}\in\B$ \\
\phantom{$\B$ is an algebra, i.e., $f$, }$\phantom{g\in\B}\implies \lambda f\in\B\footnote{with $f+g\in\B$, vector space + linear algebra $fg\in\B$}$, $\lambda\in\R$, multiplicative
\item[ii)] constant function $1\in\B$
\item[iii)] $\B$ separates points of $A$
\end{enumerate}
then the closure of $\B$, denoted $\overline\B$, equals $C(A,\R)$

\eg $A=[a,b]$, $\B=\set{p(x)}{\text{$p$ is a polynomial on $[a,b]$}}$ \\
i, ii, iii) obvious, (iii) take the identity. \\
Every continuous function in $[a,b]$ can be approximated by a polynomial \\
\pf By the Weierstrass approximation theorem, for every $n$, exists $p_n$ such that
\[ \abs{\abs{t}-p_n(t)} < 1/n \text{ for } -n \leq t \leq n \]
Thus $\abs{\abs{f(x)}-p_n(f(x))}<1/n$ for $-n\leq f(x)\leq n$ ($n$ be large enough since $A$ is compact).

This shows that $\overline\B$ is closed under taking absolute value, i.e., $f\in\overline\B$ implies $\abs{f}\in\overline\B$. \\
First $\B$ is an algebra, is $\overline\B$ also an algebra?  Yes, since
\[ \left. \begin{aligned}
f\in\overline\B &\implies \exists\text{ an approx} \implies \abs{f-f_n}<\epsilon \\
g\in\overline\B &\implies \exists\text{ an approx} \implies \abs{g-g_n}<\epsilon
\end{aligned} \right\} f+g\in\overline\B \]
%xx
Check $+$ is a continuous function on $C(A,\R)\times C(A,\R)$ to $C(A,\R)$

Similarly, $x$ is also continuous, $f\in\overline\B$, $g\in\overline\B\implies fg\in\overline\B$ \\
$\leadsto\overline\B$ is an algebra %\\

If $f\in\overline\B$, so is $p_n(f)$ (because $\overline\B$ is an algebra) \\
%
Also, $p_n(f)(x)=p_n(f(x))$ and $\abs[\big]{\abs{f(x)}-\underbrace{p_n(f(x))}_{\in\overline\B}}<1/n$ means that $\abs{f(x)}$ can be approximated by an element of $\overline\B$, then $\abs{f(x)}\in\overline\B$ since $\overline\B$ is closed and $\abs{f(x)}$ is a limit point of $\overline\B$.

Aside: $A$ is compact, $f$ is bounded on $A$, there exists large enough $n$ such that $-n\leq f(x)\leq n$
\begin{gather*}
\begin{aligned}
\text{Define } f\vee g &= \max(f,g) \text{ pointwise} \\
f\wedge g &= \min(f,g) \text{ pointwise}
\end{aligned}\\
\begin{aligned}
\text{and observe that } f\vee g &= \frac{f+g}{2} + \frac{\abs{f-g}}{2} \\
f\wedge g &= \frac{f+g}{2} - \frac{\abs{f-g}}{2}
\end{aligned}\end{gather*}\marginpar{figure: distance between $\frac{a+b}{2}$ and $b$ on real line}%
We see that $\overline\B$ is closed under maximum and minimum.

Let $h\in C(A,\R)$ and $x_1\neq x_2\in A$, then by (iii), there exists $g\in\B$ such that $g(x_1)\neq g(x_2)$.  By choosing $\alpha$, $\beta\in\R$ correctly, we can have
\begin{align*}
\alpha g + \beta \text{ achieving } (\alpha g+\beta)(x_1) &= h(x_1) \\
                                    (\alpha g+\beta)(x_2) &= h(x_2)
\end{align*}
Call such $\alpha g+\beta$ by the name: $f_{x_1x_2}$ --- That is $f_{x_1x_2}\in\B$ and
\begin{align*}
f_{x_1x_2} &= h(x_1) \\
f_{x_1x_2} &= h(x_2)
\end{align*}
--- textbook 5.8.2
\begin{align*}
f_{yx}(y) = h(y) &\implies f_{yx}(y) > h(y) - \epsilon \\
\text{for } z\in U\subset\U(y) &\implies f_{yx}(z) > h(z) - \epsilon \text{ by continuity of $h$}
\end{align*}\vspace{-2\baselineskip}\marginpar{should be also by continuity of $f_{yx}$}\vspace{2\baselineskip}
%
%xx
Is the metric used?
