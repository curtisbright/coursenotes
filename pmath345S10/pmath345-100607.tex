\defn A ring $R$ is Noetharian \emph{iff} every ideal of $R$ is finitely generated.  That is, $R$ is Noetharian \emph{iff} every ideal $I$ of $R$ can be written in the form $I=(r_1,\dotsc,r_n)$ for some $r_1$, $\dotsc$, $r_n\in R$.%rings that are not Noetharian are basically criminal.

\thm A ring $R$ is Noetharian \emph{iff} it satisfies the Ascending Chain Condition. \\
\pf Forwards: Say $R$ is Noetharian, and let $I_1\subset I_2\subset\dotsb$ be an ascending chain of ideals.  We want to show that there is an index $n$ such that $I_n=I_m$ for all $m\geq n$.

We've already seen that $I=\bigcup_k I_k$ is an ideal, so since $R$ is Noetharian, $I=(r_1,\dotsc,r_m)$ for some $r_1$, $\dotsc$, $r_m\in R$.  For each $i$, $r_i\in I$ implies $r_i\in I_m$, for some $m_i$.

If $n=\max\brace{m_i}$, then $r_i\in I_n$ for all $i$.  So $I=(r_1,\dotsc,r_m)\subset I_n\subset I$, and therefore $I=I_n$ and $I_m=I_n$ for all $m\geq n$.

Backwards: We'll skip. \qed

\thm (Hilbert Basis Theorem) Let $R$ be a Noetharian ring.  Then $R[x]$ is also Noetharian. \\
\remarks Every field is Noetharian, as is every PID.  By induction, HBT implies that $F[x_1,\dotsc,x_n]$ is Noetharian for every field $F$. \\
\pf Let $I\subset R[x]$ be any ideal.  We want to find a finite set of elements $f_1$, $\dotsc$, $f_n\in R[x]$ such that $I=(f_1,\dotsc,f_n)$.  Let $L=\text{set of leading coefficients of elements of $I$}$ (leading coefficient of $0$ is $0$).

\claim $L$ is an ideal of $R$. \\
\pf \begin{enumerate}[label=(\arabic*)]
\item $0\in L$ \checkmark
\item Say $l_1$, $l_2\in L$.  Let $f_1$, $f_2\in I$ have leading coefficients $l_1$, $l_2$ respectively.  If $\deg f_1\geq\deg f_2$, then $f_1-x^{\deg f_1-\deg f_2}f_2$ is in $I$ and has leading coefficient $l_1-l_2$, so $l_1-l_2\in L$.  Otherwise, $x^{\deg f_2-\deg f_1}f_1-f_2$ will do.
\item Say $l\in L$, $r\in R$, $f\in I$ with leading coefficient $l$.  Then $rf$ has leading coefficient $lr$, so $lr\in L$. \qed
\end{enumerate}
Since $R$ is Noetharian, we get $L=(a_1,\dotsc,a_n)$ for some $a_1$, $\dotsc$, $a_n\in R$.  Let $f_1$, $\dotsc$, $f_n\in I$ have leading coefficients $a_1$, $\dotsc$, $a_n$, respectively.  For each integer $d\geq0$, define
\[L_d=\brace{\text{set of leading cofficients of elements of $I$ of degree $d$}}\cup\brace{0}\]
It turns out (by a proof similar to Claim's) that $L_d$ is an ideal of $R$, so we can write $L_d=(b_{d,1},\dotsc,b_{d,n_d})$ for some $b_{d,i}\in R$.  Let $f_{d,i}\in I$ have leading coefficient $b_{d,i}$, with $\deg f_{d,i}=d$. \\
Let $N=\max\brace{\deg f_i}$.

\claim $I$ is generated by $f_1$, $\dotsc$, $f_n$ and $f_{d,i}$ for $d_i\leq N$. \\
\textbf{Proof of claim:}~It's clear that every $f_i$ and $f_{d,i}$ is contained in $I$, so it suffices to show that every element of $I$ can be written in terms of $f_i$ and $f_{d,i}$.

Assume $f\in I$ is the element of smallest degree that cannot be written as an $R[x]$-linear combination of the $f_i$ and $f_{d,i}$. ($d=\deg f$)

\textbf{Case I:}~$\deg f\geq N$.  Let $a=\text{leading coefficient of $f$}$.  Since $a\in L$, we can write $a=r_1a_1+\dotsb+r_na_n$ for some $r_i\in R$.  So $f-r_1x^{d-\deg f_1}f_1-\dotsb-r_n x^{d-\deg f_n}f_n=g$ has degree less than $d$, and is nonzero by construction of $f$.  This implies that $g$ cannot be written as an $R[x]$-linear combination of $f_i$ and $f_{d,i}$, which contradicts minimality of $f$.

\textbf{Case II:}~$\deg f< N$.  Then $a\in L_d$ for $\deg f=d<N$, so the Case I argument applies to $L_d$ instead of $L$.  By contradiction, we're done. \qed
