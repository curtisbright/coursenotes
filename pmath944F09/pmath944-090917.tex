\textbf{Theorem 1:} Let $\Lambda_1$ be a sublattice of $\Lambda$ in $\R^n$.
\begin{enumerate}
\item[A)] $\checkmark$
\item[B)] If $w_1\c \dotsc\c w_n$ is a basis for $\Lambda_1$, then there is a basis $v_1\c \dotsc\c v_n$ of $\Lambda$ such that~\eqref{star090915} holds, i) and ii) hold, and
\begin{enumerate}
\item[iii)$'$] $0\leq a_{ij} < a_{ii}$, for $1\leq j<i\leq n$.
\end{enumerate}
\end{enumerate}
\textbf{Proof B):} Let $w_1\c \dotsc\c w_n$ be a basis for $\Lambda_1$. \\ Let $D$ be the index of $\Lambda_1$ in $\Lambda$.  Recall that $D\Lambda$ is a sublattice of $\Lambda_1$.  In particular, by part A), there is a basis $Dv_1\c \dotsc\c Dv_n$ of $D\Lambda$ such that
\begin{align*}
Dv_1 &= a_{11}w_1 \\
&\eqvdots \\
Dv_n &= a_{n1}w_1 + \dotsb + a_{nn}w_n
\end{align*}
with $a_{ij}\in\Z$.

Put
\[ A = \begin{pmatrix}
a_{11} & & \\
\vdots & \ddots & \\
a_{n1} & \cdots & a_{nn}
\end{pmatrix} \]
\[ A \begin{pmatrix}
w_1 \\
\vdots \\
w_n
\end{pmatrix} = 
D \begin{pmatrix}
v_1 \\
\vdots \\
v_n
\end{pmatrix} \]
and so
\[ \begin{pmatrix}
w_1 \\
\vdots \\
w_n
\end{pmatrix} =
D\cdot\frac{\adj A}{\det(A)} \begin{pmatrix}
v_1 \\
\vdots \\
v_n
\end{pmatrix} \]
Further,%\marginpar{\vspace{2em}[lower triangular]}%
\[ \adj A = \begin{pmatrix}
b_{11} & & \\
\vdots & \ddots & \\
b_{n1} & \cdots & b_{nn}
\end{pmatrix} \]
with $b_{ij}\in\Z$.  Note that $w_i$ can be expressed as a rational linear combination of $v_1\c \dotsc\c v_n$ and that it is an integral linear combination of $v_1\c \dotsc\c v_n$.%\marginpar{[?Because of the uniqueness of representation in a basis??]}%

Thus we obtain~\eqref{star090915} with i) holding.  To obtain ii), it suffices to change the sign of $v_i$ if necessary, for $i=1\c \dotsc\c n$.

Finally, to obtain iii)$'$, we replace $v_i$ by $v'_i$, where
\[ v'_i = c_{i1}v_1 + \dotsb + c_{i,i-1}v_{i-1}+ v_i \]
where the $c_{ij}$s are integers chosen as in A) to ensure iii)$'$.

\textbf{Corollary 1:} Let $\Lambda$ be a lattice in $\R^n$, and let $w_1\c \dotsc\c w_m$ be linearly independent vectors of $\Lambda$.  Then there exists a basis $v_1\c \dotsc\c v_n$ of $\Lambda$ for which
\begin{align*}
w_1 &= a_{11}v_1 \\
w_2 &= a_{21}v_1 + a_{22}v_2 \\
&\eqvdots \\
w_m &= a_{m1}v_1 + \dotsb + a_{mm}v_m
\end{align*}
with $a_{ij}$s in $\Z$, $a_{ii}>0$, and $0\leq a_{ij}<a_{ii}$ for $1\leq j<i\leq m$.

\pf Extend $w_1\c \dotsc\c w_m$ to a set of $n$ linearly independent vectors $w_1\c \dotsc\c w_n$ of $\Lambda$.  Consider the sublattice $\Lambda_1$ generated by the basis $w_1\c \dotsc\c w_m$ and apply Theorem 1.%\marginpar{[?Given $m$ linearly independent vectors in a lattice $\Lambda$ in $\R^n$,  when can one extend these vectors to a basis for $\Lambda$?]}%

\textbf{Corollary 2:} Let $w_1\c \dotsc\c w_m$ be linearly independent vectors from a lattice $\Lambda$ in $\R^n$, with $m<n$.  There exist $w_{m+1}\c \dotsc\c w_n$ in $\Lambda$ such that $w_1\c \dotsc\c w_n$ is a basis for $\Lambda$, if and only if every vector $a_1w_1+\dotsb+a_mw_m$ in $\Lambda$ with $a_i\in\R$ for $i=1\c \dotsc\c m$ has in fact $a_i\in\Z$ for $i=1\c \dotsc\c m$.

\pf $\Longrightarrow$: immediate.

$\Longleftarrow$: We apply Corollary 1 to get a basis $v_1\c \dotsc\c v_n$ of $\Lambda$ with
\[
\begin{aligned}
w_1 &= a_{11}v_1 \\
&\eqvdots \\
w_m &= a_{m1}v_1 + \dotsb + a_{mm}v_m
\end{aligned} \qquad a_{ij}\in\Z, a_{ii}>0
\]
Thus, $v_1=\frac{1}{a_{11}}w_1$, and we get by hypothesis $\frac{1}{a_{11}}\in\Z$, hence $a_{11}=1$\footnote{$a_{11}=\pm1$, $a_{11}>0$}. \\
Next, $w_2=a_{21}v_1+a_{22}v_2$, hence $\frac{1}{a_{22}}w_2=\frac{a_{21}}{a_{22}}w_1\footnote{$=v_1$}+v_2$, $\implies a_{22}=1$. \\
In this way, we find $a_{11}=a_{22}=\dotsb=a_{mm}=1$. \\
Then $w_1\c \dotsc\c w_m\c v_{m+1}\c \dotsc\c v_n$ is a basis for $\Lambda$.

\textbf{Corollary 3:} Let $v_1\c \dotsc\c v_n$ be a basis for $\Lambda$ and let $w=a_1v_1+\dotsb+a_nv_n$ be in $\Lambda$, so $a_i\in\Z$ for $i=1\c \dotsc\c n$. \\
Let $m$ be an integer with $1\leq m\leq n-1$. \\
Then \[ \text{$v_1\c \dotsc\c v_{m-1}\c w$ can be extended to a basis for $\Lambda$ $\iff$ $\gcd(a_m,\dotsc,a_n)=1$.} \]

\pf $\Longrightarrow$: Let $g=\gcd(a_m,\dotsc,a_n)$. \\
If $v_1\c \dotsc\c v_{m-1}\c w$ can be extended by say $w_{m+1}\c \dotsc\c w_n$ to a basis for $\Lambda$, then
\[ w - a_1v_1 - \dotsb - a_{m-1}v_{m-1} = a_mv_m + \dotsb + a_nv_n \]
therefore
\[ \tfrac1g(w-a_1v_1-\dotsb-a_{m-1}v_{m-1}) = \tfrac{a_{m}}{g}v_m + \dotsb + \tfrac{a_n}{g}v_n . \]
Now, $\frac{a_t}{g}\in\Z$, for $t=m\c \dotsc\c n$%\marginpar{[therefore R.H.S. $\in\Lambda$]}%

Thus $\frac1gw-\frac{a_1}gv_1-\dotsb-\frac{a_{m-1}}gv_{m-1}$ is in $\Lambda$.  We now apply Corollary 2 to conclude $\frac1g\in\Z$, hence $g=1$.

$\Longleftarrow$: We wish to find $w_{m+1}\c \dotsc\c w_n$ in $\Lambda$ for which $v_1\c \dotsc\c v_{m-1}\c w\c w_{m+1}\c \dotsc\c w_n$ is a basis for $\Lambda$. \\
Then:
\begin{align*}
v_1 &= v_1 \\
&\eqvdots \\
v_{m-1} &= v_{m-1} \\
w &= a_1v_1 + \dotsb + a_mv_m + \dotsb + a_nv_n \\
w_{m+1} &= b_1v_1 + \dotsb + b_mv_m + \dotsb + b_nv_n \qquad b_i\in\Z \\
&\eqvdots \\
w_n &= z_1v_1 + \dotsb + z_mv_m + \dotsb + z_nv_n \qquad z_i\in\Z
\end{align*}
It suffices to show that we can choose the coefficients $b_1\c \dotsc\c b_n\c \dotsc\c z_1\c \dotsc\c z_n$ as integers in such a way that the associated coefficient matrix has determinant $\pm1$.  Notice that it is enough to show that the row $(a_m,\dotsc,a_n)$ can be extended to an $(n-m+1)\times(n-m+1)$ matrix with integer entries and determinant $\pm1$.

Consider the standard lattice $\Lambda_0$ in $\R^{n-m+1}$.  It now suffices to show that we can extend $(a_m,\dotsc,a_n)$ to a basis for $\Lambda_0$.  We appeal to Corollary 2.  Notice that if $\alpha\in\R$ with $\alpha\neq0$, and $\alpha(a_m,\dotsc,a_n)$ is in $\Lambda_0$, then $\alpha\in\Q$, say $\alpha=\frac pq$ with $p$ and $q$ coprime non-zero integers. \\
Then
\[ \paren*{\frac{pa_m}{q},\dotsc,\frac{pa_n}{q}}\in\Lambda_0 \]
hence $q\mid pa_m\c \dotsc\c q\mid pa_n$, and so, since $p$ and $q$ are coprime, $q\mid\gcd(a_m,\dotsc,a_n)$. %$\Box$

Recall the standard dot product of two vectors $v=(a_1,\dotsc,a_n)$ and $w=(b_1,\dotsc,b_n)$ in $\R^n$, given by $v\cdot w=a_1b_1+\dotsb+a_nb_n$.  Let $v_1\c \dotsc\c v_n$ be a basis for a lattice $\Lambda$ in $\R^n$.  Since $v_1\c \dotsc\c v_n$ are linearly independent, there exist vectors $v^*_1\c \dotsc\c v^*_n$ such that
\[ v^*_j \cdot v_i = \begin{cases}
1 & \text{if $i=j$} \\
0 & \text{if $i\neq j$}
\end{cases} \]
$v^*_1\c \dotsc\c v^*_n$ are linearly independent, and they generate a lattice $\Lambda^*$ in $\R^n$.  $\Lambda^*$ is known as the \emph{polar lattice} of $\Lambda$, and one can show that it does not depend on the choice of basis for $\Lambda$.

\textbf{Theorem 2:}  Let $\Lambda$ be a lattice in $\R^n$.  The polar lattice $\Lambda^*$ of $\Lambda$ consists of all vectors $v^*$ in $\R^n$ for which $v^*\cdot v$ for all $v$ in $\Lambda$.  Further,
\[ d(\Lambda)\cdot d(\Lambda^*) = 1 . \]
\pf If $v_1\c \dotsc\c v_n$ is\ldots
