\begin{enumerate}
\item Preliminaries
\item Units/Zero Divisors
\item Polynomials
\item Homomorphisms
\item Ideals and Quotients
\item Localization and fields of fractions
\item \emph{Euclidean domains}
\end{enumerate}

Recall the division algorithm for $\Z$. \\
Given $a,b\in\Z$, $a\neq0$ there exists $q,r\in\Z$ such that
\[ b = qa+r \]
and
\[ \abs{r} < \abs{a} . \]
\defin An integral domain $R$ is an \emph{Euclidean domain} if there exists a function $N\colon R\to\N$ with $N(0)=0$\footnote{for convenience}, such that given $a,b\in R$, $a\neq0$, there exists $q,r\in R$ with
\[ b = qa+r \qquad\text{and}\qquad N(r)<N(a) . \]
\ex $R=\Z$, $N(a)=\abs{a}$. \\
Such an $N$ is often referred to as a Euclidean norm for $R$.

\prop $F$ a field.  Given $f,g\in F[x]$, $f\neq0$.  There exist $q,r\in F[x]$ such that $g=qf+r$ where either $r=0$ or $\deg(r)<\deg(f)$. \\
\cor $F[x]$ is a Euclidean domain ($F$ a field) with
\[ N \coloneqq \begin{cases}
0 & \text{if $f=0$} \\
\deg(f)+1 & \text{if $f\neq0$}
\end{cases} . \]
\pf If $g=0$ then let $q=r=0$. $\checkmark$ \\
Assume $g\neq0$. \\
If $\deg(g)<\deg(f)$ then let $q=0$, $r=g$. $\checkmark$ \\
Assume $\deg(g)\geq\deg(f)$. \\
Induction on $\deg(g)$.
\[ \deg(g)=0 \implies \deg(f)=0. \]
Therefore $f,g\in F$, so units in $F$.
\[ g = \paren*{\frac{g}{f}} f + 0 \quad\checkmark \]
$\deg(g)=n$:
\begin{align*}
g &= b_0 + b_1x + \dotsb + b_nx^n & b_n &\neq 0 \\
f &= a_0 + a_1x + \dotsb + a_mx^m & a_m &\neq 0
\end{align*}
$m\leq n$ \\
Consider $g^* = g - \underbrace{f\cdot\paren*{\frac{b_n}{a_m}x^{n-m}}}$.  OK since $a_m\neq0$ in a field $F$. \\
The underbrace has leading term $(a_mx^m)(\frac{b_n}{a_m}x^{n-m})=b_nx^n=\text{leading term of $g$}$. \\
So $\deg(g^*)<\deg(g)=n$.  By Induction Hypothesis,
\[ g^* = q^* f + r \text{ where either $r=0$ or $\deg(r)<\deg(f)$} . \]
\[ g - f\cdot\paren*{\frac{b_n}{a_m}x^{n-m}} = q^* f + r \]
\cor (Factor Theorem): $F$ a field, $g\in F[x]$, $\lambda\in F$ \\
If $g(\lambda)=0$ (i.e., $\lambda$ is a \emph{root} of $g$) \\
then $(x-\lambda)$ is a \emph{factor} of $g$. \\
(i.e., $g=(x-\lambda)f$, for some $f\in F[x]$) \\
The converse is true as well.

\pf If $g=(x-\lambda)f$,
$g(\lambda)=(\lambda-\lambda)f=0f=0$ $\checkmark$ \\
Conversely, suppose $\lambda$ is a root of $g$. \\
By the proposition, there exists $f,r\in F[x]$ such that
\[ g = (x-\lambda)f + r \]
(we are dividing $g$ by $(x-\lambda)$) \\
with $N(r)<N(x-\lambda)=2$ \\
$\implies N(r)=0$ or $1$.

If $N(r)=1$ then $\deg r=0$ so $r=a_0\in F$, $a_0\neq 0$.
\begin{gather*}
g = (x-\lambda) f + a_0 \\
g(\lambda) = 0\cdot f + a_0 = a_0 \neq 0
\end{gather*}
contradiction.  Therefore $N(r)=0$, therefore $r=0$, therefore $g=(x-\lambda)f$. \\
\cor $F$ field. \\
$g\in F[x]$, $\deg(g)=n$ ($g\neq0$) \\
Then $g$ has at most $n$ roots.

\pf Induction on $n$. \\
$n=0$: $g$ is nonzero constant polynomial $\implies$ $g$ has no roots \\
$n>0$: $\lambda_1,\dotsc,\lambda_l$ be distinct roots of $g$. \\
Divide $(x-\lambda_l)$ into $g$ to get \\
$g=(x-\lambda_l)q$ (by previous corollary) \\
But $\deg(q)=n-1$ (since $F$ is an integral domain $\deg(PQ)=\deg(P)+\deg(Q)$) \\
\emph{For each} $i<l$,
\[ 0 = g(\lambda_i) = \underbrace{(\lambda_i-\lambda_l)}_{\mathclap{\text{$\neq0$ since $\lambda_i\neq\lambda_l$}}}q(\lambda_i) \]
By Induction Hypothesis, $l-1\leq n-1\implies l\leq n$.
