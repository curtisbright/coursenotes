\defin $F\subseteq L$ field extension. \\
Let $R\subseteq F$ subring of $F$ such that \\
$F=Q(R)$ (special case: $R=F$) \\
$a_1,\dotsc,a_n\in L$ \\
\begin{align*}
R[a_1,\dotsc,a_n] &= \text{\emph{The subring of $L$ generated by $a_1,\dotsc,a_n$ over $\R$}} \\
&= \text{intersection of all subrings of $L$ that contain $R$ and $a_1,\dotsc,a_n$} \\
F(a_1,\dotsc,a_n) &= \text{\emph{the subfield of $L$ generated by $a_1,\dotsc,a_n$ over $F$}} \\
&= \text{fraction field of $R[a_1,\dotsc,a_n]$}
\end{align*}
%figure L - F(a_1,...,a_n) - R[a_1,...,a_n] - R
%                          \ F              /
\[ \xymatrix{
L \ar@{-}[r] & F(a_1,\dotsc,a_n) \ar@{-}[r] & R[a_1,\dotsc,a_n] \ar@{-}[r] & R \\
  &                   & \ar@{-}[ul] F \ar@{-}[ur]                 &
} \]
\textbf{Exercises:}
\begin{enumerate}
\item[(a)] $R[a_1,\dotsc,a_n]$ is a subring of $L$
\item[(b)] $F(a_1,\dotsc,a_n)$ is the intersection of all subfields of $L$ with respect to $a_1,\dotsc,a_n$ and $F$.
\item[(c)] $R[a_1,\dotsc,a_n]=\set{f(a_1,\dotsc,a_n)}{f\in R[a_1,\dotsc,a_n]\footnotemark}\subseteq L$ \\
Need:
\begin{itemize}
\item Show $\supseteq$
\item Show RHS is a subring of $L$ and contains $R$, $a_1,\dotsc,a_n$
\end{itemize}
\item[(d)]
\[ F(a_1,\dotsc,a_n) = \set{\frac{f(a_1,\dotsc,a_n)}{g(a_1,\dotsc,a_n)}}{f,g\in F[x_1,\dotsc,x_n]\text{, }g(a_1,\dotsc,a_n)\neq0} \]
\end{enumerate}\footnotetext{polynomial ring}

\thm $F\subseteq L$ field extension, $a\in L$, $F$-algebraic, $h=\text{minimal polynomial of $a$ over $F$}$
\[ F[x]/(h) \cong F[a] = F(a) \]
and $[F(a):F]=\deg h$ \\
\pf Consider
\begin{align*}
\phi\colon &F[x] \to F[a] \\
& f \mapsto f(a)
\end{align*}
``evaluation at $a$ map'' ring homomorphism \\
By exercise (c), $\phi$ is surjective \\
%\[ \overset{\text{1st isomorphism theorem}}{\implies} F[x]/\ker\phi \cong F[a] \]
%\[ \xRightarrow{\text{1st isomorphism theorem}} F[x]/\ker\phi \cong F[a] \]
\[ \xRightarrow{\text{1st iso.~thm.}} F[x]/\ker\phi \cong F[a] \]
If $h\mid f$ then $f=hg$
\begin{align*}
&\implies f(a) = h(a)g(a) = 0 \\
&\implies f \in \ker\phi
\end{align*}
We have proved the reverse: if $f(a)=0$ then $h\mid f$. \\
Therefore $\ker\phi=(h)$, therefore $F[x]/(h)\cong F[a]$
\begin{align*}
\text{$h$ irreducible nonzero} &\implies \text{$(h)\neq(0)$ is prime in $F[x]$, $F[x]$ pid} \\
&\implies \text{$(h)$ is maximal} \\
&\implies \text{$F[a]$ is a field} \\
&\implies F[a] = F(a)
\end{align*}
[Why?  $(h)\subseteq(f)\subseteq F[x]$ \\
$\implies h=fg$ for some $g$ \\
$\implies$ $f$ is a unit $\implies (f)=F[x]$ \\
\phantom{$\implies$} or \\
\phantom{$\implies$} $g$ is a unit $\implies$ $f=g^{-1}h\implies f\in(h)\implies(f)=(h)$ \\
$h=a_0+a_1x+\dotsb+a_mx^m$ \\
$m=\deg h$, $a_m\neq0$ \\
$B=\brace{1,a,a^2,\dotsc,a^{m-1}}\subseteq F(a)$
% L - F(a) - F
%   \ ---- /
\[ \xymatrix{
L \ar@{-}`d[r]`[rr] \ar@{-}[r] & F(a) \ar@{-}[r] & F
} \]
\textbf{Claim:} $B$ is $F$-linearly independent \\
\pf $r_0\cdot1+r_1\cdot a+\dotsb+r_{m-1}a^{m-1}=0$, $r_i\in F$ \\
$\implies$ $f(a)=0$ where $f=r_0+r_1x+\dotsb+r_{m-1}x^{m-1}$ \\
$m=\text{smallest degree of a nonzero polynomial vanishing at $a$}$ \\
$\implies$ $f=0$ $\implies$ $r_i=0$: Claim 1

\textbf{Claim 2:} $\Span_F(B)=F(a)$ \\
\pf $b\in F(a)=F[a]$ \\
$\implies$ $b=f(a)$ for some $f\in F[x]$ \\
$f=r_0+r_1x+\dotsb+r_nx^n$ \\
$n=\deg f$ \qquad $r_n\neq0$ \\
Show $f(a)\in\Span_F(B)$ by induction on $n$.

$n<m$: $f(a)=r_0+r_1a+\dotsb+r_na^n\in\Span_F(B)$ \\
since $1,a,\dotsb,a^n\in B$ $\checkmark$

$n=m$: $b=f(a)=r_0+\dotsb+r_ma^m$
\[ \implies a_m = - \paren*{\frac{r_0}{r_m}+\frac{r_1}{r_m}a+\dotsb+\frac{r_{m-1}}{r_m}a^{m-1}} \in \Span_F(B) \]
Therefore $1,a,\dotsc,a^m\in\Span_F(B)$ \\
$\implies$ $b=r_0+r_1a+\dotsb+r_ma^m\in\Span_F(B)$

$n>m$: Induction Hypothesis: $1,a,\dotsc,a^{n-1}\in\Span_B(F)$
\begin{align*}
a^n = a(a^{n-1}) &= a(s_0+s_1a+\dotsb+s_{m-1}a^{m-1}) \\
&= s_0a + s_1a^2 + \dotsb + s_{m-1}a^m \\
&\in\Span_F\brace{a,a^2,\dotsc,a^m}\subseteq\Span_F(B)
\end{align*}
since $B=\brace{1,\dotsc,a^{m-1}}$ and $a^m\in\Span_F(B)$ by case $m=n$ \\
$b=f(a)=r_0+r_1a+\dotsb+r_na^n\in\Span_F(B)$: Claim 2
\[ [F(a):F] = \abs B = m = \deg h \]
\cor The above proof shows more: \\
$F\subseteq L$ field extension, $a\in L$ algebraic over $F$, $\deg(a/F)=m$ \emph{then} $\brace{1,a,\dotsc,a^{m-1}}$ is an $F$-basis for $F(a)$.
\[ \xymatrix{
 & a\in L & \\
F \ar@{-}[ur]\ar@{-}[rr]_{\deg=\deg(a/F)} & & \ar@{-}[ul] F(a)
} \]
