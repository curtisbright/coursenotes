$\Z_3=\Z/3\Z=\text{``Integers mod 3''}$

\defn Let $R$, $S$ be rings, $f\colon R\to S$ a homomorphism.  Then $f$ is an isomorphism \emph{iff} there is another homomorphism $g\colon S\to R$ such that $f\circ g=\id$ and $g\circ f=\id$.

\eg $f\colon\C\to\C$, $f(z)=\overline{z}$.  This is an isomorphism; the inverse of $f$ is $f$.%\marginpar{figure: complex plane, $i^2=-1$}
\begin{center}
\begin{small}
\begin{picture}(4,4)(-2,-2)
\put(0,-2){\line(0,1){4}}
\put(-2,0){\line(1,0){4}}
\put(0,0){\makebox(0,0){$\bullet$}}
\put(0,1){\makebox(0,0){$\times$}}
\put(0,1){\makebox(0,0)[r]{$z$~~}}
\put(0,-1){\makebox(0,0){$\times$}}
\put(1,0){\makebox(0,0){$\shortmid$}}
\put(1,-0.2){\makebox(0,0)[t]{$1$}}
\put(2,-0.1){\makebox(0,0)[tl]{Re}}
\put(0.1,2){\makebox(0,0)[tl]{Im}}
\put(2,1){$i^2=-1$}
\end{picture}
\end{small}
\end{center}

To prove $z=i$, we'd have to have some relationship between $z$, real numbers, and $+$ and $\cdot$:
\[ a_n z^n + \dotsb + a_1 z + a_0 = 0 \]
where $a_i\in\R$.  Then:
\[ a_n \overline{z}^n + \dotsb + a_1\overline{z} + a_0 = 0 \]
So there's no way to tell the difference between $i$ and $-i$.

\defn Let $f\colon R\to S$ be a homomorphism.  The image of $f$ is the set:
\begin{align*}
\im(f) &= \set{f(x)}{x\in R} \\
&= \text{range of $f$}
\end{align*}
and the kernel of $f$:
\[ \ker(f) = \set{x\in R}{f(x)=0} \]
\thm Let $f\colon R\to S$ be a homomorphism.  Then $f$ is 1--1 \emph{iff} $\ker(f)=\brace{0}$. \\
\pf Forwards is trivial, because $f(0)=0$. \\
\textbf{Backwards:} Assume $\ker f=\brace0$.  We want to show $f$ is 1--1.  If $f(a)=f(b)$, then $f(a-b)=0$, so $a-b\in\ker f$, so $a-b=0\implies a=b$. \qed

\thm Let $f\colon R\to S$ be a homomorphism.  Then:
\begin{enumerate}[label=(\arabic*)]
\item $f(0)=0$
\item The composition of homomorphisms is a homomorphism
%\item If $x$ is a unit/zero~divisor/nilpotent, then so is $f(x)$, unless $f(x)=0$.
\item If $x$ is a unit, then so is $f(x)$.
\end{enumerate}
\thm Let $f\colon R\to S$ be a homomorphism.  Then $\ker f$ is usually not a subring of $R$.  In fact, $\ker f$ is a subring of $R$ \emph{iff} $\ker f=R$.

\defn Let $R$ be a ring.  An ideal of $R$ is a subset $I\subset R$ satisfying:
\begin{enumerate}[label=(\arabic*)]
\item $0\in I$
\item If $a$, $b\in I$ then $a-b\in I$
\item If $a\in I$, $r\in R$, then $ar\in I$.
\end{enumerate}
\thm Let $f\colon R\to S$ be a homomorphism.  Then $\ker f$ is an ideal of $R$. \\
\pf \begin{enumerate}[label=(\arabic*)]
\item $f(0)=0\implies0\in\ker f$.
\item If $a$, $b\in\ker f$, then $f(a)=f(b)=0$.  We want $a-b\in\ker f$, i.e., $f(a-b)=0$.  This is trivial.
\item If $a\in\ker f$, $r\in R$, then $f(a)=0$, so $f(ra)=f(r)f(a)=f(r)\cdot0=0$.  So $ra\in\ker f$. \qed
\end{enumerate}
\eg What are the ideals of $\Z$? \\
$\brace0$ is the trivial or zero ideal. \\
$\Z$ is the improper or unit ideal. \\
$I=\brace{\text{even integers}}$ is an ideal, often written $2\Z$. \\
In fact, $\brace{\text{multiples of $n$}}=n\Z$ is an ideal of $\Z$. \\
Better yet, every ideal of $\Z$ is of the form $n\Z$ for some $n\in\Z$.

\defn Let $R$ be a ring, $a\in R$ any element.  The principal ideal of $R$ generated by $a$ is the set:
\[ (a) = aR = \set{aR}{r\in R} . \]
\thm $(a)$ is an ideal of $R$. \\
\pf Easy. \qed
