\url{~pmat345}

\begin{itemize}
\item $\Z$ Integers $\brace{\dotsc,-2,-1,0,1,2,\dotsc}$
\item $C[0,1]$ all continuous functions $f\colon[0,1]\to\R$
\end{itemize}
In both cases: \\
can ``add'': $(f+g)\colon[0,1]\to\R$, $x\mapsto f(x)+g(x)$ \\
can ``multiply'': $(fg)\colon[0,1]\to\R$, $x\mapsto f(x)g(x)$ \\
both $0$ and $1$\marginpar{figure: $0$ function and $1$ function}%

\defin A \emph{ring} $R$ is a set with two distinguished elements, $0$ and $1$, and two binary functions
\begin{align*}
\mathord{+}\colon R^2 \to R \\
\mathord{\times}\colon R^2 \to R
\end{align*}
i.e., given two elements $x,y$ we can add them $x+y\in R$, we can multiply them $xy\in R$\footnote{Note: drop the $\times$ sometimes.} \\
such that: for all $x,y,z\in R$,
\begin{enumerate}
\item Associativity of addition:
\[ (x+y)+z = x+(y+z)\footnote{Note: so we just write $x+y+z$} \]
\item Commutativity of addition:
\[ x+y = y+x \]
\item Neutrality of zero:
\[ x+0 = x\footnote{zero is also called ``additive identity''} \]
\item Existence of additive inverse: \\
For all $x\in R$ there is some $y\in R$ such that
\[ x+y=0\footnote{Note: We write $-x$ for $y$ here and call it the negative of $x$} \]
\item Associativity of multiplication:
\[ (xy)z = x(yz)\footnote{we just write $xyz$} \]
\item Neutrality of one:
\[ x1 = x = 1x \]
%\item Existence of multiplicative inverses: \\
%For all $x\neq0$ there is a $y\in R$ such that
%\[ xy=1 \]
\item Distributivity:
\begin{align*}
(x+y)z &= xz+yz \\
z(x+y) &= zx+zy
\end{align*}
\end{enumerate}
\textbf{Remarks:}
%alpha list
\begin{enumerate}
\item \textbf{WARNING:} What we call a ring here is a ``ring with identity'' for some people. \\
For us rings always have $1$. \\
\textbf{Example:} $2\Z$ set of even integers \\
For Dummit and Foote this is a ring, for us it is \emph{not}.
\item \textbf{Notation:} $x-y$ means $x+(-y)$
\item We don't ask $\times$ to be commutative.  Why? \\
\textbf{Example:} $M_2(\R)=\set{\paren{\begin{smallmatrix}a&b\\c&d\end{smallmatrix}}}{a,b,c,d\in\R}$
\begin{itemize}
\item $0=\paren{\begin{smallmatrix}0&0\\0&0\end{smallmatrix}}$
\item $1=\paren{\begin{smallmatrix}1&0\\0&1\end{smallmatrix}}$
\item $+$ matrix addition
\item $\times$ matrix multiplication
\end{itemize}
\textbf{Check:} This is a \emph{ring}.  $\times$ is not commutative.

Why should $+$ be commutative? \\
Because it is \emph{forced} by the other axioms.
\begin{align*}
(\overset{x}{1}+\overset{y}{1})(\overset{z}{a+b}) &= 1(a+b) + 1(a+b) \\
&= (a+b) + (a+b) \\
(\overset{z}{1+1})(\overset{x}{a}+\overset{y}{b}) &= (1+1)a + (1+1)b \\
&= (1a+1a) + (1b+1b) \\
&= (a+a) + (b+b) \\
(a+b) + (a+b) &= (a+a) + (b+b) \\
a+b+a+b &= a+a+b+b \\
\intertext{add $(-a)$ to both sides on the left}
b+a+b &= a+b+b \\
\intertext{add $(-b)$ to both sides on the right}
b+a &= a+b %boxed
\end{align*}
\end{enumerate}