No class Monday \\
(next class Tuesday)

Gauss' Structure Theorem
\[ U_{\prod p_i^{k_i}} \cong \prod U_{p_i^{k_i}} \]
$U_2=\brace1$, $U_4=\brace{1,3}$, $U_{2^k}=\chev{-1,5}\cong\chev{-1}\footnote{$\cong\Z_2$}\oplus\chev{5}\footnote{$\cong\Z_{2^{k-2}}$}$ \\
$U_{p^k}\cong\Z_{p^{k-1}(p-1)}$ \\
\cor $U_n$ is cyclic (equivalently $\exists$ primitive root mod $n$) \\
$\iff n=1,2,4,p^k,2p^k$ \\
\pf $\phi(p^k)$ is even except when $p=2$, $k=1$ and $\Z_a\oplus\Z_b$ is cyclic $\iff\gcd(a,b)=1$

\defn For a finite abelian group $G$, the \emph{universal exponent} of $G$ is
\[ \kappa(G) = \max_{a\in G}(\ord(a)) \]
For $G\cong\Z_{n_1}\oplus\dotsb\oplus\Z_{n_l}$ with $n_1\div n_2$, $n_2\div n_3$, $\dotsc$, $n_{l-1}\div n_l$ we have
\[ \kappa(G) = n_l \]
and for all $a\in G$
\[ \ord(a) \div n_l = \kappa(G) . \]
For $G=U_n$ we write
\[ \kappa(n) = \kappa(U_n) \]
For $n=\prod_{i=1}^l p_i^{k_i}$ where the $p_i$ are distinct primes and each $k_i\geq1$ we have
\[ \kappa(n) = \lcm(\kappa(p_1^{k_1}),\dotsc,\kappa(p_l^{k_l})) \]
with $\kappa(2)=1$, $\kappa(4)=2$, $\kappa(2^k)=\frac12\phi(2^k)=2^{k-2}$, $\kappa(p^k)=\phi(p^k)=p^{k-1}(p-1)$.

\eg Find the number of $x\in\Z_n$ such that $x^2=1$, where $n=\prod_{i=1}^l p_i^{k_i}$ \\
\soln Note that when $x^2=1\bmod n$ we must have $\gcd(x,n)=1$ so $x\in U_n$.

When $p_i$ is odd $U_{p_i^{k_i}}$ is cyclic and $\abs{U_{p_i^{k_i}}}$ is even so $U_{p_i^{k_i}}$ has one element of order $1$ and one of order $2$ so there are $2$ solutions to $x^2=1$ in $U_{p_i^{k_i}}$. \\
In $U_2$ there is $1$ solution (no elements of order $2$). \\
In $U_4$ there are $2$ solutions. \\
In $U_{2^k}$ there are $2$ solutions. ($U_{2^k}\cong\Z_2\oplus\Z_{2^{k-2}}$ so there are $2$ solutions in each of $Z_2$, $Z_{2^{k-2}}$) \\
The number of solutions to $x^2=1$ in $U_n$ for $n=\prod p_i^{k_i}$ with $p_1<p_2<\dotsb<p_l$ is
\begin{alignat*}{2}
&2^l &\qquad&\text{if $p_1\neq2$} \\
&2^{l-1} &&\text{if $p_1=2$, $k_1=1$} \\
&2^l &&\text{if $p_1=2$, $k_1=2$} \\
&2^{l+1} &&\text{if $p_1=2$, $k\geq3$}
\end{alignat*}
\textbf{Primality Testing} \\
\thm (Wilson's) Let $n\in\Z^+$.  Then $2\leq n\in\Z$ is prime
\[ \iff (n-1)! = -1 \bmod n \]
\pf Suppose $n=p$ is prime. \\
Then for all $a\in U_p$
\[ a^{p-1} = 1 \bmod p \]
so each $a\in U_p$ is a root of $f(x)=x^{p-1}-1$ over $\Z_p$. \\
$\therefore$ The roots of $f(x)$ are the elements $1$, $2$, $\dotsc$, $p-1\in U_n$.
\[ x^{p-1}-1 = f(x) = (x-1)(x-2)\dotsm(x-(p-1)) \]
Put in $x=0$ to get
\begin{align*}
-1 &= (-1)(-2)\dotsm(-(p-1)) \\
&= (-1)^{p-1}(p-1)! \\
&= (p-1)! \qquad\text{if $p$ is odd so $(-1)^{p-1}=1$}
\end{align*}
and when $p=2$,
\[ (p-1)! = 1! = 1 = -1 \bmod 2 \]
Conversely, suppose $n$ is composite, say $n=kl$ with $1<k,l<n$.
\begin{align*}
(n-1)! &= 1 \cdot 2 \dotsm k \dotsm (n-1) \\
&= 0 \bmod k \\
\text{and } 0 &= -1 \bmod k \\
\text{so } (n-1)! &\neq -1 \bmod k \\
\therefore (n-1)! &\neq -1 \bmod n
\end{align*}
\defn If $n$ is prime then
\[ b^n = b \bmod n \text{ for all $b\in\Z$ (by F$\ell$T)} \]
Given $2\leq n\in\Z$, for $b\in\Z^+$ if $b^n=b\bmod n$ then we say $n$ \emph{passes} the \emph{base\/ $b$ test} for primality (in this case, $n$ is probably prime) \\
if $b^n\neq b\bmod n$ we say $n$ \emph{fails} the base $b$ test (in this case $n$ is composite) \\
When $n$ passes the base $b$ test but is composite, we call $n$ a \emph{base\/ $b$ pseudo prime}. \\
A base $2$ pseudo prime is just called a \emph{pseudo prime}. \\
If $n$ is a base $b$ pseudo prime for \emph{every} base $b$ then $n$ is called a \emph{Carmichael} number. \\
\eg Show that $341$ is a pseudo prime (it's the smallest one). \\
\soln $n=341=11\cdot31$ \\
We need to show that
\begin{align*}
2^n &= 2 \bmod n \\
2^{341} &= 2 \bmod 341
\end{align*}
\textbf{mod $11$}, powers repeat every $\phi(11)=10$ terms so $2^{341}=2^1=2\bmod31$, powers repeat every $\phi(31)=30$ terms so $2^{341}=2^{11}$ and we have
\[ \begin{array}{c|cccccc}
k & 0 & 1 & 2 & 3 & 4 & 5 \\ \hline
2^k & 1 & 2 & 4 & 8 & 16 & 1
\end{array} \]
so powers of $2$ repeat every $5$ terms so
\[ 2^{341} = 2^1 = 2 \bmod 31 \]
$\therefore 2^{341}=2 \bmod 341$ by CRT

\eg Show that if $n$ is a pseudo prime then so is $M_n=2^n-1$. \\
\soln Suppose $n$ is a pseudo prime. \\
Since $n$ is composite so $M_n$ is composite.
\[ \brack{n=kl \implies M_n = 2^n - 1 = (2^k-1)((2^k)^{l-1}+\dotsb+1) } \]
and we have $2^n=2\bmod n$. \\
Say $2^n-2=nq$ (we need to show that
\begin{align*}
2^{M_n} &= 2 \bmod M_n \\
2^{2^n-1} &= 2 \bmod 2^n-1 )
\end{align*}
\begin{align*}
\text{Then } 2^{2^n-1}-2 &= 2^{nq+1} - 2 \\
&= 2(2^{nq}-1) \\
&= 2(2^n-1)((2^n)^{q-1}+\dotsb+1)
\end{align*}
$\therefore(2^n-1)\div2^{2^n-1}-2$, $2^{2^n-1}=2\bmod 2^n-1$ \\
\cor There are infinitely many pseudo primes.

For $2<n\in\Z$, $n$ is a Carmichael Number
\[ \iff n = p_1 p_2 \dotsm p_l \]
for some distinct primes $p_i$ with $l\geq2$ such that
\[  (p_i-1) \div (n-1) \]
for all $i$.
