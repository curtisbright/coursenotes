%\marginpar{
$f\in\overline \B$\\
$\implies p(f)\in\overline \B$\\
$f^2+2f+10$%}%
%\marginpar{
$f_{x_1x_2}=h(x_1)$, $f_{x_1x_2}(x_2)=h(x_2)$\\
$f_{xy}$\\
Let $\epsilon>0$ and $x\in A$.  For $y\in A$, $\exists$ neighborhood $\U(y)$ of $y$ such that
\[ f_{yx}(z) > h(z)-\epsilon \text{ for all $z\in\U(y)$} \]
(simply because $h$ is continuous) %incorrect statement? (oversight)
%}%
%\savenotes\marginpar{
\begin{gather*}
f_{yx}(y)=h(y) \\
f_{yx}(y\footnote{$z\in\U(y)$})>h(y\footnote{$z$})-\epsilon \\
f_{yx}(z)>h(z)-\epsilon
\end{gather*}%}\spewnotes%
%
\textbf{Baire's Category Theorem} \\
Reference on page 175, chapter 3, Exercise 33.  Let $M$ be a metric space.  A set $S\subset M$ is called \emph{nowhere} dense (in $M$) if for every [nonempty] open $U$, we have $\cl(S)\cap U\neq U$, or equivalently
\[ \Int(\cl(S)) \savenotes\mathrel{\mathord=\footnote{(typo $\neq$ in text)}}\spewnotes \emptyset \]
Show that $\R^n$ cannot %\\
be written as a countable union of nowhere dense sets.

\defn A set $A\subset M$ is of \emph{first} category (in $M$) if it is the union of countably many nowhere dense sets.  Else $A$ is of second category.

The exercise above can be phrased as: $\R^n$ is of 2nd category.

\thm (Baires) Every complete metric space $M$ is of 2nd category (in $M$).

\egs Let the metric space $M$ be $\R$.  Is $\N\subseteq\R$ of 1st category or 2nd category?  Answer: 1st.  $\N$ is of first category in $\R$.

Baire's Theorem gives:
\begin{center}$\N$ is of 2nd category in $\N$\end{center}
\marginpar{In $\N$,
\begin{align*}
\cl(\brace2)&=\brace2 \\
\Int(\cl\brace2)&=\Int(\brace2)\\&=\brace2\neq\emptyset
\end{align*}
}%
