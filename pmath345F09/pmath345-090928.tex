\defin $R$ commutative ring, $P\in R[x]$ \\
Suppose $S$ is an extension of $R$ \\
Given that $s\in S$, we can \emph{substitute $s$ for $x$}

$P(s)\in S$ as follows: \\
if $P=a_0+a_1x+\dotsb+a_nx^n$, $n=\deg P$ \\
then $P(s)=\underbrace{a_0+a_1s+a_2s^2+\dotsb+a_ns^n}_{\footnotemark}$\footnotetext{$+$ and $-$ are happening in $S$} \\
each $a_i\in R\subseteq S$ \\
$s\in S$

Another way of describing this is: \\
$R$ is a subring of $S$ \\
so $R[x]$ is a subring of $S[x]$ (check) \\
so $P\in S[x]$ and consider \\
$f_P\colon S\to S$ \\
Then $P(s)\coloneqq f_P(s)$ \\
\emph{``$P$ evaluated at $s$''}

\textbf{Homomorphisms} \\
\defin $R$, $S$ rings.  A \emph{homomorphism} $\phi\colon R\to S$ is a function with
\begin{align*}
\phi(1) &= 1\footnotemark \\
\footnotemark\phi(a+b) &= \phi(a) + \phi(b) \\
\phi(ab) &= \phi(a)\phi(b)
\end{align*}\addtocounter{footnote}{-1}\footnotetext{$*$ Different from text}\addtocounter{footnote}{1}\footnotetext{$a$, $b\in R$}
\textbf{Remark:} If $\phi$ is a homomorphism, then $\phi(0)=0$ and $\phi(a)=-\phi(a)$. \\
\pf \begin{align*}
0 + \phi(0) &= \phi(0+0) = \phi(0) + \phi(0) \\
\implies 0 &= \phi(0) \\
\varphi(-a) + \varphi(a) &= \varphi(-a+a) \\
&= \varphi(0) = 0 \\
\implies \phi(-a) &= -\phi(a)
\end{align*}

The \emph{image} of $\phi\colon R\to S$
\[ \phi(R) = \set{\phi(a)}{a\in R} \subseteq S \]
\textbf{Check:} $\phi(R)$ is a subring of $S$.

The \emph{kernel} of $\phi$
\[ \ker\phi = \set{a\in R}{\varphi(a)=0} \subseteq R \]
\textbf{Remark:} $\ker\phi$ is a subring $\iff$ $\ker\phi=R$ $\iff$ $S=\brace0$. \\
As long as $S$ is nontrivial, here it is \emph{not} a subring.\footnote{(for us, different in DF)}

\ex
\begin{enumerate} %alpha list
\item[(a)] $R$ is a subring of $S$ and
\begin{align*}
\phi\colon & R \to S \qquad\text{is the inclusion} \\
& r \mapsto r \qquad\text{$\phi$ is a homomorphism}
\end{align*}
When $R=S$ we call this the identity homomorphism
\item[(b)]%
\begin{align*}
\phi\colon & \C \to \C \qquad\text{homomorphisms} \\
& z \mapsto \overline{z} \qquad\text{conjugation map}
\end{align*}
$z=r+si$, $\overline{z}=r-si$
\item[(c)]%
\begin{align*}
\res\colon & \Z \to \Z_n, \qquad\text{$n$ fixed $\geq2$} \\
& a \mapsto \overline{a}=\set{b\in\Z}{a\equiv b\pmod n}
\end{align*}
homomorphism
\begin{gather*}
\res(1) = \overline1 = \text{identity in $\Z_n$} \\
\res(ab) = \overline{ab} = \overline{a}\overline{b} \\
\res(a+b) = \overline{a+b} = \overline{a}+\overline{b}
\end{gather*}
\item[(d)] What about homomorphisms from $\Z$ to $\Z$? \\
Suppose $\phi\colon\Z_n\to\Z$ was a homomorphism, then:
\begin{align*}
\phi(\overline1) &= 1 \\
\phi(\overline1+\overline1) &= \phi(\overline1)+\phi(\overline1)=1+1=2 \\
&\eqvdots \\
0 = \phi(\overline0) = \phi(\overline n) = \phi(\underbrace{\overline1+\overline1+\dotsb+\overline1}_{\text{$n$ times}}) &= n \qquad\text{in $\Z$}\footnotemark
\end{align*}\footnotetext{contradiction}
\emph{No} homomorphisms from $\Z_n$ to $\Z$.
\item[(e)] Fix any ring $R$, what are the homomorphisms from $\Z$ to $R$? \\
Consider $\phi\colon\Z\to R$
$a>0$ in $\Z$, $\phi(a)\coloneqq\overbrace{1_R+_R+\dotsb+_R1_R}^{\text{$a$ times}}$ \\
$a<0$ in $\Z$, $\phi(a)=-\phi(a)$ \\
$\phi(0)=0$

\textbf{check:} $\phi$ is a homomorphism \\
This is the \emph{only} possible since if 
$\psi\colon\Z\to R$ is any other my homomorphism.

then for $a>0$,
\begin{align*}
\psi(a) &= \psi(\underbrace{1+\dotsb+1}_{\text{$a$ times}}) \\
&= \psi(1) + \dotsb + \psi(1) \\
&= 1_R + \dotsb + 1_r = \phi(a)
\end{align*}
Hence $\psi=\phi$.
\end{enumerate}
\textbf{Point:} For any $R$ there is a unique homomorphism in $\Z$ to $R$.

\defin $\phi\colon R\to S$ a ring homomorphism
\begin{enumerate}
\item $\phi$ is \emph{injective} if $\phi$ is 1-to-1. \\
Also called \emph{embedding}, \emph{monomorphism}
\item $\phi$ is a \emph{surjective homomorphism} if
\[ \phi(R) = S \]
Also called a \emph{epimorphism}.
\item If $R=S$, then a homomorphism $\phi\colon R\to R$ is called \emph{endomorphism}
\item An \emph{isomorphism} is an injective \emph{and} surjective homomorphism.
\item If $\phi\colon R\to R$ is an isomorphism we call it an \emph{automorphism}.
\end{enumerate}

Suppose $\phi\colon R\to R$ is a homomorphism. \\
\lem $\phi\colon R\to S$ is an endomorphism iff $\ker\phi=\brace0$. \\
\pf If $\phi$ is an embedding and $\phi(a)=0=\phi(0)\implies a=0$, \\
i.e., $\ker\phi=\brace0$. \\
Conversely, suppose $\ker\phi=\brace0$.
%
\begin{align*}
\phi(a) &= \phi(b) \\
\phi(a)-\phi(b) &= 0 \\
\phi(a) + -(\phi(b)) &= 0 \\
\phi(a) + \phi(-b) &= 0 \\
\phi(a+(-b)) &= 0 \\
a + (-b) \in \ker\phi &= \brace0 \\
\implies a + (-b) &= 0 \\
\implies a &= b
\end{align*}
%
Ideals and Quotients \\
\defin An ideal $I$ of a ring $R$ is a \emph{nonempty} subset such that
\begin{enumerate}
\item $a,b\in I$, $(a+b)\in I$
\item for any $r\in R$ and $a\in I$, $ra\in I$ and $ar\in I$ in R
\end{enumerate}
\remark $0\in I$ \\
let $a\in I$, $-a=(-1)a$