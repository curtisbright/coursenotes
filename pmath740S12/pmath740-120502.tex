\textbf{Analytic Number Theory} \\
home page at \url{www.math.uwaterloo.ca/~snew} \\
outline, assignments, solutions

(Introduction to) \\
Analytic Number Theory by Apostol \\
Chapter 1, Chapter 5.

\textbf{Chapter 1:} Fundamental Theorem of Arithmetic \\
\thm (Division Algorithm) \\
Given $a$, $b\in\Z$ with $a\neq0$ there exist unique $q$, $r\in\Z$ with $b=qa+r$, $0\leq r<\abs{a}$.

\defn For $a$, $b\in\Z$ we say \emph{$a$ divides\/ $b$} (or $a$ is a \emph{factor of\/ $b$} or $b$ is a multiple of $a$) when $b=qa$ for some $q\in\Z$.

\textbf{basic facts}
\begin{align*}
& 0\div a\iff a=0 && a\div0\text{ for all $a\in\Z$} \\
& 1\div a\text{ for all $a\in\Z$} && a\div1\text{ for $a=\pm1$}
\end{align*}
\defn For $a$, $b\in\Z$ we say that $d$ is the \emph{greatest common divisor} of $a$ and $b$ when
\begin{itemize}
\item[] $d\geq0$
\item[] $d\div a$ and $a\div b$, and
\item[] if $c\div a$ and $c\div b$ then $c\div d$
\end{itemize}
\textbf{basic facts}
\begin{itemize}
\item[] $\gcd(0,0)=0$
\item[] $\gcd(0,a)=\abs{a}$
\item[] if $a\div b$ then $\gcd(a,b)=\abs{a}$
\item[] $\gcd(a,b)=\gcd(b,a)$
\item[] $\gcd(a,b)=\gcd(\abs{a},\abs{b})$
\item[] $\gcd(ac,bc)=\abs{c}\gcd(a,b)$
\item[] If $b=qa+r$ then $\gcd(a,b)=\gcd(b,r)$
\end{itemize}
\thm (Euclidean Algorithm with Back-Substitution) \\
Given $a$, $b\in\Z$
\[ d=\gcd(a,b) \text{ exists} \]
and there exist $x$, $y\in\Z$ such that $ax+by=d$ \\
\pf If $b\div a$ then
\[ d=\gcd(a,b)=\abs{b} \]
and we can easily find $x$, $y$. \\
Suppose $0<b<a$ and $b\nmid a$. \\
We apply the division algorithm repeatedly to get
\begin{alignat*}{2}
a &= bq_1 + r_1 &\qquad 0<r_1&<b \\
b &= r_1q_2 + r_2 & 0<r_2&<r_1 \\
r_1 &= r_2q_3 + r_3 & 0<r_3&<r_2 \\
&\eqvdots \\
r_{n-2} &= r_{n-1}q_n + r_n & 0<r_n&<r_{n-1} \\
r_{n-1} &= r_nq_{n+1} + r_{n+1} & r_{n+1} &= 0
\end{alignat*}
Then $\gcd(a,b)=\gcd(b,r_1)=\gcd(r_1,r_2)=\dotsb=\gcd(r_{n-1},r_n)=\gcd(r_n,0)=r_n$

This method of finding $\gcd(a,b)$ is called the Euclidean Algorithm
\begin{align*}
\text{We have } \gcd(a,b) &= r_n = r_{n-2} - r_{n-1}q_n \\
&= r_{n-2}u_1 + r_{n-1}u_2 \qquad\text{where $u_1=1$, $u_2=-q_n$} \\
&= r_{n-2}u_1 + (r_{n-3}-r_{n-2}q_{n-1})u_2 \\
&= r_{n-3}u_2 + r_{n-2}u_3, \qquad u_3=u_1-q_{n-1}u_2 \\
&= r_1u_{n-2} + r_2u_{n-1} \\
&= r_1u_{n-2} + (b-r_1q_2)u_{n-1} \\
&= bu_{n-1} + r_1u_n, \qquad u_n=u_{n-2}-q_2u_{n-1} \\
&= bu_{n-1} + (a-bq_1)u_n \\
&= au_n + bu_{n+1}, \qquad u_{n+1}=u_{n-1}-q_1u_n
\end{align*}
Thus we can define $u_1=1$, $u_2=-q_n$, $u_n=u_{k-2}-q_{n-k+2}u_{k-1}$ and then we can take $x=u_n$, $y=u_{n+1}$ to get $ax+by=d$.
\[
\begin{array}{@{}r@{}r@{}r@{}}
&& 2 \\ \cline{2-3}
429 & \big) & 1196 \\
& & 858 \\ \cline{3-3}
&& 338
\end{array}\qquad
\begin{array}{@{}r@{}r@{}r@{}}
&& 1 \\ \cline{2-3}
338 & \big) & 429 \\
& & 338 \\ \cline{3-3}
&& 91
\end{array}\qquad
\begin{array}{@{}r@{}r@{}r@{}}
&& 3 \\ \cline{2-3}
91 & \big) & 338 \\
& & 273 \\ \cline{3-3}
&& 65
\end{array}\qquad
\begin{array}{@{}r@{}r@{}r@{}}
&& 1 \\ \cline{2-3}
65 & \big) & 91 \\
& & 65 \\ \cline{3-3}
&& 26
\end{array}\qquad
\begin{array}{@{}r@{}r@{}r@{}}
&& 2 \\ \cline{2-3}
26 & \big) & 65 \\
& & 52 \\ \cline{3-3}
&& 13
\end{array}\qquad
\begin{array}{@{}r@{}r@{}r@{}}
&& 2 \\ \cline{2-3}
13 & \big) & 26 \\
& & 26 \\ \cline{3-3}
&& 0
\end{array}
\]
$\therefore d=\gcd(a,b)=13$ \\
We have
\[ \begin{array}{c|cccccc}
k & 1 & 2 & 3 & 4 & 5 & 6 \\ \hline
u_k & 1 & -2 & 3 & -11 & 14 & -39
\end{array} \]
$\therefore (1196)(14)+(429)(-39)=13$

\defn For $a$, $b\in\Z$ we say that $a$ and $b$ are \emph{coprime} when $\gcd(a,b)=1$.

\textbf{basic facts}
\begin{itemize}
\item[] $\gcd(a,b)=1\iff\exists x,y\in\Z\quad ax+by=1$
\item[] If $d=\gcd(a,b)$ then $\gcd(a/d,b/d)=1$.
\item[] If $a\div c$ and $b\div c$ and $\gcd(a,b)=1$ then $ab\div c$.
\item[] If $a\div bc$ and $\gcd(a,b)=1$ then $a\div c$.
\end{itemize}

\thm (The linear diophantine equation theorem) \\
Let $a$, $b$, $c\in\Z$ with $a$, $b$ not both zero.  Consider the equation $ax+by=c$.  This equation has an integer solution $(x,y)\iff\gcd(a,b)\div c$ and in this case if $(x_0,y_0)$ is one solution then the general solution is $(x,y)=(x_0,y_0)+t(-b/d,a/d)$ for $t\in\Z$ where $d=\gcd(a,b)$.

\defn Let $n\in\Z$.  $n$ is called \emph{prime} when $n>1$ and $n$ has exactly $2$ positive divisors, namely $1$ and $n$. \\
$n$ is called \emph{composite} when $n>1$ and $\exists k,l$ with $1<k,l<n$, $n=kl$.

\textbf{basic facts}
\begin{itemize}
\item[] for $a\in\Z$ either $p\div a$ or $\gcd(p,a)=1$
\item[] for $a$, $b\in\Z$ if $p\div ab$ then either $p\div a$ or $p\div b$
\item[] for $a_i\in\Z$ if $p\div a_1a_2\dotsm a_l$ then $p\div a_i$ for some $i$
\item[] for $a\in\Z$ if $p\div a^n$ then $p\div a$ where $n\in\Z^+$
\item[] If $n$ is composite then $n$ has a prime factor $p$ with $p\leq\sqrt{n}$.
\end{itemize}
