For $G=\Z_{n_1}\oplus\dotsb\oplus\Z_{n_l}$, $H$ abelian
\[ \Hom(G,H) = \set{f_{a_1,\dotsc,a_l}}{\text{each $a_i\in H$, $a_i^{n_i}=1$}} \]
$f_{a_1,\dotsc,a_l}(k_1,\dotsc,k_l)=\prod a_i^{k_i}$
\[ \hat G = \Hom(G,\C^*) = \set{f_{a_1,\dotsc,a_l}}{\text{each $a_i\in C_{n_i}$}} \]
\begin{align*}
\hat G &= \Hom(G,\C^*) \\
&= \Hom(G,\mathbb{S}) \qquad \mathbb{S}=\set{z\in\C^*}{\abs{z}=1} \\
&= \Hom\paren[\Big]{G,\bigcup_{n=1}^\infty} \qquad e^{i2\pi q} \qquad q\in\Q/\Z \\
&\cong \Hom(G,\Q/\Z)
\end{align*}
$G\cong\hat G$ under $\Phi\colon G\to\hat G$ defined by
\begin{align*}
\Phi(e_j) &= f_{1,\dotsc,1,e^{i2\pi/n_j},1,\dotsc,1} \\
\Phi(k_1,\dotsc,k_l) &= f_{(e^{i2\pi k_1/n_1},\dotsc,e^{i2\pi k_l/n_l})}
\end{align*}
For $\chi\in\hat U_l$ we associate the Dirichlet character
\[ \chi \colon \Z \to \C^* \]
given by
\[ \chi(k\footnote{$\in U_l$}) = \begin{cases}
\chi(k) & \text{if $\gcd(k,l)=1$ so $k\in U_l$} \\
0 & \text{otherwise}
\end{cases} \]
\notes $\chi\colon\Z\to\C^*$ is periodic with period $l$. \\
$\chi(k)$ is a $\phi(l)^k$ root of $1$ for all $k\in\Z$. \\
$\chi$ is completely multiplicative. \\
$\chi$ satisfies the following orthogonality relations\footnote{Aside: for $a\in U_l$,
\[ \sum_{\chi\in\hat U_l}\chi(a) = \begin{cases}
\phi(l) & \text{if $a=1$} \\
0 & a\neq1
\end{cases} \]
for $\chi\in\hat U_l$,
\[ \sum_{a\in U_l}\chi(a) = \begin{cases}
\phi(l) & \text{if $\chi=\1$} \\
0 & \text{}
\end{cases} \]
for $a$, $b\in U_l$,
\[ \sum_{\chi\in\hat U_l}\chi(a)\overline{\chi}(b) = \begin{cases}
\phi(l) & \text{if $a=b$} \\
0 & \text{}
\end{cases} \]
}
for $n\in\Z$,
\[ \sum_{\chi\in\hat U_l}\chi(n) = \begin{cases}
\phi(l) & \text{if $n=1\bmod l$} \\
0 & \text{otherwise}
\end{cases} \]
for $n$, $k\in\Z$,
\[ \sum_{\chi\in\hat U_l}\chi(n)\overline{\chi}(k) = \begin{cases}
\phi(l) & \text{if $n=k\bmod l$} \\
0 & \text{otherwise}
\end{cases} \]
for $\chi\in\hat U_l$,
\[ \sum_{k=0}^{l-1} \chi(k) = \begin{cases}
\phi(l) & \text{if $\chi=\1$} \\
0 & \text{otherwise}
\end{cases} \]
for $\chi$, $\psi\in\hat U_l$

\question Does $\sum_{n=1}^\infty\frac{\alpha^n}{n}$ converge where $\alpha=e^{i2\pi/l}$? \\
Does $\sum_{n=1}^\infty\frac{\chi(n)}{n}$ converge?

\thm Let $f\colon[1,\infty)\to\R^+$ by $\mathcal{C}^1$ and eventually decreasing with $\lim_{x\to\infty}f(x)=0$.  Let $\chi\in\hat U_l$ with $\chi\neq\1$.  Then $\sum_{n=1}^\infty\chi(n)f(n)$ converges and
\[ \sum_{n\leq x}\chi(n)f(n) = \sum_{n=1}^\infty \chi(n)f(n) + g(x) \]
where $\abs{g(x)}\leq\phi(l)\abs{f(x)}$ so $g(x)=O(f(x))$. \\
\pf Let $A(x)=\sum_{n\leq x}\chi(n)$. \\
Since $\chi(n)$ is periodic and
\[ A(l) = \sum_{n=1}^\infty \chi(n) = 0 \] 
$\therefore A(n)$ is periodic. \\
We have
\[ \abs{A(x)} = \max_{1\leq k\leq l}\abs{A(k)} = \max_{1\leq k\leq l}\abs{\chi(1)+\chi(2)+\dotsb+\chi(k)} \]
We have $\chi(1)+\chi(2)+\dotsb+\chi(k)+\chi(k+1)+\dotsb+\chi(l)=0$
\[ \abs{\chi(1)+\dotsb+\chi(k)} = \abs{\chi(k+1)+\dotsb+\chi(l)} \]
\begin{align*}
\text{LS} &\leq a = \text{\# of $i$ with $1\leq i\leq k$ such that $\gcd(i,l)=1$} \\
\text{RS} &\leq b = \text{\# of $i$ with $k+1\leq i\leq l$ with $\gcd(i,l)=1$} \\
a+b &= \phi(l) \\
\text{LS} &\leq \min(a,b) \leq \frac{\phi(l)}{2} \\
\therefore \abs{A(x)} &\leq \frac{\phi(l)}{2} \text{ for all $x$.}
\end{align*}
Consider the sum $\sum_{n=1}^\infty\chi(n)f(n)$. \\
For $x<y$.  By Abel's Summation Formula,
\begin{align*}
\sum_{n\leq y}\chi(n)f(n) - \sum_{n\leq x}\chi(n)f(n) &=
A(y)f(y) - \int_1^y A(t)f'(t)\d t - A(x)f(x) + \int_1^x A(t)f'(t)\d t \\
&= A(y)f(y) - A(x)f(x) - \int_x^y A(t) f'(t) \d t
\end{align*}
Each term can be made (arbitrarily) small by choosing $x$ large and $y>x$ since
\[ \abs{A(y)f(y)} \leq \frac{\phi(l)}{2} \abs{f(y)} \leq \frac{\phi(l)}{2} f(x) \to 0 \]
(since $f$ is eventually decreasing)
\[ \abs{A(x)f(x)} \leq \frac{\phi(l)}{2}f(x) \to 0 \]
and
\begin{align*}
\abs[\Big]{\int_x^y A(t) f'(t)\d t} &\leq \int_x^y\abs{A(t)}\abs{f'(t)}\d t \\
&\leq \int_x^\infty \abs{A(t)}\abs{f'(t)} \d t \\
&\leq \frac{\phi(l)}{2} \int_x^\infty \abs{f'(t)} \d t \\
&= \frac{\phi(l)}{2}\int_x^\infty-f'(t)\d t \\
&= \frac{\phi(l)}{2}\brack[\Big]{-f(t)}_x^{\infty} \\
&= \frac{\phi(l)}{2}f(x) \qquad\text{(since $f(t)\to0$ as $f\to\infty$)} \\
&\to 0
\end{align*}
By Cauchy's Criterion,
\[ \sum_{n=1}^\infty \chi(n)f(n) \text{ converges} \]
Also, taking the limit as $y\to\infty$ gives
\[ \sum_{n=1}^\infty \chi(n)f(n) -\sum_{n\leq x}\chi(n)f(n) = -A(x)f(x)-\int_x^\infty A(t)f'(t)\d t \]
\emph{so}
\[ \sum_{n\leq x}\chi(n)f(n) = \sum_{n=1}^\infty \chi(n)f(n) + g(x) \]
where
\[ g(x) = A(x)f(x) + \int_x^\infty A(t)f'(t)\d t \]
and
\begin{align*}
\abs{A(x)f(x)} &\leq \frac{\phi(l)}{2} f(x) \\
\abs[\Big]{\int_x^\infty A(t)f'(t)\d t} &\leq \int_x^\infty \abs{A(t)}\abs{f'(t)}\d t \\
&\leq \frac{\phi(l)}{2}\int_x^\infty -f'(t)\d t \\
&= \frac{\phi(l)}{2}f(x)
\end{align*}
so
\[ g(x) \leq \phi(l)f(x) \]
for large $x$.

\eg $\sum_{n=1}^\infty\frac{\chi(n)}{n}$ converges and $\sum_{n\leq x}\frac{\chi(n)}{n}+g(x)$ with $\abs{g(x)}\leq\phi(l)\frac1x$ \\
$\sum_{n=1}^\infty\frac{\chi(n)\log n}{n}$ converges
\[ \sum_{n\leq x}\frac{\chi(n)\log n}{n} = \sum_{n=1}^\infty \frac{\chi(n)\log n}{n} + g(x) \]
with $\abs{g(x)}\leq\phi(l)\frac{\log x}{x}$ for large $x$.
\begin{align*}
\sum_{\substack{p\leq x\\p=k\bmod l}}\frac{\log p}{p} &= \frac{1}{\phi(l)}\log\log x + O(1) \\
\sum\frac{\log p}{p} &\approx \sum\frac{\Lambda}{n} \\
\sum\frac1p &\approx \sum\frac{p(n)}{n} \\
\sum_{d\div n}\Lambda(d) &= \log n \\
\sum_{d\div n}p(d) &= \text{\# of distinct prime factors of $n$}
\end{align*}
