Theorem 44 \dots \\
$\brace{\alpha_1,\dotsc,\alpha_n}$ a basis for $I$
\begin{align*}
\disc\brace{\alpha_1,\dotsc,\alpha_n} &= \paren*{\begin{pmatrix}
a_{11} & & 0 \\
\vdots & \ddots & \\
a_{n1} & \cdots & a_{nn}
\end{pmatrix}}^2 D \\
&= (a_{11}\dotsm a_{nn})^2 D
\end{align*}
We showed that $NI\geq a_{11}\dotsm a_{nn}$.

To conclude suppose $\gamma\in\A\cap K$.  Then $\gamma=b_1\omega_1+\dotsb+b_n\omega_n$ with $b_i\in\Z$; here $\brace{\omega_1,\dotsc,\omega_n}$ is an integral basis for $\A\cap K$.  Note that, by the Division Algorithm, $b_n=q_n a_{nn} + r_n$ with $0\leq r_n<a_{nn}$ and then $\gamma-q_n\alpha_n=d_1\omega_1+\dotsb+d_{n-1}\omega_{n-1}+r_n\omega_n$.

Repeating this $n-1$ times we find that there exist integers $q_1$, $\dotsc$, $q_{n-1}$ so that
\[ \gamma - q_n\alpha_n - q_{n-1}\alpha_{n-1} + \dotsb + q_1\alpha_1 = r_1\omega_1 + \dotsb + r_n\omega_n \]
with $0\leq r_i<a_{ii}$.  Thus
\[ NI \leq a_{11}\dotsm a_{nn} \implies NI = a_{11}\dotsm a_{nn} . \]
\textbf{Corollary 50:} $[K:\Q]<\infty$.  Let $\alpha$ be a non-zero element of $\A\cap K$.  Then $N(\alpha)=\abs{N_\Q^K(\alpha)}$. \\
\pf Let $\brace{\omega_1,\dotsc,\omega_n}$ be an integral basis for $\A\cap K$.  Then the principal ideal $(\alpha)$ has $\brace{\alpha\omega_1,\dotsc,\alpha\omega_n}$ as an integral basis.

Let $\sigma_1$, $\dotsc$, $\sigma_n$ be the embeddings of $K$ in $\C$ which fix $\Q$.  Then
\begin{align*}
\disc\brace{\alpha\omega_1,\dotsc,\alpha\omega_n} &= \paren*{\det(\sigma_i(\alpha\omega_j))}^2 \\
D = \disc\brace{\omega_1,\dotsc,\omega_n} &= \paren*{\det(\sigma_i(\omega_j))}^2
\end{align*}
But we have
\begin{align*}
\disc\brace{\alpha\omega_1,\dotsc,\alpha\omega_n} &= \paren*{\det\begin{pmatrix}
\sigma_1(\alpha) & & 0 \\
& \ddots & \\
0 & & \sigma_n(\alpha)
\end{pmatrix}}^2\cdot D \\
&= (N_\Q^K(\alpha))^2 \cdot D .
\end{align*}
By Theorem~49 $\implies(N(\alpha))^2=(N_\Q^K(\alpha))^2$.  Thus $N(\alpha)=\abs{N_\Q^K(\alpha)}$ since $N(\alpha)$ is a non-negative integer.

\textbf{Theorem 51:} (Fermat's Theorem) Let $[K:\Q]<\infty$ and let $P$ be a prime ideal of $\A\cap K$.  Let $\alpha$ be an element of $\A\cap K$ with $P\nmid(\alpha)$ then
\[ \alpha^{NP-1} \equiv 1 \bmod P . \]
\pf Let $\beta_1$, $\dotsc$, $\beta_{NP}$ be a complete set of representatives for the cosets $\A\cap K/P$ (in $\A\cap K$ modulo $P$).  We may suppose $\beta_{NP}$ is congruent to $0\bmod P$.  Then since $P\nmid(\alpha)$ we see that
\[ \alpha\beta_1,\dotsc,\alpha\beta_{NP} \]
is again a complete set of representatives mod $P$ with $\alpha\beta_{NP}$ congruent to $0$ modulo $P$.  Therefore
\begin{align*}
\alpha\beta_1\dotsm\alpha\beta_{NP-1} &\equiv \beta_1\dotsm\beta_{NP-1} \bmod{P} . \\
\implies \alpha^{NP-1} &\equiv 1 \bmod{P}
\end{align*}
as required.

\textbf{Proposition 52:} Let $[K:\Q]<\infty$.  Let $A$ be a non-zero ideal of $\A\cap K$.  Then $NA\in A$. \\
\pf Let $\beta_1$, $\dotsc$, $\beta_{NA}$ be a complete set of representatives modulo $A$.  Then
\[ 1+\beta_1,\dotsc,1+\beta_{NP} \]
is also a complete set of representatives modulo $A$.
\begin{align*}
\implies \beta_1 + \dotsb + \beta_{NA} &\equiv (1+\beta_1) + \dotsb + (1+\beta_{NA}) \bmod{A} \\
0 &\equiv NA \bmod A
\end{align*}
Notice that for any positive integer $t$ there are only finitely many ideals $A$ of $\A\cap K$ with $NA=t$.

Still to show: The norm map on ideals is multiplicative, i.e., for $A$, $B$ ideals in $\A\cap K$
\[ NAB = NA\cdot NB . \]
If we have this and
\[ NA = p \text{ with $p$ a prime} \]
then $A$ is a prime ideal.  Further if $p$ is a prime in $\Z$ then 
\[ N(p) = \abs{N_\Q^K p} = p^n \text{ where $n=[K:\Q]$} . \]
Every prime ideal $P$ of $\A\cap K$ divides $(p)$ for exactly one prime.
\[ \implies NP = p^f \]
for some integer $f$ with $1\leq f\leq n$.
