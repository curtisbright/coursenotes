Let $\x_1\c \dotsc\c \x_m$ be on $S^{n-1}$ in $\R^n$.  Let $\theta_{ij}$ be the (angular) distance between $\x_i$ and $\x_j$ on $S^{n-1}$.
\setlength{\unitlength}{1cm}
\[ \begin{picture}(2,1)(-1,0)
\arc{2}{4.19}{5.24}
\put(0,0){\circle*{0.05}}
\path(0,0)(0.5,0.865)
\path(0,0)(-0.5,0.865)
\arc{0.5}{4.19}{5.24}
\put(0,0.5){\small\makebox(0,0){$\theta_{ij}$}}
\put(0.5,0.865){\makebox(0,0)[lb]{$\x_j$}}
\put(-0.5,0.865){\makebox(0,0)[rb]{$\x_i\vphantom{\x_j}$}}
\end{picture} \]
%figure: $\theta_{ij}$ is the angle between $\x_i$ and $\x_j$.
Notice that for any real numbers $t_1\c \dotsc\c t_m$ we have
\begin{align*}
\norm{t_1\x_1+\dotsb+t_m\x_m} &= (t_1\x_1+\dotsb+t_m\x_m,t_1\x_1+\dotsb+t_m\x_m) \\
&= \sum_{i=1}^m\sum_{j=1}^m t_i t_j \cos\theta_{ij} \geq 0 .
\end{align*}
Equivalently
\[ (t_1,\dotsc,t_m) \begin{pmatrix}\cos\theta_{ij}\end{pmatrix} \begin{pmatrix}t_1\\\vdots\\t_m\end{pmatrix} \geq 0 . \]
Thus the matrix $(\cos\theta_{ij})_{\substack{i=1,\dotsc,m\\j=1,\dotsc,m}}$ is positive semi-definite.

In 1943 Schoenberg proved that the matrix $(G^{(n)}_k(\cos(\theta_{ij})))_{\substack{i=1,\dotsc,m\\j=1,\dotsc,m}}$ is again positive semi-definite for any set of points $\x_1\c \dotsc\c \x_m$ on $S^{n-1}$ where the $G_k^{(n)}$s are Gegenbauer polynomials.

Schoenberg also proved that if $(f(\cos\theta_{ij}))_{\substack{i=1,\dotsc,m\\j=1,\dotsc,m}}$ is positive semi-definite for all choices of $\x_1\c \dotsc\c \x_m$ in $S^{n-1}$ then $f$ can be expressed as a linear combination (perhaps infinite) with non-negative coefficients of Gegenbauer polynomials.

We may define polynomials $C_k^{(n)}(t)$ by the expansion
\[ (1-2rt+r^2)^{(2-n)/2} = \sum_{k=0}^\infty r^k C_k^{(n)}(t) \qquad \text{for $n=3\c 4\c \dotsc$} . \]
We then put
\[ G_k^{(n)}(t) = \frac{C_k^{(n)}(t)}{C_k^{(n)}(1)}, \qquad \text{so $G_k^{(n)}(1)=1$} . \]
We may also define $G_k^{(n)}(t)$ for $n=1\c 2\c \dotsc$ recursively by the rules
\begin{gather*}
G_0^{(n)}(t) = 1, \qquad G_1^{(n)}(t) = t \qquad \text{and} \\
G_k^{(n)}(t) = \frac{(2k+n-4)tG_{k-1}^{(n)}(t) - (k-1)G_{k-2}^{(n)}(t)}{k+n-3}
\end{gather*}
In the special case that $n=3$ the polynomials are known as the Legendre polynomials.

Since $(G_k^{(n)}(\cos\theta_{ij}))_{\substack{i=1,\dotsc,m\\j=1,\dotsc,m}}$ is positive semi-definite, if $a_0\c \dotsc\c a_d$ are non-negative real numbers then
\[ (a_0G_0^{(n)}(\cos\theta_{ij})+\dotsb+a_dG_d^{(n)}(\cos\theta_{ij})) \]
is also positive semi-definite.  We put
\[ f(n,a_0,\dotsc,a_d)(t) = f(t) = a_0G_0^{(n)}(t) + \dotsb + a_dG_d^{(n)}(t) \]
and we define $S_f(\x_1,\dotsc,\x_m)$ by
\begin{align*}
S_f(\x_1,\dotsc,\x_m) &= \sum_{i=1}^m\sum_{j=1}^m f(\cos\theta_{ij}) \\
&= \sum_{k=0}^d a_k \sum_{i=1}^m\sum_{j=1}^m G_k^{(n)}(\cos(\theta_{ij})) .
\end{align*}
Thus, since $a_0\c \dotsc\c a_d$ are non-negative and $\sum_{i=1}^m\sum_{j=1}^m G_k^{(n)}(\cos\theta_{ij})\geq0$ for $k=0\c \dotsc\c d$ we see that
\begin{equation} S_f(\x_1,\dotsc,\x_m) \geq a_0 \sum_{i=1}^m\sum_{j=1}^m G_0^{(n)}(\cos(\theta_{ij})) = a_0 m^2 \label{star091110} . \end{equation}

Let us suppose now that $\x_1\c \dotsc\c \x_m$ is a configuration of $m$ points on $S^{n-1}$ which correspond to the $m$ points of contact by $m$ spheres of radius $1$ which surround $S^{n-1}$ in a kissing configuration.  Then $\theta_{ij}\geq\frac\pi3$ provided that $i\neq j$ hence $\cos(\theta_{ij})\leq\frac12$ for $i\neq j$.
%figure: three touching circles with equilateral triangle joining centres

Suppose that $a_0\c \dotsc\c a_d$ are non-negative real numbers for which $f(t)\leq0$ for $t$ in the range $[-1,\frac12]$.  Then
\begin{align*}
S_f(\x_1,\dotsc,\x_m) &\leq m f(1) \qquad\text{and so by~\eqref{star091110}, if $a_0>0$,} \\
m &\leq \frac{f(1)}{a_0}
\end{align*}%[terms with $i\neq j$ will be $\leq0$; drop these, keep only these with $i=j$]
The strategy is now to choose $a_0\c \dotsc\c a_d$ so that $f(t)\leq0$ for $[-1,\frac12]$ and such that $a_0$ is large and $f(1)$ is small.  There are two amazing applications of this approach.  They were found independently in 1979 by Odlyzko and Sloan and by Levenshtein and they treat the cases $n=8$ and $n=24$. %\\
For $n=8$ we consider
\[ f(t) = G_0^{(8)} + \tfrac{16}{7}G_1^{(8)} + \tfrac{200}{63}G_2^{(8)} + \tfrac{832}{231}G_3^{(8)} + \tfrac{1216}{429}G_4^{(8)} + \tfrac{5120}{3003}G_5^{(8)} + \tfrac{2560}{4641}G_6^{(8)} \]
then
\[ f(t) = \tfrac{320}{3}(t+1)(t+\tfrac12)^2t^2(t-\tfrac12) . \]
One can check that $f(t)\leq0$ for $[-1,\frac12]$.  Thus $\tau_8\leq\frac{320}{\cancel3}\cdot\cancel2\cdot\frac{3^{\cancel2}}{2^2}\cdot\cancel{\frac{1}{2}}=240$.

For $n=24$ one can find a non-negative linear combination of the $G_k^{(24)}$s to give $f(t)$ where
\[ f(t) = \tfrac{1490944}{15}(t+1)(t+\tfrac12)^2(t+\tfrac14)^2t^2(t-\tfrac14)^2(t-\tfrac12) \]
and $f(t)\leq0$ for $t$ in $[-1,\frac12]$.  Thus
\[ \tau_{24} \leq 196{,}560 . \]

We'll show that the $E_8$ lattice has kissing number $240$ and the Leech lattice has kissing number $196{,}560$.  Thus $\tau_8$ and $\tau_{24}$ are determined.  In general things don't go quite so smoothly.  This approach gives $\tau_3\leq13$ and $\tau_4\leq25$, yet we know $\tau_3=12$ and $\tau_4=24$.  The choice of $a_0\c \dotsc\c a_d$ is made after running linear programming packages.

Let us now return to lattices.  Recall $E_8$ has diagram %.-.-.|-.-.-.-.
\setlength{\unitlength}{0.5cm}\phantom{$\Bigm|$}\begin{picture}(6,0)(0,0.25)\multiput(0,0)(1,0){7}{\circle*{0.2}}\put(2,1){\circle*{0.2}}\put(0,0){\line(1,0){6}}\put(2,0){\line(0,1){1}}\end{picture}\phantom{$\Bigm|$}
.  With each vector normalized to have length $\sqrt2$ we have that the matrix $B$ of inner products is
\[ B = B(E_8(\sqrt2)) = \begin{pmatrix}
2 & 1 & 0 & 0 & 0 & 0 & 0 & 0 \\
1 & 2 & 1 & 0 & 0 & 0 & 0 & 0 \\
0 & 1 & 2 & 1 & 1 & 0 & 0 & 0 \\
0 & 0 & 1 & 2 & 0 & 0 & 0 & 0 \\
0 & 0 & 1 & 0 & 2 & 1 & 0 & 0 \\
0 & 0 & 0 & 0 & 1 & 2 & 1 & 0 \\
0 & 0 & 0 & 0 & 0 & 1 & 2 & 1 \\
0 & 0 & 0 & 0 & 0 & 0 & 1 & 2
\end{pmatrix} \]

One can check that $\det B=1$.  Notice that we may realize $E_8(\sqrt2)$ in the following way:
\setlength{\unitlength}{2cm}
\[\begin{picture}(6,1.75)(0,-0.5)
\multiput(0,0)(1,0){7}{\circle*{0.1}}
\put(2,1){\circle*{0.1}}
\put(0,0){\line(1,0){6}}
\put(2,0){\line(0,1){1}}
\put(0,-0.20){\makebox(0,0){\small$(1,-1,0,0,0,0,0,0)$}}
\put(1,0.20){\makebox(0,0){\small$(0,-1,1,0,0,0,0,0)$}}
\put(2,-0.20){\makebox(0,0){\small$(0,0,1,-1,0,0,0,0)$}}
\put(3,0.20){\makebox(0,0){\small$(0,0,0,-1,1,0,0,0)$}}
\put(4,-0.20){\makebox(0,0){\small$(0,0,0,0,1,-1,0,0)$}}
\put(5,0.20){\makebox(0,0){\small$(0,0,0,0,0,-1,1,0)$}}
\put(6,-0.20){\makebox(0,0){\small$(0,0,0,0,0,0,1,-1)$}}
\put(2,1.20){\makebox(0,0){\small$(\frac12,\frac12,\frac12,-\frac12,-\frac12,-\frac12,-\frac12,-\frac12)$}}
\end{picture}\]
\vspace{-\baselineskip}