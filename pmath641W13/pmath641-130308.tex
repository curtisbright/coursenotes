Theorem 43
\begin{gather*}
\vdots \\
\gamma J \subset J
\end{gather*}
$J$ is a finitely generated ideal in $R$ so $J=(a_1,\dotsc,a_n)$.

Then there exist $m_{ij}$ in $R$ so that
\[ \gamma a_i = m_{i1}a_1 + \dotsb + m_{in}a_n \]
for $i=1$, $\dotsc$, $n$.  Then
\[ (\gamma I_n - M)\begin{pmatrix}
a_1 \\
\vdots \\
a_n
\end{pmatrix} = \begin{pmatrix}0 \\
\vdots \\
0
\end{pmatrix} \]
where $M=(m_{ij})$.  $J\neq(0)$ so $\paren*{\begin{smallmatrix}
a_1 \\
\vdots \\
a_n
\end{smallmatrix}} = \paren*{\begin{smallmatrix}0 \\
\vdots \\
0
\end{smallmatrix}}\implies\det(\gamma I_n-M)=0$.  Thus $\gamma$ is the root of a monic polynomial with entries in $R$.  But $R$ is a Dedekind domain so $R$ is integrally closed in its field of fractions $K$.  Since $\gamma\in K$ we see that $\gamma\in R$.  This is a contradiction.

\textbf{Corollary 44:} Let $A$, $B$ and $C$ be non-zero ideals in a Dedekind domain $R$ with $AC=BC$ then $A=B$. \\
\pf There exists an ideal $J$ in $R$ so that $CJ$ is principal.  Say $CJ=(\alpha)$ with $\alpha\in R$.  Note that
\[ ACJ = BCJ \]
so $A(\alpha)=B(\alpha)$.
\[ \implies A\alpha = B\alpha \]
$\implies A=B$ since $\alpha\neq0$.

\textbf{Corollary 45:} Let $A$ and $B$ be non-zero ideal in a Dedekind domain $R$.
\[ A \mid B \iff B \subseteq A . \]
\pf $\Rightarrow$ Since $A\mid B$ there exists an ideal $C$ in $R$ with $AC=B$.  Then immediately $B\subseteq A$. \\
$\Leftarrow$ By Theorem~43 there exists a non-zero element $\alpha$ in $R$ and an ideal $J$ of $R$ such that $AJ=(\alpha)$.  Consider $\frac{1}{\alpha}BJ$.  Note that $\frac{1}{\alpha}BJ$ is an ideal of $R$ since $B\subseteq A$.  Further $A(\frac{1}{\alpha}BJ)=B(\frac{1}{\alpha}AJ)=B(\frac{1}{\alpha}(\alpha))=B$.

\textbf{Theorem 46:} Every non-zero proper ideal in a Dedekind domain $R$ can be written as a product of prime ideals of $R$ and this representation as a product is unique up to ordering. \\
\pf We first prove existence.

Let $S$ be the set of non-zero proper ideals which cannot be written as a product of prime ideals.  Since $R$ is a Dedekind domain $S$ has a maximal element $M$.  Note that $M$ is contained in a maximal ideal of $R$ which, since $R$ is a Dedekind domain, is a prime ideal of $R$, say $P$.

Thus $M\subseteq P$.  Note $M\neq P$ since $M$ is in $S$.  Thus $M\subsetneq P$.  Therefore by Corollary~45 there exists an ideal $A$ such that
\[ M = PA . \]
Further $M\subsetneq A$.  But $A$ is not a product of prime ideals since otherwise by $*$ $M$ is a product of prime ideals.  But then $A\in S$ and $M$ is not maximal in $S$ which is a contradiction.  Therefore $S$ is empty as required.

``Uniqueness'' \\
Suppose that $p_1$, $\dotsc$, $p_r$ and $q_1$, $\dotsc$, $q_s$ are prime ideals with
\[ p_1\dotsm _r = q_1\dotsm q_s . \]
Note that $p_1\mid q_1\dotsm q_s$.  Thus by Corollary~45, $p_1\supseteq q_1\dotsm q_s$.  Since $p_1$ is a prime ideal $p_1\supseteq q_i$ for some $i$.  Without loss of generality we may suppose $p_1\supseteq q_1$.  Prime ideals are maximal ideals in $R$ so $p_1=q_1$.  By Corollary~44, $p_2\dotsm p_r=q_2\dotsm q_s$.  Repeating this argument the result follows.

\remark Let $[K:\Q]<\infty$.  Then $\A\cap K$ is a Dedekind domain and so we have unique factorization into prime ideals, up to ordering, in $\A\cap K$.

\defn Let $R$ be a commutative ring with identity.  An element $c$ of $R$ is said to be irreducible of $R$ if
\begin{enumerate}
\item $c\neq0$ and $c$ is not a unit of $R$.
\item If $c=ab$ with $a$, $b$ in $R$ then $a$ is a unit or $b$ is a unit.
\end{enumerate}
An element $c$ of $R$ is said to be a prime of $R$ if
\begin{enumerate}
\item $c\neq0$ and $c$ is not a unit of $R$
\item If $c\mid ab$ with $a$, $b$ in $R$ then $c\mid a$ or $c\mid b$.
\end{enumerate}
Note in UFDs the concepts are the same.