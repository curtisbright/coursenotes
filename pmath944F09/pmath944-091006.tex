Corrections: Addition of absolute values to $\abs{\det(C)}$ in Theorem 12.
\[ (A^{(n)}_\lambda)^+ = \set{(x_1,\dotsc,x_n)\in\R^n}{x_i\geq0, \quad i=1\c \dotsc\c n, \quad x_1+\dotsb+x_n\leq\lambda} \]

\textbf{Proposition 13:} \ldots \\
Every vector in $\Lambda$ is equivalent to a vector of the form $q_1\v_1+\dotsb+q_n\v_n$ with $0\leq q_i<a_{ii}$ for $i=1\c \dotsc\c n$.

Finally we should show that all vectors of the above form are inequivalent.  So suppose two are equivalent, their difference $r_1\v_1+\dotsb+r_n\v_n$ is in $\Lambda_1$ with $\abs{r_i}<a_{ii}$ for $i=1\c \dotsc\c n$.  Let $j$ be the largest integer for which $r_j\neq0$.  Then we replace $\w_j$ in the basis $\w_1\c \dotsc\c\w_n$ of $\Lambda_1$ by $\w_j$ minus a multiple of $r_1\v_1+\dotsb+r_n\v_n$ so that the %coordinate $a_{jj}$ of
resulting basis is in lower triangular form but with $a_{jj}$ replaced by a smaller non-negative integer.  The final reduction (to Hermite normal form) doesn't change the diagonal.  But the resulting determinant is different which gives a contradiction.  Therefore the index is $\prod_{i=1}^n a_{ii}$.

Let $A$ be a convex subset of $\R^n$ which is symmetric about the origin and of finite volume.  Let $\Lambda$ be a lattice in $\R^n$.  Minkowski introduced the successive minima $\lambda_1\c \dotsc\c \lambda_n$ associated with $\Lambda$ and $A$ by putting
\[ \lambda_j = \inf\set{\lambda\in\R}{\text{$\lambda A$ contains $j$ linearly independent vectors of $\Lambda$}} . \]

Then
\[ 0 < \lambda_1 \leq \lambda_2 \leq \dotsb \leq \lambda_n < \infty . \]
Minkowski, in what is known as Minkowski's Second Theorem on Convex Bodies proved that
\[ \frac{2^n d(\Lambda)}{n!} \leq \lambda_1\dotsm\lambda_n\mu(A) \leq 2^n d(\Lambda) . \footnote{upper bound implies $\lambda^n_1\mu(A)\leq2^n d(\Lambda)$ $\implies$ Minkowski's Convex Body Theorem} \]%\marginpar{upper bound implies $\lambda^n_1\mu(A)\leq2^n d(\Lambda)$ $\implies$ Minkowski's Convex Body Theorem}%
We won't give a proof: the upper bound is tricky.

The result is sharp in the sense that neither the upper bound or the lower bound can be improved in general.

Take any positive real numbers $\gamma_1\c \dotsc\c \gamma_n$ with $0<\gamma_1\leq\gamma_2\leq\dotsb\leq\gamma_n<\infty$.  Consider the lattice generated by
\[ (\gamma_1,0,\dotsc,0)\c  (0,\gamma_2,0,\dotsc,0)\c  \dotsc\c  (0,\dotsc,0,\gamma_n) . \]
Let $A$ be the cube $A=\set{(x_1,\dotsc,x_n)\in\R^n}{\abs{x_i}\leq1,\quad i=1\c \dotsc\c n}$.  ``Plainly'' $\lambda_i(A,\Lambda)=\lambda_i=\gamma_i$ for $i=1\c \dotsc\c n$.  Further $d(\Lambda)=\gamma_1\dotsm\gamma_n$.  Thus
\[ \lambda_1\dotsm\lambda_n\mu(A) = \gamma_1\dotsm\gamma_n 2^n = 2^n d(\Lambda) , \]
so the upper bound is sharp.

If we now take $A=A^{(n)}_1=\set{\x\in\R^n}{\abs{x_1}+\dotsb+\abs{x_n}\leq1}$ then
\[ \lambda_i(A,\Lambda) = \lambda_i = \gamma_i \text{ for $i=1\c \dotsc\c n$ as before} . \]
We have
\[ \lambda_1\dotsm\lambda_n\mu(A) = \gamma_1\dotsm\gamma_n\frac{2^n}{n!} = \frac{2^n}{n!}d(\Lambda) \]
and so the lower bound is sharp.

Sometimes it is useful to have another characterization of a lattice.

\pagebreak
\textbf{Theorem 14:} A subset $\Lambda$ of $\R^n$ is a lattice in $\R^n$ if and only if
\begin{enumerate}
\item[i)] If $\a\c \b$ are in $\Lambda$ then $\a+\b$ and $\a-\b$ are in $\Lambda$.
\item[ii)] $\Lambda$ contains $n$ linearly independent points $\a_1\c \dotsc\c \a_n$.
\item[iii)] $\Lambda$ is a discrete set, in other words it has no limit points.
\end{enumerate}

\pf ($\Longrightarrow$) Follows immediately from the definition of a lattice.

($\Longleftarrow$) We prove this by induction on $n$.  For $n=1$ we note by ii) that $\Lambda$ contains a non-zero point $a$.  By i) $\Lambda$ contains $0$ and $-a$.  Further since $\Lambda$ is discrete there is a smallest positive real number $a$ in $\Lambda$.  Then by i)
\[ \Lambda = \set{ga}{g\in\Z} \]
as required.

Suppose the result holds for dimension $n-1$ with $n\geq2$.  We may choose our coordinate system in $\R^n$ so that $n-1$ linearly independent points of $\Lambda$ lie in a subspace of the form $\R^{n-1}\times\brace0$ so $x_n=0$.  Then $\Lambda'=\Lambda\cap\R^{n-1}\times\brace0$ projects down to a subset of $\R^{n-1}$ which is a lattice by our inductive hypothesis.  Let $\b_1\c \dotsc\c \b_{n-1}$ be a basis for $\Lambda'$.  Then $\Lambda$ contains a point of the form $\b_n=(b_{1n},\dotsc,b_{nn})$ with $b_{nn}>0$.  In fact there is a point $\b_n$ of this form with $b_{nn}$ minimal.  Suppose otherwise.  Then we can find a sequence $\b^{(j)}_n=(b^{(j)}_{1n},\dotsc,b^{(j)}_{nn})$ in $\Lambda$ with $b^{(j)}_{nn}>0$ and
\[ b^{(j)}_{nn} \to 0 \text{ as } j\to\infty . \]
But we can translate $\b^{(j)}_{n}$ by some linear combination of $\b_1\c \dotsc\c \b_{n-1}$ so that $(b^{(j)}_{1,n},\dotsc,b^{(j)}_{n-1,n},0)$ are in the compact set
\[ \set{\lambda_1\b_1 + \dotsb + \lambda_{n-1}\b_{n-1}}{\abs{\lambda_i}\leq1} \]
so thus the $\b^{(j)}_n$s are all in a compact set and so have a limit point contradicting the fact that $\Lambda$ is discrete.  We now claim that every element of $\Lambda$ is an integer linear combination of $\b_1\c \dotsc\c \b_n$.  Let $\D\in\Lambda$ with $\D=(d_1,\dotsc,d_n)$. Then
%\[ \D' = \D - \brack*{\frac{d_n}{b_{nn}}} \b_n \in \Lambda . \]
\[ \D' = \D - \floor*{\frac{d_n}{b_{nn}}} \b_n \in \Lambda . \]
The $n$th coordinate of $\D'$ is non-negative and smaller than $b_{nn}$.  Therefore it is $0$.  Thus $\D'\in\Lambda'$ and so is an integer linear combination of $\b_1\c \dotsc\c \b_{n-1}$.  Therefore $\D$ is an integer linear combination of $\b_1\c \dotsc\c \b_n$.  Thus $\Lambda$ is a lattice basis $\b_1\c \dotsc\c \b_n$ and the result follows.

\textbf{Proposition 15:}  Let $n\c m\c k_1\c \dotsc\c k_m$ be positive integers and let $a_{ij}$, $i=1\c \dotsc\c m$, $j=1\c \dotsc\c n$ be integers.  The set $\Lambda$ of points $\u=(u_1,\dotsc,u_n)$ in $\R^n$ with integral coordinates satisfying
\[ \sum_{j=1}^n a_{ij}u_j \equiv 0 \pmod{k_i} \quad\text{for $i=1\c \dotsc\c m$} \]
is a lattice in $\R^n$ with $d(\Lambda)\leq k_1\dotsm k_m$.
