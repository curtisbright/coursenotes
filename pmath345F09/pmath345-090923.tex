\textbf{Euler's Theorem:} $a\in\Z$, $a\neq0$, $\gcd(a,n)=1$, then $a^{\phi(n)}\equiv1\pmod n$ \\
$\phi(n)=\text{\# of nonnegative integers $<n$ that are coprime with $n$}$

Need
\lem $R$ commutative ring, with a \emph{finite} set of units, say $m$ of them.  Then if $a\in R$ is a unit then $a^m\footnote{$\underbrace{a\cdot a\cdot a\dotsm a}_{\text{$m$ times}}$}=1$.

\pf $a$ a unit.  Consider $f_a\colon R\to R$ by $b\mapsto ab$.  Since $a$ is not a zero divisor, $f_a$ is injective. \\
Note that the product of units is a unit.  \\
If $U=\text{set of units in $R$}=\brace{u_1,u_2,\dotsc,a_m}$, then $f_a(U)=U$. \\
i.e., $f_a|_U\colon U\to U$ injective, hence bijective since $U$ is finite. \\
$U=\brace{u_1,\dotsc,u_m}$ \\
$U=f_a(U)=\brace{au_1,au_2,\dotsc,au_m}$ \\
$\brace{u_1,\dotsc,u_m}=\brace{au_1,\dotsc,au_m}$, so
\begin{align*}
\prod_{i=1}^m u_i\footnotemark = \prod_{i=1}^m au_i &= (au_1)(au_2)\dotsm(au_m) \\
&= a^m(u_1u_2\dotsm u_m) \\
&= a^m\prod_{i=1}^m u_i
\end{align*}\footnotetext{$u_1u_2\dotsm u_m$}%
Therefore $1\prod_i=a^m\prod_i u_i$.  Since $\prod_i u_i$ is also a unit it is not a zero divisor and hence we can cancel $\implies 1=a^m$.

\textbf{Proof of Euler's theorem:} \\
$n\geq2$, $a\neq0$, $\gcd(a,n)=1$. \\
$R=\Z_n$. \\
$U=\text{set of units in $\Z_n$}$ has $\phi(n)$ many elements in it by the previous propositions. \\
$\overline{b}$ is a unit in $\Z_n$ $\iff$ $\gcd(b,n)=1$ \\
$\Z_n=\brace{\overline0,\overline1,\dotsc,\overline{n-1}}=\text{\# of units}=\phi(n)$ \\
$\overline a\in\Z_n$ is a unit. \\
\# of units in $\Z_n$ is $\phi(n)$ so by the lemma
\begin{align*}
\overline a^{\phi(n)} &= \overline1\footnotemark \\
\implies \overline{a^{\phi(n)}} &= \overline1 \\
\implies a^{\phi(n)} &\equiv 1 \pmod n
\end{align*}\footnotetext{in $\Z_n$}%
What are the units/zero divisors in $\Z[i]=\set{a+bi}{a,b\in\Z}$? \\
zero divisors: \emph{none}. \\
$\Z[i]$ is a subring of $\C$ and $\C$ have no zero divisors. \\
($u,v\in\C$, $uv=0\implies u=0$ or $v=0$, i.e., $\C$ is an integral domain) \\
\textbf{units:} units in $\C$ are $\C\setminus\brace0$ (i.e., $\C$ is a field.) \\
$*$ This does \emph{not} mean that $\Z[i]$ is a field.  Example: $2$ is a unit in $\Q$ but not in $\Z$.

\textbf{units:} $\pm1$, $\pm i$ \\
\textbf{claim:} these are the only units \\
\pf $z\in\Z[i]$, $z=a+bi$ \\
$\abs{z}=\sqrt{a^2+b^2}$ \\
$N(z)=\abs{z}^2=a^2+b^2\in\Z$ \\
$z,w\in\Z$, $N(zw)=N(z)N(w)$ \\
If $z$ is a unit in $\Z[i]$, let $w=z^{-1}\in\Z[i]$, \\
$1=zw\implies N(1)\footnote{1}=N(zw)=N(z)N(w)$ \\
$N(w)=N(z)^{-1}$, \\
i.e., $N(z)$ is a \emph{unit} in $\Z$. \\
$\implies$ $N(z)=\pm1$ \\
$\implies$ $a^2+b^2=\pm1$ \\
$\implies$ $a^2+b^2=1$ \\
$\implies$ $a=\pm1$ and $b=0$ \\
or \\
$a=0$ and $b=\pm1$ \\
$z=1,-1,i,-i$

\textbf{Exercise:} $\Fun([0,1],\R)$.  What are the zero-divisors and the units?

\textbf{Polynomials:} \\
\defin $R$ commutative ring.  Let $x$ be an indeterminate (i.e., a variable), i.e., $x$ is just a symbol.  A \emph{polynomial in $x$ over $R$} is a formal expression\footnote{formal expression means it is just a string of symbols} of the form
\[ a_0 + a_1 x + a_2 x^2 + a_3 x^3 + \dotsb \]
where $a_i$s are in $R$ \emph{and} all but finitely many of the $a_i$s are $0$.
\[ a_0 + a_1 x + a_2 x^2 + \dotsb = b_0 + b_1 x + b_2 x^2 + \dotsb \]
if and only if each $a_i=b_i$ in $R$.

\textbf{Notational conventions:}
\begin{enumerate}
\item We use series notation:
\[ a_0 + a_1 x + a_2 x^2 + \dotsb =: \sum_{i=0}^\infty a_i x^i \]
\item We often \emph{drop} the $a_ix^i$ if $a_i=0$. \\
So for example when $R=\Z$, we write:
\[ x^2 - 2x^4 + x^6 \]
rather than
\[ 0 + 0x + 1x^2 + 0x^3 + (-2)x^4 + 1x^6 + 0x^7 + 0x^8 + \dotsb \]
\item we also write $x^2-2x^4$ instead of $x^2+(-2)x^4$
\end{enumerate}
Let $R[x]$ denote the \emph{set} of all polynomials in $x$ over $R$.

\textbf{Check:} $R[x]$ is a ring with
\begin{align*}
0 &= \sum_{i=1}^\infty 0x^i \\
1 &= 1 + 0x+ 0x^2 + \dotsb
\end{align*}
\[ \paren[\bigg]{\sum_i a_ix^i} + \paren[\bigg]{\sum_i b_ix^i} \coloneqq \sum_{i=0}^\infty (a_i+b_i)\footnote{in $R$}x^i \]
\[ \paren[\bigg]{\sum_i a_ix^i}\paren[\bigg]{\sum_i b_ix^i} \coloneqq \sum_{i=0}^\infty\paren[\bigg]{\sum_{j=0}^\infty a_{i-j}\footnote{in $R$}b_j}x^i \]