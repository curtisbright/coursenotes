Let us consider the lattice of integer points in $\R^n$ denoted by $\Lambda_0$.  The diagram is:
\[ \overbrace{\mathstrut{\bullet \quad \bullet \quad \bullet \quad \cdots \quad \bullet}}^\text{$n$ times} \]
We have $d(\Lambda_0)=1$.  The vectors of minimal non-zero length in $\Lambda_0$ are $(\pm1,0,\dotsc,0)\c \dotsc\c (0,\dotsc,0,\pm1)$ and they are of length 1.  Thus the lattice packing associated with $\Lambda_0$ consists of spheres of radius $\frac12$ around each integer point.  Thus the packing density is
\[ \frac{\pi^{n/2}}{\Gamma(1+n/2)} \paren*{\frac{1}{2}}^n . \]
In $\R^2$, it is $\frac\pi4=0.785\ldots\,$, in $\R^3$ it is $\frac\pi6=0.529\ldots\,$, in $\R^4$, $\frac{\pi^2}{32}=0.308\ldots\,$.  The kissing number associated with $\Lambda_0$ is $2n$.

The lattice $A_3$ associated with \setlength{\unitlength}{0.5cm}\phantom{$|$}\begin{picture}(2,0)(0,-0.25)\multiput(0,0)(1,0){3}{\circle*{0.2}}\put(0,0){\line(1,0){2}}\end{picture}\phantom{$|$} %.-.-.
may also be associated with \setlength{\unitlength}{0.5cm}\phantom{$\Bigg|$}\begin{picture}(0.707,0)(0,-0.25)
\put(0,0.707){\circle*{0.2}}
\put(0,-0.707){\circle*{0.2}}
\put(0.707,0){\circle*{0.2}}
\put(0,-0.707){\line(1,1){0.707}}
\put(0,0.707){\line(1,-1){0.707}}
\end{picture}\phantom{$\Bigg|$} %:>.,
which we call $D_3$.  For $n=3\c 4\c \dotsc$ we denote by $D_n$ the lattice associated with
%\[ :>.-.-. \dotsc .-. \]
\setlength{\unitlength}{0.75cm}
\[ \begin{picture}(6.707,2)(0,-1)
\put(0,0.707){\circle*{0.2}}
\put(0,-0.707){\circle*{0.2}}
\put(0.707,0){\circle*{0.2}}
\put(1.707,0){\circle*{0.2}}
\put(2.707,0){\circle*{0.2}}
\put(3.707,0){\circle*{0.2}}
\put(4.707,0){\makebox(0,0){$\cdots$}}
\put(5.707,0){\circle*{0.2}}
\put(6.707,0){\circle*{0.2}}
\put(0.707,0){\line(1,0){3}}
\put(5.707,0){\line(1,0){1}}
\put(0,-0.707){\line(1,1){0.707}}
\put(0,0.707){\line(1,-1){0.707}}
\end{picture} \]
We can represent $D_n$ as the sublattice of $\Lambda_0$ given by
\[ \set{(x_1,\dotsc,x_n)\in\Z^n}{x_1+\dotsb+x_n\equiv0\pmod2} . \]
The lattice is generated by elements of length $\sqrt2$, which is the minimal non-zero distance between vectors in the lattice.  We take:
%figure
\setlength{\unitlength}{2cm}
\[ \begin{picture}(4.707,2.5)(0,-1.25)
\put(0,0.707){\circle*{0.1}}
\put(0,-0.707){\circle*{0.1}}
\put(0.707,0){\circle*{0.1}}
\put(1.707,0){\circle*{0.1}}
\put(2.707,0){\large\makebox(0,0){$\cdots$}}
\put(3.707,0){\circle*{0.1}}
\put(4.707,0){\circle*{0.1}}
\put(0.707,0){\line(1,0){1}}
\put(3.707,0){\line(1,0){1}}
\put(0,-0.707){\line(1,1){0.707}}
\put(0,0.707){\line(1,-1){0.707}}
%\put(0.707,0.1){\line(0,1){0.3}}
%\put(0.4,0.5){\small{$(0,1,1,0,\dotsc,0)$}}
%\put(0.55,0){\small\makebox(0,0)[r]{$(0,1,1,0,\dotsc,0)$}}
\put(0,0){\small\makebox(0,0){$(0,1,1,0,\dotsc,0)$}}
\put(0,0.957){\small\makebox(0,0){$(1,1,0,\dotsc,0)$}}
\put(0,-0.957){\small\makebox(0,0){$(-1,1,0,\dotsc,0)$}}
\put(1.707,0.25){\small\makebox(0,0){$(0,0,1,1,0,\dotsc,0)$}}
\put(4.707,0.25){\small\makebox(0,0){$(0,\dotsc,0,1,1)$}}
\end{picture} \]
Thus
\[ d(D_n(\sqrt2)) = \abs*{\det\begin{pmatrix}
-1 & 1 & 0 & 0 & \cdots & 0 \\
1 & 1 & 0 & 0 & \cdots & 0 \\
0 & 1 & 1 & 0 & \cdots & 0 \\
0 & 0 & 1 & 1 & & 0 \\
\vdots & \vdots & & \ddots & \ddots & \\
0 & 0 & \cdots & 0 & 1 & 1
\end{pmatrix}} = \abs{-1\cdot1-1\cdot1} = 2 \]
The kissing number associated with the lattice $D_n(\sqrt2)$ corresponds to the number of non-zero vectors of minimal length, so it is %[two non-zero entries in vector, each of which is $1$ or $-1$]
$4\cdot\binom{n}{2}=2n(n-1)$.  We have a central sphere 
%I came in here
around $(0,\dotsc,0)$ of radius $\frac12\sqrt2=\frac{1}{\sqrt2}$ and it is touched by the $2n(n-1)$ non-overlapping spheres of radius $\frac{1}{\sqrt2}$ around $(\pm1,\pm1,0,\dotsc,0)$, $\dotsc$, $(0,\dotsc,0,\pm1,\pm1)$.  Put spheres of radius $\frac{1}{\sqrt2}$ around each lattice point to give a sphere packing.  The sphere packing density
\[ \Delta(D_n(\sqrt2)) = \frac{\pi^{n/2}}{2\Gamma(1+n/2)2^{n/2}} . \]
Note that
\[ \Delta(D_3) = \frac{\pi}{\sqrt{18}} = 0.7405\ldots \]
The sphere packing of $D_3$ corresponds to the cannonball packing.  In 1831 Gauss proved that $\Delta_3(L)=\Delta(D_3)$, that is to say that $D_3$ gives the sphere packing associated with a lattice of maximal density.

Kepler conjectured that $\Delta_3=\Delta_3(L)=\Delta(D_3)$, or equivalently that the most efficient packing of spheres in $\R^3$ is given by the cannonball packing.  In 1958 Rogus proved $\Delta_3\leq0.7796$ and in 1983 Lindsay proved $\Delta_3\leq0.7784$.  (In 1993 Hsiang claimed a proof that $\Delta_3=\Delta(D_3)$ and his ``proof'' appeared in a 92 page paper in the International Journal of Mathematics.)  Hales in 2005 in a 120 page paper in the Annals of Math%.  He
\ gave a proof of Kepler's conjecture.  It depended on a massive amount of computation and this part of the argument is very hard to check.

Consider the kissing number problem in $\R^3$.  Three spheres touching in $\R^3$: 
\setlength{\unitlength}{0.75cm}
\[ \begin{picture}(4,3.73)(-2,-1)
\put(0,0){\circle{2}}
\put(1,1.73){\circle{2}}
\put(-1,1.73){\circle{2}}
\put(0,0){\circle*{0.1}}
\put(1,1.73){\circle*{0.1}}
\put(-1,1.73){\circle*{0.1}}
\path(0,0)(-1,1.73)(1,1.73)(0,0)
\arc{0.75}{4.19}{5.24}
\put(0,0.6){\tiny\makebox(0,0){$\frac{\pi}{3}$}}
%\put(0,0){\line(3,5){1}}
%\put(0,0){\line(-3,5){1}}
%\put(-1,1.73){\line(1,0){2}}
\end{picture} \] %figure
The centres of the spheres form an equilateral triangle.  Given a configuration of spheres of radius $1$ touching a central sphere of radius $1$ we can associate to each sphere touching the central sphere a shadow or spherical cap determined by a cone of radius $\frac\pi3$ from the origin.
\setlength{\unitlength}{2cm}
\[ \begin{picture}(2,2)(-1,-1)
\put(0,0){\circle{2}}
\put(0,0){\circle*{0.05}}
\path(0,0)(0.5,0.865)
\path(0,0)(-0.5,0.865)
\path(0.5,0.865)(0,0.865)
\path(0.25,0.865)(0.5,1)
\put(0,0.6){\tiny\makebox(0,0)[r]{$\frac{\sqrt3}{2}$}}
\put(0.5,1){\tiny\makebox(0,0)[l]{$\frac12$}}
\put(0,0){\line(0,1){1}}
\put(0,0){\arc{0.37}{4.19}{5.24}}
\put(0,0.865){\ellipse{1}{0.1}}
\put(0,-0.15){\small\makebox(0,0){$(0,0,\dotsc,0)$}}
\put(0.07,0.28){\tiny\makebox(0,0){$\frac{\pi}{3}$}}
%\put(0,0){\ellipse{2}{0.2}}
\end{picture} \]
%figure
The surface area of the shadow is $2\pi h$ where $h$ is the height of the spherical cap.  Here $h=1-\frac{\sqrt3}{2}$ so the area is $(2-\sqrt3)\pi$.  The total surface area of the sphere is $4\pi$ and so the kissing number $\tau_3$ is at most $\frac{4\pi}{(2-\sqrt3)\pi}=8+4\sqrt3<15$.  Thus $\tau_3\leq14$.  The packing associated with $D_3$ gives $\tau_3\geq12$.

In fact $\tau_3=12$ as was first proved in 1953 by Schutte and van der Waerden.  The following kissing numbers are known: $\tau_1=2$, $\tau_3=6$, $\tau_3=12$, $\tau_4=24$, $\tau_8=240$ and $\tau_{24}=196{,}560$.  How do we find such results?

The arguments depend on linear programming and the study of positive semidefinite functions on the sphere $S^{n-1}$ in $\R^n$.

Let $\brace{\x_1,\dotsc,\x_m}$ be points on $S^{n-1}$ in $\R^n$.  Thus $\x_i\cdot\x_i=1$ for $i=1\c \dotsc\c m$ and $\x_i\in\R^n$.
\[ S^{n-1} = \set{(x_1,\dotsc,x_n)\in\R^n}{x_1^2+\dotsb+x_n^2=1} . \]
Let $\theta_{ij}$ be the distance between $\x_i$ and $\x_j$ on the surface of $S^{n-1}$, so the length of the geodesic between $\x_i$ and $\x_j$.
\setlength{\unitlength}{1cm}
\[ \begin{picture}(2,2)(-1,-1)
\put(0,0){\circle{2}}
\put(0,0){\circle*{0.05}}
\path(0,0)(0.5,0.865)
\path(0,0)(-0.5,0.865)
\arc{0.5}{4.19}{5.24}
\put(0,0.5){\small\makebox(0,0){$\theta_{ij}$}}
\put(0.5,0.865){\makebox(0,0)[lb]{$\x_j$}}
\put(-0.5,0.865){\makebox(0,0)[rb]{$\x_i\vphantom{\x_j}$}}
\end{picture} \] %figure
It is just the angle in radians determined by the points.

Notice that for any real numbers $t_1\c \dotsc\c t_m$ we have
\begin{align*}
\norm{t_1\x_1+\dotsb+t_m\x_m}^2 &= (t_1\x_1+\dotsb+t_m\x_m,t_1\x_1+\dotsb+t_m\x_m) \\
&= \sum_{i=1}^m \sum_{j=1}^m t_i t_j \cos(\theta_{ij}) \geq 0
\end{align*}
Equivalently the matrix
\[ (\cos(\theta_{ij}))_{\substack{i=1,\dotsc,n\\j=1,\dotsc,n}} \]
is positive semidefinite.
%(Odlyzko and Sloan 1979)
